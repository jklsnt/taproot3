% Created 2022-06-19 Sun 13:41
% Intended LaTeX compiler: xelatex
\documentclass[letterpaper]{article}
\usepackage{graphicx}
\usepackage{grffile}
\usepackage{longtable}
\usepackage{wrapfig}
\usepackage{rotating}
\usepackage[normalem]{ulem}
\usepackage{amsmath}
\usepackage{textcomp}
\usepackage{amssymb}
\usepackage{capt-of}
\usepackage{hyperref}
\usepackage[margin=1in]{geometry}
\setlength{\parindent}{0pt}
\usepackage[margin=1in]{geometry}
\usepackage{fontspec}
\usepackage{svg}
\usepackage{tikz}
\usepackage{cancel}
\usepackage{pgfplots}
\usepackage{indentfirst}
\setmainfont[ItalicFont = HelveticaNeue-Italic, BoldFont = HelveticaNeue-Bold, BoldItalicFont = HelveticaNeue-BoldItalic]{HelveticaNeue}
\newfontfamily\NHLight[ItalicFont = HelveticaNeue-LightItalic, BoldFont       = HelveticaNeue-UltraLight, BoldItalicFont = HelveticaNeue-UltraLightItalic]{HelveticaNeue-Light}
\newcommand\textrmlf[1]{{\NHLight#1}}
\newcommand\textitlf[1]{{\NHLight\itshape#1}}
\let\textbflf\textrm
\newcommand\textulf[1]{{\NHLight\bfseries#1}}
\newcommand\textuitlf[1]{{\NHLight\bfseries\itshape#1}}
\usepackage{fancyhdr}
\usepackage{csquotes}
\pagestyle{fancy}
\usepackage{titlesec}
\usepackage{titling}
\makeatletter
\lhead{\textbf{\@title}}
\makeatother
\rhead{\textrmlf{Written} \today}
\lfoot{\theauthor\ \textbullet \ \textbf{2021-2022}}
\cfoot{}
\rfoot{\textrmlf{Page} \thepage}
\renewcommand{\tableofcontents}{}
\titleformat{\section} {\Large} {\textrmlf{\thesection} {|}} {0.3em} {\textbf}
\titleformat{\subsection} {\large} {\textrmlf{\thesubsection} {|}} {0.2em} {\textbf}
\titleformat{\subsubsection} {\large} {\textrmlf{\thesubsubsection} {|}} {0.1em} {\textbf}
\setlength{\parskip}{0.45em}
\renewcommand\maketitle{}
\author{Dylan Wallace}
\date{\today}
\title{Rotational Dynamics Theorem: Torque and Angular Momentum 5}
\hypersetup{
 pdfauthor={Dylan Wallace},
 pdftitle={Rotational Dynamics Theorem: Torque and Angular Momentum 5},
 pdfkeywords={},
 pdfsubject={},
 pdfcreator={Emacs 28.0.91 (Org mode 9.4.6)}, 
 pdflang={English}}
\begin{document}

\maketitle
\tableofcontents


\section{Problem 1)}
\label{sec:org811da3d}
We will take the liberty of claiming that \(g &= 10 ms^{-2}\) for the simple reason that I am lazy and that we are dealing with objects that we can hold in our hands, hypothetically, and therefore this level of precision is adequate.
\subsection{a)}
\label{sec:org663a9a7}
We know the force of friction that is acting on the cylinder. We can find the friction coefficient by solving for the normal and dividing it from the force of friction.
We know the mass of the cylinder, so we know the force of gravity. We also know the angle of the ramp. Therefore, we can calculate the normal force as being

\begin{aligned}
F_{N} &= Mg\cos{(\theta)} \\
\end{aligned}

Therefore,

\begin{aligned}
\mu \ge \frac{F_f}{F_N} &= \frac{F_f}{Mg\cos{(\theta)}}
\end{aligned}

We plug in our values:

\begin{aligned}
\mu \ge \frac{2N}{1kg \cdot 9.8ms^{-2}\cdot \cos{(30^{\circ})}} \\
&= \frac{2N}{10N \cdot \frac{\sqrt{3}}{2}} \\
&= \frac{2N}{10N} \cdot \frac{2}{\sqrt{3}} \\
&= \frac{4\sqrt{3}}{10\cdot 3} \\
&= \frac{2\sqrt{3}}{15} \\
\end{aligned}
\subsection{b)}
\label{sec:orgb369050}
We can find the linear acceleration of the CM by summing all the forces and dividing by the mass.
The forces acting on the cylinder are those of gravity, normal force, and friction. Note that the force of friction applies in the opposite direction as the direction of the sum of gravity and normal force. Due to the normal force, we can disregard any force perpendicular to the ramp:

\begin{aligned}
F_{net} &= F_{g,ramp} - F_{f} \\
F_{g,ramp} &= -F_{g}\sin{(\theta)} \\
&= gM\sin{(\theta)} \\
F_{net} &= gM\sin{(\theta)} - F_{f} \\
\end{aligned}

We know that \(F &= ma\), so

\begin{aligned}
a_{ramp} &= \frac{F_{net}}{M} \\
&= \frac{gM\sin{(\theta)} - F_{f}}{M} \\
&= g\sin{(\theta)} - \frac{F_{f}}{M} \\
\end{aligned}

Plugging in values:
\begin{aligned}
a_{ramp} &= 10 ms^{-2} \sin{(30^{\circ})} - \frac{2.0N}{1.0 kg} \\
&= 5ms^{-2} - 2ms^{-2} \\
&= 3ms^{-2} \\
\end{aligned}

\subsection{c)}
\label{sec:orgab6cc83}
We've established in a different problemset that the following is true:

\begin{aligned}
\vec{\tau}_{net}' &= I_{CM}\vec{\alpha}' \\
\vec{\alpha}' &= \frac{\vec{\tau}_{net}'}{I_{CM}} \\
\end{aligned}

We know the rotational inertia. We can solve for the torque of the cylinder and divide by inertia to find acceleration.
We can find the torque by adding up all torques by all forces. To find the torque, we define a coordinate system where \(\hat{z}\) is pointing out of the page. We also consider "right" to be \(\hat{x}\) and "up" to be \(\hat{y}\).

To find net torque, we add up all torque components induced by every force. The forces acting on the object are gravity and friction. But gravity is a force applied to the center of mass of the object, so we can ignore it. As such, we just need to find the torque induced by friction, as that will be our net torque. We know that the direction of the force of friction is perpendicular to its position vector with respect to the center of mass, 

\begin{aligned}
\vec{\tau}_{f}' &= \vec{R} \times \vec{F}_{f} \\
&= -RF_{f}\hat{z} \\
\end{aligned}

Because friction acts in the opposite direction as the ball's acceleration, we know that the torque has to be "into" the page. As a result, we just get:

\begin{aligned}
\vec{\tau}'_{net} &= \vec{\tau}'_{f} &= -RF_{f} \hat{z} \\
\end{aligned}

As such,

\begin{aligned}
\vec{\alpha}' &= \frac{\vec{\tau}_{net}'}{I_{0}} \\
&= -\frac{RF_{f}}{I_{0}}\hat{z} \\
\end{aligned}

We plug in values:

\begin{aligned}
\vec{\alpha}' &= -\frac{(0.5)(2)}{(0.2)}\hat{z} \\
&= -5\hat{z} \\
&= 5 \frac{rad}{s^2}
\end{aligned}

We can verify this using intuition: the acceleration is in the negative z direction, or into the page, meaning that the ball is accelerating in the forwards direction as it rolls.

\subsection{d)}
\label{sec:org1d7d974}
If the cylinder slides, the friction is considered dynamic. If it does not, it is considered static.
If the cylinder slides, the acceleration calculated by the torque should be higher than the acceleration calculated from the linear acceleration. We can find this acceleration simply by dividing by the radius:

\begin{aligned}
\alpha &= \frac{a_{ramp}}{R} \\
&= \frac{3ms^{-2}}{0.5m} \\
&= 6\frac{rad}{s^2} \\
\end{aligned}

This value is more than the \(5 \frac{rad}{s^2}\) we got from the torque method, so we know that the cylinder doesn't slip, and the coefficient is static.

\subsection{e)}
\label{sec:org8f561e9}
To find the initial torque of the cylinder from the reference frame of the right vertex of the triangle, we need the distance from the vertex to the Center of Mass of the cylinder, as well as the net force acting on the cylinder. We will work in a different coordinate system from before, with \(\hat{x}\) being parallel to the ramp surface, and \(\hat{y}\) being perpendicular to it, both retaining their general direction. \(\hat{z}\) stays the same. We will keep the coordinates of the origins the same.

We know that the net force acting on the cylinder is the sum of the gravitational force, the normal force, and the force of friction.
We will list out each of these forces:

\begin{aligned}
\vec{F}_{g} &= Mg(\sin{(\theta)}\hat{x} - \cos{(\theta)}\hat{y})\\
\vec{F}_{N} &= Mg\cos{(\theta)}\hat{y}\\
\vec{F}_{f} &= -F_f\hat{x} \\
\end{aligned}

We know the net force acting on the cylinder on the center of mass to be the following:

\begin{aligned}
\vec{F}_{net} &= \vec{F}_{g} + \vec{F}_{N} + \vec{F}_{f} \\
&= (Mg\sin{(\theta)} - F_{f})\hat{x} \\
\end{aligned}

We now consider the length of the lever arm. To clarify the ambiguity, we will assume that the starting position of the cylinder is in such a position that the vector drawn from the origin to the center of mass of the cylinder is perpendicular to the surface of the ramp. In other words, the point that the cylinder touches the ramp, along with the origin and the other base point of the ramp, forms a right triangle. In that case,

\begin{aligned}
\vec{R}_{lever} &= (b\sin{(\theta)} + R)\hat{y} \\
\end{aligned}

Now, all that is left is to find the net torque. Recall that net torque is the sum of torque from forces around the center of mass and forces from the center of mass. Then, we must add the center of mass torque to the calculation:

\begin{aligned}
\vec{\tau}_{net} &= \vec{R}_{lever} \times \vec{F}_{net} + \vec{\tau}'_{net} \\
&= (b\sin{(\theta)} + R)\hat{y} \times (Mg\sin{(\theta)} - F_{f})\hat{x} - RF_{f}\hat{z}\\
&= -((b\sin{(\theta)} + R)(Mg\sin{(\theta)} - F_{f}) - RF_{f})\hat{z} \\
\end{aligned}

\subsection{f)}
\label{sec:org3358b2c}
We know that \(\vec{L}' &= I_{CM}\vec{\omega}'\). Therefore, we know that

\begin{aligned}
\vec{L}_{sys} &= \vec{R} \times M\vec{v}_{CM} + \sum \vec{r_{i}}' \times m_i \vec{v_{i}}' \\
&= \vec{R} \times M\vec{v}_{CM} + \vec{L}' \\
&= \vec{R} \times M\vec{v}_{CM} + I_{CM}\vec{\omega}' \\
\end{aligned}

We established above that the length of the position vector is \(b\sin{(\theta)} + R\). In addition, we know that the velocity of the center of mass is in the same direction as the net force, so the cross product will be in the direction \(-\hat{z}\). We get

\begin{aligned}
\vec{L}_{sys} &= -(b\sin{(\theta)} + R)Mv_{CM} \hat{z} + I_{CM} \vec{\omega}' \\
\end{aligned}

In fact, we already know that our angular velocity is in the direction \(-\hat{z}\) because from our frame the cylinder is rotating clockwise:

\begin{aligned}
\vec{L}_{sys} &= -((b\sin{(\theta)} + R)Mv_{CM} + I_{CM}\omega')\hat{z} \\
\end{aligned}

We take the time derivative:

\begin{aligned}
\frac{d\vec{L}}{dt} &= -\frac{d}{dt} (b\sin{(\theta)} + R)Mv_{CM}\hat{z} - \frac{d}{dt} I_{CM}\omega'\hat{z} \\
&= -((b\sin{(\theta)} + R)Ma_{CM} + I_{CM}\alpha')\hat{z} \\
\end{aligned}

\subsection{g)}
\label{sec:orgd9b695f}

We know that \uline{e} and \uline{f} are equivalent if we plug in \uline{b} and \uline{c}.

First are a few statements we know for a fact:

\begin{aligned}
a_{ramp} &= \frac{gM\sin{(\theta)} - F_{f}}{M} \\
I_{CM}\alpha &= \vec{\tau}'_{net} = -RF_{f}\\
\end{aligned}

We also know from \uline{e} and \uline{f}:

\begin{aligned}
\vec{\tau}_{net} &= -((b\sin{(\theta)} + R)(Mg\sin{(\theta)} - F_{f}) - RF_{f}) \\
\frac{d\vec{L}}{dt} &= -((b\sin{(\theta)} + R)Ma_{CM} + I_{CM}\alpha') \\
\end{aligned}

Then, we can start replacing:

\begin{aligned}
\frac{d\vec{L}}{dt} &= -((b\sin{(\theta)} + R)Ma_{CM} + I_{CM}\alpha') \\
&= -((b\sin{(\theta)} + R)M\cdot \frac{gM\sin{(\theta)} - F_{f}}{M} + I_{CM}\alpha') \\
&= -((b\sin{(\theta)} + R)(Mg\sin{(\theta)} - F_{f}) + I_{CM}\alpha')\\
&= -((b\sin{(\theta)} + R)(Mg\sin{(\theta)} - F_{f}) + (\vec{\tau}'_{net}))\\
&= -((b\sin{(\theta)} + R)(Mg\sin{(\theta)} - F_{f}) - RF_{f}) \\
&= \vec{\tau}_{net} \\
\end{aligned}
\end{document}