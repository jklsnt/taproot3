% Created 2022-05-05 Thu 23:58
% Intended LaTeX compiler: xelatex
\documentclass[letterpaper]{article}
\usepackage{graphicx}
\usepackage{longtable}
\usepackage{wrapfig}
\usepackage{rotating}
\usepackage[normalem]{ulem}
\usepackage{amsmath}
\usepackage{amssymb}
\usepackage{capt-of}
\usepackage{hyperref}
\usepackage[margin=1in]{geometry}
\setlength{\parindent}{0pt}
\usepackage[margin=1in]{geometry}
\usepackage{fontspec}
\usepackage{svg}
\usepackage{tikz}
\usepackage{cancel}
\usepackage{pgfplots}
\usepackage{indentfirst}
\setmainfont[ItalicFont = HelveticaNeue-Italic, BoldFont = HelveticaNeue-Bold, BoldItalicFont = HelveticaNeue-BoldItalic]{HelveticaNeue}
\newfontfamily\NHLight[ItalicFont = HelveticaNeue-LightItalic, BoldFont       = HelveticaNeue-UltraLight, BoldItalicFont = HelveticaNeue-UltraLightItalic]{HelveticaNeue-Light}
\newcommand\textrmlf[1]{{\NHLight#1}}
\newcommand\textitlf[1]{{\NHLight\itshape#1}}
\let\textbflf\textrm
\newcommand\textulf[1]{{\NHLight\bfseries#1}}
\newcommand\textuitlf[1]{{\NHLight\bfseries\itshape#1}}
\usepackage{fancyhdr}
\usepackage{csquotes}
\pagestyle{fancy}
\usepackage{titlesec}
\usepackage{titling}
\makeatletter
\lhead{\textbf{\@title}}
\makeatother
\rhead{\textrmlf{Written} \today}
\lfoot{\theauthor\ \textbullet \ \textbf{2021-2022}}
\cfoot{}
\rfoot{\textrmlf{Page} \thepage}
\renewcommand{\tableofcontents}{}
\titleformat{\section} {\Large} {\textrmlf{\thesection} {|}} {0.3em} {\textbf}
\titleformat{\subsection} {\large} {\textrmlf{\thesubsection} {|}} {0.2em} {\textbf}
\titleformat{\subsubsection} {\large} {\textrmlf{\thesubsubsection} {|}} {0.1em} {\textbf}
\setlength{\parskip}{0.45em}
\renewcommand\maketitle{}
\author{Houjun Liu}
\date{\today}
\title{MVC 2 PS\#27}
\hypersetup{
 pdfauthor={Houjun Liu},
 pdftitle={MVC 2 PS\#27},
 pdfkeywords={},
 pdfsubject={},
 pdfcreator={Emacs 28.0.91 (Org mode 9.5.2)}, 
 pdflang={English}}
\begin{document}

\maketitle
\tableofcontents


\section{Electric Change}
\label{sec:orgab846a4}
We are finally taking a surface integral! This is essentially multiplying the surface area of the shape of the function to the value of the function itself.

Firstly, taking the area \(dA\) by \(dV\):

\begin{align}
   dA &= \sqrt{1+\left(\frac{\partial z}{\partial x}\right)^2+\left(\frac{\partial z}{\partial y}\right)^2} \\
&= \sqrt{1+\left(3\right)^2+\left(2\right)^2} \\
&= \sqrt{14} 
\end{align}

Supplying the value into the function:

\begin{align}
   &\int_0^7 \int_0^{11} (3x+2y+7)\sqrt{14}\ dy\ dx\\
\Rightarrow &\sqrt{14}\int_0^7 \int_0^{11} (3x+2y+7)\ dy\ dx\\
\Rightarrow &\sqrt{14}\int_0^7 \left (3xy+y^2+7y)\right|_0^{11} \ dy\ dx\\
\Rightarrow &\sqrt{14} \left\left(\frac{33x^2}{2}+198x\right)\right|_0^7\\
\Rightarrow & \frac{4389\sqrt{14}}{2}
\end{align}

The charge, therefore, is proportional to \(\frac{4389\sqrt{14}}{2}\rho\).

\section{Infinite wire}
\label{sec:orgbbebccc}
Recall first that a semicircle with radius 7 can be defined as:

\begin{align}
   y &= \sqrt{7^2 - x^2} \\
&= \sqrt{49 - x^2}
\end{align}

Let's first figure the value of this function \(dA\):

\begin{align}
   dA &= \sqrt{1+\left(\frac{d}{dx}\sqrt{49-x^2}\right)^2} \\
&= \sqrt{1+\left(\frac{d}{dx}\sqrt{49-x^2}\right)^2}\\
&= \sqrt{1-\frac{x^2}{x^2-49}}
\end{align}

We will take the line integral of this function, and proceed to multiply by the value of \(xy\) at that point.

\begin{equation}
   \int_0^7 \int_0^7\ xy\ \sqrt{1-\frac{x^2}{x^2-49}}\ dx\ dy
\end{equation}

\begin{verbatim}
f(x,y) = x*y*sqrt(1-x^2/(x^2-49))
f.integrate(x, 0,7).integrate(y,0,7)
\end{verbatim}

\begin{verbatim}
2401/2
\end{verbatim}


Looks like the solution for the wire's weight is about \(\frac{2401}{2}\) grams.

\section{More Difficult Polar Coordinates}
\label{sec:org3c25cf7}
Recall that, to figure the unit sphere volume, we can convert an \(\mathbb{R}^2\to\mathbb{R}^1\) result into circular coordinates.

That, by pythagoras, \(x^2+y^2=r^2\). Therefore, the expression of:

\begin{equation}
   f(x,y) = \frac{1}{(x^2+y^2)^k} \Rightarrow f(r,\theta) = \frac{1}{r^{2k}}
\end{equation}

We also note that, due to the correction factor, \(dA = r\ dr\ d\theta\).

Taking the actual integral, therefore, will result in:

\begin{align}
&\int_0^{2\pi} \int_0^1 r^{-k}\ dr\ d\theta\\
\Rightarrow &\int_0^{2\pi} \lim_{x\to 0} \left(\frac{1}{-k+1}-\frac{1}{x^{k-1}}\frac{1}{-k+1}\right)\ d\theta
\end{align}

Evidently, when \(k\leq 1\), the second term would become infinity large.

Now, we essentially want to take this idea and expand it to \(n\) dimensions, to figure the correct spherical coordinates.

Turns out, the \href{https://math.stackexchange.com/questions/1482747/integral-in-n-dimensional-spherical-coordinates}{naïve version} of the \(n\) sphere integral is the same correction factor multiplied by \(\sin^{n-\{2\ldots(n-1)\}}\). Therefore, the same logic from above actually holds for \(n\) volcano as well: that, by very high dimension Pythagoras, \({x_1}^2 + {x_2}^2 + \ldots + {x_n}^2 = r^2\).

Therefore:

\begin{equation}
   \frac{1}{\left({x_1}^2 + {x_2}^2 + \cdots + {x_n}^2\right)^k}  = \frac{1}{r^{2k}}
\end{equation}

We will again note than the correction factor: \(dA = r^{n-1}\ sin^{n-2}\ sin^{n-3}\ dr\ d\theta_1\ \ldots\ d\theta_n\).

Therefore:

\begin{align}
    & \int \ldots \int_0^1 \frac{1}{r^{2k-n+1}}\ sin^{n-2}\ sin^{n-3}\ dr\ d\theta_1\ \ldots\ d\theta_n\\
\Rightarrow & \int \ldots \int_0^1 {r^{-2k+n-1}}\ sin^{n-2}\ sin^{n-3}\ dr\ d\theta_1\ \ldots\ d\theta_n\\
\Rightarrow & \int \ldots \int \left \frac{r^{-2k+n}}{-2k+n} \right|_0^1\ sin^{n-2}\ sin^{n-3}\ dr\ d\theta_1\ \ldots\ d\theta_n
\end{align}

At this point, we can analyze the solution. The first term will always be \(1\) over a certain value. The second term, however, is more interesting.

Case 1: \(-2k+n<0\). If the value of \(-2k+n <0\) , the \(r\) would have to be transported below the fraction. Therefore, taking \(\lim_{r\to 0}\frac{1}{r^{2k-n}(-2k+n)}\) would \(= +\infty\). 

Case 2: \(-2k+n>=0\). This would render no problem: the second term would be simply \(0\).

Therefore, if the terms that make up the infinite volcano \(-2k+n >0\), the results would be infinite. Otherwise, the results are finite.
\end{document}