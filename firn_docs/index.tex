% Created 2021-09-27 Mon 11:52
% Intended LaTeX compiler: xelatex
\documentclass[letterpaper]{article}
\usepackage{graphicx}
\usepackage{grffile}
\usepackage{longtable}
\usepackage{wrapfig}
\usepackage{rotating}
\usepackage[normalem]{ulem}
\usepackage{amsmath}
\usepackage{textcomp}
\usepackage{amssymb}
\usepackage{capt-of}
\usepackage{hyperref}
\setlength{\parindent}{0pt}
\usepackage[margin=1in]{geometry}
\usepackage{fontspec}
\usepackage{svg}
\usepackage{cancel}
\usepackage{indentfirst}
\setmainfont[ItalicFont = LiberationSans-Italic, BoldFont = LiberationSans-Bold, BoldItalicFont = LiberationSans-BoldItalic]{LiberationSans}
\newfontfamily\NHLight[ItalicFont = LiberationSansNarrow-Italic, BoldFont       = LiberationSansNarrow-Bold, BoldItalicFont = LiberationSansNarrow-BoldItalic]{LiberationSansNarrow}
\newcommand\textrmlf[1]{{\NHLight#1}}
\newcommand\textitlf[1]{{\NHLight\itshape#1}}
\let\textbflf\textrm
\newcommand\textulf[1]{{\NHLight\bfseries#1}}
\newcommand\textuitlf[1]{{\NHLight\bfseries\itshape#1}}
\usepackage{fancyhdr}
\pagestyle{fancy}
\usepackage{titlesec}
\usepackage{titling}
\makeatletter
\lhead{\textbf{\@title}}
\makeatother
\rhead{\textrmlf{Compiled} \today}
\lfoot{\theauthor\ \textbullet \ \textbf{2021-2022}}
\cfoot{}
\rfoot{\textrmlf{Page} \thepage}
\renewcommand{\tableofcontents}{}
\titleformat{\section} {\Large} {\textrmlf{\thesection} {|}} {0.3em} {\textbf}
\titleformat{\subsection} {\large} {\textrmlf{\thesubsection} {|}} {0.2em} {\textbf}
\titleformat{\subsubsection} {\large} {\textrmlf{\thesubsubsection} {|}} {0.1em} {\textbf}
\setlength{\parskip}{0.45em}
\renewcommand\maketitle{}
\date{}
\title{Taproot}
\hypersetup{
 pdfauthor={Taproot},
 pdftitle={Taproot},
 pdfkeywords={},
 pdfsubject={},
 pdfcreator={Emacs 28.0.50 (Org mode 9.4.4)}, 
 pdflang={English}}
\begin{document}


\section{Welcome}
\label{sec:orgf030131}
Howdy 👋, welcome to Taproot, the connected Zettlekastenish braindump of Huxley, Albert, Zach, Houjun, Dylan, and David. Take a look around, either in person or using this \href{https://taproot3.sanity.gq}{Handy Web Portal}.

This is the third version of the Taproot infrastructure. The website is set in \texttt{Helvetica Neue}, the generated PDFs are set in \texttt{Liberation Sans} and built using \texttt{xelatex}. Source files are written in or transpiled to \texttt{org} using \texttt{pandoc}, and built using \texttt{org-publish} and \texttt{firn}.

The software stack of Taproot is licensed under \texttt{GPL-3.0 License} and the text is licensed by \texttt{CC BY-NC-SA 4.0}. The license texts are available to view \href{https://www.gnu.org/licenses/gpl-3.0.en.html}{here} and \href{https://creativecommons.org/licenses/by-nc-sa/2.0/}{here} respectively.

Taproot is organized under a few global indexes of subject-specific areas. Enter Taproot by following down these lovely rabbit holes of notes:

\section{Notes}
\label{sec:org7bf776e}

\subsection{School Notes}
\label{sec:org86c8e41}
\begin{itemize}
\item Physics (Mechanics and EM): \href{physics/index.org}{Physics Global Index}
\item Biology: \href{biology/index.org}{Biology Global Index}
\item Humanities @TheNuevaSchool
\begin{itemize}
\item English: \href{english/index.org}{English Global Index}
\item History: \href{history/index.org}{History Global Index}
\end{itemize}
\item Mathematics (Calc, not calc, etc. etc.): \href{mathematics/index.org}{Mathematics Global Index}
\item Computer Science: \href{cs/index.org}{Computer Science Global Index}
\end{itemize}

There are secretly actually others, like Japanese or Spanish. But those are too bad for mere mortals to cast eyes on.

\subsection{Projects}
\label{sec:orgfbfc518}
Coming Soon!

\subsection{Personal Corners}
\label{sec:org4f4f3b0}
\begin{itemize}
\item Huxley: \href{corners/huxley/KBxSort.org}{Sort! with Huxley}
\item Jack: \href{corners/jack/index.org}{Jack's Personal Index}
\item David's entire notes: \href{corners/david/index.org}{David's Notes Index}
\end{itemize}

\section{Philosophy}
\label{sec:org285011a}
Zettelkasten, maybe. But basically, create a repository of knowledge that should be easy to refer back to and effective for relearning things.
We strive to create atomic, self contained notes that link to other references. Think a more granular Wikipedia.

\section{Joining Us}
\label{sec:orgdb2b23f}
Its probably a bit hard (we all go to school in the same area, we generally take similar classes, etc.)

\section{Epilogue}
\label{sec:orgfe89425}
Thanks for stopping by!

A "taproot" is a "large, central, and dominant root from which other roots sprout laterally" \href{https://en.wikipedia.org/wiki/Taproot}{See Wikipedia}. Also, it's a food!
\end{document}
