% Created 2021-09-27 Mon 12:00
% Intended LaTeX compiler: xelatex
\documentclass[letterpaper]{article}
\usepackage{graphicx}
\usepackage{grffile}
\usepackage{longtable}
\usepackage{wrapfig}
\usepackage{rotating}
\usepackage[normalem]{ulem}
\usepackage{amsmath}
\usepackage{textcomp}
\usepackage{amssymb}
\usepackage{capt-of}
\usepackage{hyperref}
\setlength{\parindent}{0pt}
\usepackage[margin=1in]{geometry}
\usepackage{fontspec}
\usepackage{svg}
\usepackage{cancel}
\usepackage{indentfirst}
\setmainfont[ItalicFont = LiberationSans-Italic, BoldFont = LiberationSans-Bold, BoldItalicFont = LiberationSans-BoldItalic]{LiberationSans}
\newfontfamily\NHLight[ItalicFont = LiberationSansNarrow-Italic, BoldFont       = LiberationSansNarrow-Bold, BoldItalicFont = LiberationSansNarrow-BoldItalic]{LiberationSansNarrow}
\newcommand\textrmlf[1]{{\NHLight#1}}
\newcommand\textitlf[1]{{\NHLight\itshape#1}}
\let\textbflf\textrm
\newcommand\textulf[1]{{\NHLight\bfseries#1}}
\newcommand\textuitlf[1]{{\NHLight\bfseries\itshape#1}}
\usepackage{fancyhdr}
\pagestyle{fancy}
\usepackage{titlesec}
\usepackage{titling}
\makeatletter
\lhead{\textbf{\@title}}
\makeatother
\rhead{\textrmlf{Compiled} \today}
\lfoot{\theauthor\ \textbullet \ \textbf{2021-2022}}
\cfoot{}
\rfoot{\textrmlf{Page} \thepage}
\renewcommand{\tableofcontents}{}
\titleformat{\section} {\Large} {\textrmlf{\thesection} {|}} {0.3em} {\textbf}
\titleformat{\subsection} {\large} {\textrmlf{\thesubsection} {|}} {0.2em} {\textbf}
\titleformat{\subsubsection} {\large} {\textrmlf{\thesubsubsection} {|}} {0.1em} {\textbf}
\setlength{\parskip}{0.45em}
\renewcommand\maketitle{}
\author{Huxley}
\date{\today}
\title{Mason Chapter Four}
\hypersetup{
 pdfauthor={Huxley},
 pdftitle={Mason Chapter Four},
 pdfkeywords={},
 pdfsubject={},
 pdfcreator={Emacs 28.0.50 (Org mode 9.4.4)}, 
 pdflang={English}}
\begin{document}

\tableofcontents

\#flo \#disorganized \#incomplete

\noindent\rule{\textwidth}{0.5pt}

\section{Mason? Chapter\ldots{} 4?}
\label{sec:org7e3b5e8}
\begin{itemize}
\item 1848 bunch of monarchies overthrown
\item within a few years, all the revolutions failed / were reversed
\item Individualism was growing due to the enlightenment and romance art
(the kind we learned about in english last year)
\item 'hodgepodge' of states
\end{itemize}

\subsection{Liberalism and nationalism in the early nineteenth century}
\label{sec:org16910a1}
\begin{itemize}
\item liberalism and nationlisk originated in the enlightenment,
\item started to contradict conservative values
\item two types:

\begin{itemize}
\item political liberalism

\begin{itemize}
\item goverment by consent

\item constitutionalism

\item tolerence of differing points of view

\item \begin{quote}
that one person's free- dom could be restricted only if it
impinged on the individual freedom of another: "The only purpose
for which power can be rightfully exercised over any member of a
civilized community, against his will, is to prevent harm to
others."
\end{quote}
\end{itemize}

\item economic liberalism

\begin{itemize}
\item derived from adam smith's the wealth of nations (eyy)
\item about the invisible hand
\item also want to limit the power of the governernment, but mostly in
relation to the economy
\end{itemize}

\item about difference in emphasis

\item grew alot in the 19th century with the growing middle class

\item also nationalism was a thing

\item some guy in the 18th century

\begin{itemize}
\item \begin{quote}
"Neither pope nor king," he declared, "only God and the people."
Later, he created an international branch of his organization,
Young Europe, which trained a network of conspirators across the
Continent to agitate for democratic constitutions.
\end{quote}
\end{itemize}
\end{itemize}
\end{itemize}

\subsection{precursors to 1848: the 1830 revolution in france}
\label{sec:orgc7b2b28}
\begin{itemize}
\item liberal and national movements (both anti goverment ig)
\item revolted a bunch in the early 1800's
\item like greece against the ottomans, who won!
\item one of frances leaders tried to revoke a bunch of the progress people
had made, and the society rioted and he fled
\item end of the 'bourbon' monarchy? \#what?
\end{itemize}

\subsection{the actual revolution of 1848}
\label{sec:org761ab0b}
\begin{itemize}
\item recession and food shortages led to unrest (1846-47)

\item and the people had less power

\item the monarch fled again

\item during the industrial revolution!

\item socialism starts gaining traction\ldots{}

\item \begin{quote}
In the Bloody June Days of June 24 to 26, several thousand people
were killed and eleven thousand insurgents were imprisoned or
deported. The specter of socialist revolution had been suppressed,
but the events of June sent a shudder through all the governments of
Europe.
\end{quote}

\item and then napolean comes and totally dissolves the democracy

\item wheeee
\end{itemize}

\subsection{revolt spreads through europe}
\label{sec:org7d6d5be}
\begin{itemize}
\item this one's pretty self explanatory
\end{itemize}
\end{document}
