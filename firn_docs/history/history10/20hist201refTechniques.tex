% Created 2021-09-27 Mon 12:00
% Intended LaTeX compiler: xelatex
\documentclass[letterpaper]{article}
\usepackage{graphicx}
\usepackage{grffile}
\usepackage{longtable}
\usepackage{wrapfig}
\usepackage{rotating}
\usepackage[normalem]{ulem}
\usepackage{amsmath}
\usepackage{textcomp}
\usepackage{amssymb}
\usepackage{capt-of}
\usepackage{hyperref}
\setlength{\parindent}{0pt}
\usepackage[margin=1in]{geometry}
\usepackage{fontspec}
\usepackage{svg}
\usepackage{cancel}
\usepackage{indentfirst}
\setmainfont[ItalicFont = LiberationSans-Italic, BoldFont = LiberationSans-Bold, BoldItalicFont = LiberationSans-BoldItalic]{LiberationSans}
\newfontfamily\NHLight[ItalicFont = LiberationSansNarrow-Italic, BoldFont       = LiberationSansNarrow-Bold, BoldItalicFont = LiberationSansNarrow-BoldItalic]{LiberationSansNarrow}
\newcommand\textrmlf[1]{{\NHLight#1}}
\newcommand\textitlf[1]{{\NHLight\itshape#1}}
\let\textbflf\textrm
\newcommand\textulf[1]{{\NHLight\bfseries#1}}
\newcommand\textuitlf[1]{{\NHLight\bfseries\itshape#1}}
\usepackage{fancyhdr}
\pagestyle{fancy}
\usepackage{titlesec}
\usepackage{titling}
\makeatletter
\lhead{\textbf{\@title}}
\makeatother
\rhead{\textrmlf{Compiled} \today}
\lfoot{\theauthor\ \textbullet \ \textbf{2021-2022}}
\cfoot{}
\rfoot{\textrmlf{Page} \thepage}
\renewcommand{\tableofcontents}{}
\titleformat{\section} {\Large} {\textrmlf{\thesection} {|}} {0.3em} {\textbf}
\titleformat{\subsection} {\large} {\textrmlf{\thesubsection} {|}} {0.2em} {\textbf}
\titleformat{\subsubsection} {\large} {\textrmlf{\thesubsubsection} {|}} {0.1em} {\textbf}
\setlength{\parskip}{0.45em}
\renewcommand\maketitle{}
\author{Exr0n, Houjun Liu}
\date{\today}
\title{History Techniques}
\hypersetup{
 pdfauthor={Exr0n, Houjun Liu},
 pdftitle={History Techniques},
 pdfkeywords={},
 pdfsubject={},
 pdfcreator={Emacs 28.0.50 (Org mode 9.4.4)}, 
 pdflang={English}}
\begin{document}

\tableofcontents

\#ref

\begin{itemize}
\item When upset by a reading (Sushu @ Kennedy)

\begin{itemize}
\item 31 August 2020, Sushu Block 1
\item Helpful to diagram out the author's argument

\begin{itemize}
\item What are the author's claims?
\item Where is the evidence?
\item What spin does the author put on it? (Seen as a strength or
weakness?)
\item Examples

\begin{itemize}
\item \href{20hist201srcBreakoutKennedyChinaBreakdown.png.org}{20hist201srcBreakoutKennedyChinaBreakdown.png}
\end{itemize}
\end{itemize}

\item Analyze evidence

\begin{itemize}
\item Does it exist?
\item Where are the gaps in the argument?
\item How does the author define terms?
\end{itemize}

\item \textbf{This feels like thinking like a sports fan}
\end{itemize}

\item Doing Readings

\begin{itemize}
\item \textbf{Create atomic notes, like a Zettelkasten}

\item Skim for length, decide what not to read
\item Get a note taking doc
\item Annotate for

\begin{enumerate}
\item Structure: "x reasons/characteristics, Firstly, primarily,
secondly, in addition"
\item Claims
\item key points
\item Things that remind you of other readings / claims
\end{enumerate}

\item Write summary phrases for paragraphs
\item Skip things you don't need to read (if you already know about it)
\end{itemize}
\end{itemize}

\noindent\rule{\textwidth}{0.5pt}
\end{document}
