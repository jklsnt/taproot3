% Created 2021-09-27 Mon 12:01
% Intended LaTeX compiler: xelatex
\documentclass[letterpaper]{article}
\usepackage{graphicx}
\usepackage{grffile}
\usepackage{longtable}
\usepackage{wrapfig}
\usepackage{rotating}
\usepackage[normalem]{ulem}
\usepackage{amsmath}
\usepackage{textcomp}
\usepackage{amssymb}
\usepackage{capt-of}
\usepackage{hyperref}
\setlength{\parindent}{0pt}
\usepackage[margin=1in]{geometry}
\usepackage{fontspec}
\usepackage{svg}
\usepackage{cancel}
\usepackage{indentfirst}
\setmainfont[ItalicFont = LiberationSans-Italic, BoldFont = LiberationSans-Bold, BoldItalicFont = LiberationSans-BoldItalic]{LiberationSans}
\newfontfamily\NHLight[ItalicFont = LiberationSansNarrow-Italic, BoldFont       = LiberationSansNarrow-Bold, BoldItalicFont = LiberationSansNarrow-BoldItalic]{LiberationSansNarrow}
\newcommand\textrmlf[1]{{\NHLight#1}}
\newcommand\textitlf[1]{{\NHLight\itshape#1}}
\let\textbflf\textrm
\newcommand\textulf[1]{{\NHLight\bfseries#1}}
\newcommand\textuitlf[1]{{\NHLight\bfseries\itshape#1}}
\usepackage{fancyhdr}
\pagestyle{fancy}
\usepackage{titlesec}
\usepackage{titling}
\makeatletter
\lhead{\textbf{\@title}}
\makeatother
\rhead{\textrmlf{Compiled} \today}
\lfoot{\theauthor\ \textbullet \ \textbf{2021-2022}}
\cfoot{}
\rfoot{\textrmlf{Page} \thepage}
\renewcommand{\tableofcontents}{}
\titleformat{\section} {\Large} {\textrmlf{\thesection} {|}} {0.3em} {\textbf}
\titleformat{\subsection} {\large} {\textrmlf{\thesubsection} {|}} {0.2em} {\textbf}
\titleformat{\subsubsection} {\large} {\textrmlf{\thesubsubsection} {|}} {0.1em} {\textbf}
\setlength{\parskip}{0.45em}
\renewcommand\maketitle{}
\author{Huxley}
\date{\today}
\title{The Quiz!}
\hypersetup{
 pdfauthor={Huxley},
 pdftitle={The Quiz!},
 pdfkeywords={},
 pdfsubject={},
 pdfcreator={Emacs 28.0.50 (Org mode 9.4.4)}, 
 pdflang={English}}
\begin{document}

\tableofcontents

\#ref \#ret

\noindent\rule{\textwidth}{0.5pt}

\section{begin!}
\label{sec:org6fe6a5b}
Honor Code

I, Huxley Marvit, affirm that I will only utilize the internet during
this assessment for the purpose of accessing my class notes and
documents linked on the class Canvas site. I will close all other
internet browser windows before beginning the assessment. I will not use
any other resources, including using search engines to look up terms. I
will not discuss the assessment with anyone but Jehnna, including after
it's completed. If I am confused about wording or terminology used on
the assessment, I will reference the allowed materials and/or ask Jehnna
to clarify by sending a private zoom message to her. I agree that I will
learn best by authentically engaging with the assessment rather than
searching for answers on the internet or from my friends. I understand
that I will be offered a reassessment opportunity if I need it. I affirm
that I, Huxley Marvit, have read this honor code and will abide by it.

\noindent\rule{\textwidth}{0.5pt}

\subsection{1}
\label{sec:org46bef3e}
\begin{enumerate}
\item A
\label{sec:org82fce9f}
mom: pp, dad: PP pP -> P In any combination, their would not be two
mutant PAH alleles. Thus, the child would have a 0\% chance of having the
disorder.

\item B
\label{sec:org7ce4fec}
child: pP, partner: pp

pp, (p1p1) -> p pp, (p1p2) -> p Pp, (Pp1) -> P Pp, (Pp2) -> P

50\% chance of having PKU disorder

\begin{enumerate}
\item \textbf{A promoter mutation that reduces expression of PAH protein to 50\% of
normal levels.}

\begin{enumerate}
\item Given that "classic PKU" results from near complete loss of PAH
function, a 50\% loss would most likely be classified as "mild PKU"
\end{enumerate}

\item \textbf{A missense mutation that changes an amino acid in the PAH enzyme's
active site, preventing any phenylalanine from binding there.}

\begin{enumerate}
\item This would completely inhibit PAH function, leading to "classic
PKU"
\end{enumerate}

\item \textbf{A frameshift mutation very early in the coding sequence of the PAH
gene.}

\begin{enumerate}
\item A frameshift mutation early on would cause almost the entire
sequence to be translated incorrectly. Most likely, this would
lead to near complete loss of function, and thus, "classic PKU"
\end{enumerate}

\item *A missense mutation that changes an amino acid in an allosteric site
(an enzyme site that is not directly involved in breaking down
phenylalanine), leading to a 40\% reduction in the rate of enzyme
activity.*

\begin{enumerate}
\item 40\% reduction is not near-complete, and would most likely be
classified as "mild PKU"
\end{enumerate}
\end{enumerate}

\noindent\rule{\textwidth}{0.5pt}
\end{enumerate}

\subsection{2}
\label{sec:orgff642a8}
\begin{enumerate}
\item A
\label{sec:orgd1d7aae}
Most woman have two X chromosomes, whereas most men have a X chromosome
and a Y chromosome. Since hemophilia is located on the X chromosome, in
men, it doesn't have a chance to be dominated.

\item B
\label{sec:org75e0cb9}
For the woman to be healthy and have any hemophilia-associated alleles,
they must have a singular recessive mutation.

Somatic cells, carrying 23 pairs of chromosomes, have all the genetic
information which is all copied by mitosis. Thus, they will contain the
hemophilia-associated allele.

Meiosis produces haploid cells with only 23 singular 2-chromatid
chromosomes. Thus, only half would have the mutant x chromosome.

\item C
\label{sec:org97f7500}
woman: xX, man: XY

For their child to be male, the man has to pass down his Y, leaving
options:

xY, XY.

Thus, their is a 50\% chance of hemophilia.

\noindent\rule{\textwidth}{0.5pt}
\end{enumerate}

\subsection{3}
\label{sec:orgdb38da0}
\begin{enumerate}
\item A
\label{sec:org34f68f2}
Given that p53 is a negative regulator, it most likely has a loss of
function mutation, making it less effective at doing its job of pausing
cell cycle progression when needed and initiating cell death. When p53
is less effective, an uncontrolled overgrowth of cells is more likely to
occur.

\item B
\label{sec:org3c4be4c}
Given that RET is a positive regulator, MEN2 mutations would most likely
cause a gain of function. The RET protein would be more prone to
signaling for cell progression, pushing cell division forward before
everything is necessarily ready. This would increase the likelihood of
cancer.
\end{enumerate}
\end{document}
