% Created 2021-09-27 Mon 11:53
% Intended LaTeX compiler: xelatex
\documentclass[letterpaper]{article}
\usepackage{graphicx}
\usepackage{grffile}
\usepackage{longtable}
\usepackage{wrapfig}
\usepackage{rotating}
\usepackage[normalem]{ulem}
\usepackage{amsmath}
\usepackage{textcomp}
\usepackage{amssymb}
\usepackage{capt-of}
\usepackage{hyperref}
\setlength{\parindent}{0pt}
\usepackage[margin=1in]{geometry}
\usepackage{fontspec}
\usepackage{svg}
\usepackage{cancel}
\usepackage{indentfirst}
\setmainfont[ItalicFont = LiberationSans-Italic, BoldFont = LiberationSans-Bold, BoldItalicFont = LiberationSans-BoldItalic]{LiberationSans}
\newfontfamily\NHLight[ItalicFont = LiberationSansNarrow-Italic, BoldFont       = LiberationSansNarrow-Bold, BoldItalicFont = LiberationSansNarrow-BoldItalic]{LiberationSansNarrow}
\newcommand\textrmlf[1]{{\NHLight#1}}
\newcommand\textitlf[1]{{\NHLight\itshape#1}}
\let\textbflf\textrm
\newcommand\textulf[1]{{\NHLight\bfseries#1}}
\newcommand\textuitlf[1]{{\NHLight\bfseries\itshape#1}}
\usepackage{fancyhdr}
\pagestyle{fancy}
\usepackage{titlesec}
\usepackage{titling}
\makeatletter
\lhead{\textbf{\@title}}
\makeatother
\rhead{\textrmlf{Compiled} \today}
\lfoot{\theauthor\ \textbullet \ \textbf{2021-2022}}
\cfoot{}
\rfoot{\textrmlf{Page} \thepage}
\renewcommand{\tableofcontents}{}
\titleformat{\section} {\Large} {\textrmlf{\thesection} {|}} {0.3em} {\textbf}
\titleformat{\subsection} {\large} {\textrmlf{\thesubsection} {|}} {0.2em} {\textbf}
\titleformat{\subsubsection} {\large} {\textrmlf{\thesubsubsection} {|}} {0.1em} {\textbf}
\setlength{\parskip}{0.45em}
\renewcommand\maketitle{}
\author{Exr0n}
\date{\today}
\title{Axler 3.E ex18}
\hypersetup{
 pdfauthor={Exr0n},
 pdftitle={Axler 3.E ex18},
 pdfkeywords={},
 pdfsubject={},
 pdfcreator={Emacs 28.0.50 (Org mode 9.4.4)}, 
 pdflang={English}}
\begin{document}

\tableofcontents

\section{Problem: Axler 3.E exercise 18}
\label{sec:org29639c7}
\begin{quote}
Suppose \(T \in \mathcal L(V, W)\) and \(U\) is a subspace of \(V\). Let \(\pi\) denote the quotient map from \(V\) onto \(V/U\). Prove that there exists \(S \in \mathcal L(V/U, W)\) such that \(T = S \circ \pi\) if and only if \(U \subseteq \text{null }T\).
\end{quote}
Intuitively, if we mod out part of the \(\text{null }T\), then we should still be able to have a map that does what \(T\) would do. If we are able to do what \(T\) would do, then when modding out \(U\) we only removed part of \(\text{null }T\) and lost no information.

\section{Forward Direction}
\label{sec:org79ac384}

Intuitively, we can treat \(S \circ \pi\) as a single map and take a basis of \(V\) to the same place that \(T\) would, and the maps would be equal.

Let \(S\) be a relation between \(V/U\) and \(W\) defined by
\[ S(U+v) = Tv \]

If \(S\) is well defined (every element in \(V/U\) is mapped to exactly one place), then \(S\) will inherit additivity and homogeneity from \(T\) and \(S \circ \pi\) will equal \(T\).

Let \(v \in V\) and \(v' \in V/U\) s.t. \(v' = \pi v\) (\(v'\) is where \(\pi\) takes \(v\)). Then, to show that \(S\) is well defined, we must show that \(v\) has atleast one and at most one image through \(S \circ \pi\).

Because \(\pi v\) is well defined, and \(U+v\) was arbitrary in the definition of \(S\), each \(v\) must have atleast one image in \(W\).

Take \(S\) to be an arbitrary linear map. The only restriction on \(S\) that could cause \(S(U+v) \neq Tv\) is \(S(0) = 0\) (this statement is not watertight).
Thus, \(S\) is defined if \(\forall U+v = U = 0\), \(Tv = 0\). Equivalently, \(S\) is defined if \(U \subseteq \text{null }T\), which is given in the problem.

Thus, \(S\) is well defined. To show that it inherits additivity and homogeneity:
\[ S(U+u) + S(U+v) = Tu + Tv = T(u+v) = S(U+u + U+v) = S(U+(u+v)) \]

\[ \lambda\left(S(U+v)\right) = \lambda Tv = T(\lambda v) = S(U+(\lambda v)) \]

Thus, \(S\) is linear, and \(S \circ \pi = T\) if \(U \subseteq \text{null }T\).

\section{Reverse Direction by Contrapositive}
\label{sec:org71cd9b6}
Intuitively, if we lost information, then we can't reconstruct what \(T\) would do.

Assume \(U \nsubseteq \text{null }T\). There exists \(v \in U\) s.t. \(Tv \neq 0\). This is some of the "information" that was "lost". Because \(v \in U\),

\[ \pi v = U + v = U \]

Because \(U\) is the additive identity (\(0\)) in \(V/U\), and because linear maps take zero to zero, \(S \in \mathcal L(V/U, W)\) must take \(\pi v = 0\) to zero.
Thus, either \(S(\pi v) \neq Tv\) or \(S\) is not a linear map, both of which are contradictions.

This shows that if \(U \nsubseteq \text{null }T\), then \(S \notin \mathcal L(V/U, W)\) or \(T \neq S \circ \pi\). Thus, if \(S \in \mathcal L(V/U, W)\) and \(T = S \circ \pi\), then \(U \subseteq \text{null }T\).
\end{document}
