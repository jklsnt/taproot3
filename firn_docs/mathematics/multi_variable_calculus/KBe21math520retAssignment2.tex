% Created 2021-09-27 Mon 11:53
% Intended LaTeX compiler: xelatex
\documentclass[letterpaper]{article}
\usepackage{graphicx}
\usepackage{grffile}
\usepackage{longtable}
\usepackage{wrapfig}
\usepackage{rotating}
\usepackage[normalem]{ulem}
\usepackage{amsmath}
\usepackage{textcomp}
\usepackage{amssymb}
\usepackage{capt-of}
\usepackage{hyperref}
\setlength{\parindent}{0pt}
\usepackage[margin=1in]{geometry}
\usepackage{fontspec}
\usepackage{svg}
\usepackage{cancel}
\usepackage{indentfirst}
\setmainfont[ItalicFont = LiberationSans-Italic, BoldFont = LiberationSans-Bold, BoldItalicFont = LiberationSans-BoldItalic]{LiberationSans}
\newfontfamily\NHLight[ItalicFont = LiberationSansNarrow-Italic, BoldFont       = LiberationSansNarrow-Bold, BoldItalicFont = LiberationSansNarrow-BoldItalic]{LiberationSansNarrow}
\newcommand\textrmlf[1]{{\NHLight#1}}
\newcommand\textitlf[1]{{\NHLight\itshape#1}}
\let\textbflf\textrm
\newcommand\textulf[1]{{\NHLight\bfseries#1}}
\newcommand\textuitlf[1]{{\NHLight\bfseries\itshape#1}}
\usepackage{fancyhdr}
\pagestyle{fancy}
\usepackage{titlesec}
\usepackage{titling}
\makeatletter
\lhead{\textbf{\@title}}
\makeatother
\rhead{\textrmlf{Compiled} \today}
\lfoot{\theauthor\ \textbullet \ \textbf{2021-2022}}
\cfoot{}
\rfoot{\textrmlf{Page} \thepage}
\renewcommand{\tableofcontents}{}
\titleformat{\section} {\Large} {\textrmlf{\thesection} {|}} {0.3em} {\textbf}
\titleformat{\subsection} {\large} {\textrmlf{\thesubsection} {|}} {0.2em} {\textbf}
\titleformat{\subsubsection} {\large} {\textrmlf{\thesubsubsection} {|}} {0.1em} {\textbf}
\setlength{\parskip}{0.45em}
\renewcommand\maketitle{}
\author{Albert H}
\date{\today}
\title{Assignment 2: parameterization continued}
\hypersetup{
 pdfauthor={Albert H},
 pdftitle={Assignment 2: parameterization continued},
 pdfkeywords={},
 pdfsubject={},
 pdfcreator={Emacs 28.0.50 (Org mode 9.4.4)}, 
 pdflang={English}}
\begin{document}

\tableofcontents

\href{https://www.desmos.com/calculator/hhb49omfkj}{Desmos graphs}

\begin{latex}
\setcounter{section}{3}
\end{latex}

\section{witch of Maria Agnesi}
\label{sec:org625a3f4}

Let \(B\) be the center of the orange circle with radius \(a\), let \(D\) be the closest point to \(C\) on the x-axis, and let \(Q\) be the closest point to \(A\) on the y-axis.

\subsection{\(x(t)\)}
\label{sec:orgf3c852a}

\[\begin{aligned}
    \tan \theta &= \frac{\overline{CD}}{\overline{OD}}\\
    \cot  \theta &= \frac{\overline{OD}}{\overline{CD}}\\
    \overline{CD} \cot  \theta &= \overline{OD} \\
    2a \cot  \theta &= x
   \end{aligned}\]

\subsection{\(y(t)\)}
\label{sec:org722161b}


First, note that the distances
\[\begin{aligned}
   \overline{AB} &= \overline{BO} = a\\
   \overline{PD} &= \overline{QO} = \overline{QB} + \overline{BO} = \overline{QB} + a = y\\
   \end{aligned}\]

Using some geometry:

\[\begin{aligned}
   \angle AOB &= 90-\theta \\
   \angle OAB &= 90-\theta && \quad\text{(isocelese triangle)}\\
   \angle ABO &= 2\theta \\
   \end{aligned}\]

Which implies:

\[\begin{aligned}
   \overline{QB} &= -a \cos  (2 \theta )  \\
   &= -a \left( 1 - 2 \sin ^2 \theta \right)  \\
   &= -a + 2a \sin  ^2 \theta 
   \end{aligned}\]

By going back to the original distance relations, we have 
\[\begin{aligned}
   y &= \overline{QB} + a \\
   &= \cancel{a - a} + 2a \sin  ^2 \theta 
   &= 2 a \sin  ^2 \theta 
   \end{aligned}\]


\section{parameterization of an elipse}
\label{sec:orgbf43b35}
\url{https://www.desmos.com/calculator/wcu1okhjyz}

\[\begin{aligned}
  x(t) = a \sqrt{c} \sin  t\\
  y(t) = b \sqrt{c} \cos  t
  \end{aligned}\]

\section{mystery curve}
\label{sec:org1666d40}
it's just \((a \cos  t, b \sin  t)\) because of how the right triangle aligns with the axes. 
\begin{latex}
\setcounter{section}{7}
\end{latex}

\section{swallowtail catastrophe curves}
\label{sec:org5bf186b}
Defined by 
\[\begin{aligned}
  x &= 2ct - 4t^3\\
  y &= -ct^2 + 3 t^4
  \end{aligned}\]
\subsection{features}
\label{sec:org37841aa}
\subsubsection{approaches a parabola-like shape above the y-axis}
\label{sec:org49545d0}
\subsubsection{approaches a parabola-like shape below the x-axis if \(c > 0\)}
\label{sec:org1de5dea}
\subsubsection{has a cross-over in a triangle shape}
\label{sec:orgea30445}
\begin{enumerate}
\item gets bigger when \(c\) gets bigger
\label{sec:org61e5b88}
\end{enumerate}
\subsubsection{it looks like a dorito that scales with the value of \(c\)}
\label{sec:org025a485}
\begin{enumerate}
\item as \(c\) approaches zero from the positive direction, the swollowtail gets smaller
\label{sec:org49279cc}
\end{enumerate}
\section{Lissajous Figures}
\label{sec:org7b92191}
Defined by 
\[\begin{aligned}
  x &= a \sin (nt)\\
  y &= b \cos  t
  \end{aligned}\]
\subsection{features}
\label{sec:org1cc22e0}
\subsubsection{spring-like coil shape (almost like standing waves) with tighter "loops" at the ends}
\label{sec:org5386bc5}
\subsubsection{\(a, b\) control the size of the coil (default \(-1 \le x, y \le 1\) because of range of \(\sin, \cos\)}
\label{sec:orgf3ab426}
\subsubsection{number of y-intercepts is \(n+1\) except in the degenerate cases \(n \le 0\)}
\label{sec:orgddd5f62}

\begin{latex}
\setcounter{section}{10}
\end{latex}
\section{cycloid}
\label{sec:org5c14b75}
Suppose instead that the circle slides along the surface and the point rotates at one radian per radian traveled. Let's start with the radian rotation\ldots{}

\[\begin{aligned}
  x(t) &= &r\sin t\\
  y(t) &= r + &r\cos t\\
  \end{aligned}\]

Then, we just have to move the origin as well:

\[\begin{aligned}
  x(t) &= t + r \sin  t\\
  y(t) &= r + r \cos t
  \end{aligned}\]
\section{first order derivative}
\label{sec:orgc96003c}

I think I did not come to this conclusion on my own on 30 Aug. because I didn't realize we could assume we had \(y(x)\).
\[\begin{aligned}
  y &= y(x(t))\\
  \frac{dy}{dt} &= y'(x(t)) x'(t) = \frac{dy}{dx} \frac{dx}{dt} && \quad \text{(chain rule)}\\
  \frac{dy}{dx} &= \frac{\frac{dy}{dt}}{\frac{dx}{dt}}
  \end{aligned}\]
\section{second order derivative}
\label{sec:orge81cf76}


\[\begin{aligned}
  x &= f(t)\\
  y &= g(t) = g(f(t))\\
  \end{aligned}\]


\[\begin{aligned}
  \frac{dy}{dt} &= \frac{dy}{dx}\frac{dx}{dt}\\
  \frac{d^2y}{dt^2} &= \frac{dy}{dx} \frac{d}{dt}\frac{dx}{dt} + \frac{dx}{dt} \frac{d}{dt}\frac{dy}{dx}\\
  &= \frac{dy}{dx} \frac{d^2x}{dt^2} + \frac{dx}{dt} \frac{d^2y}{dxdt (??)}\\
  \end{aligned}\]


\[\begin{aligned}
  \frac{d^2x}{dt^2} &= \frac{d}{dt} \frac{dx}{dt} 
  \end{aligned}\]

um\ldots{} that seems like it didn't actually do anything. I'm kind of stuck\ldots{} lets try working backwards:


\[\begin{aligned}
  \frac{d^2y}{dx^2} &= \frac{\dot x \ddot y - \dot y \ddot x}{(\dot x)^3} \\
  &= \dot x \frac{d}{dx} \left( \frac{\dot{y}}{\dot{x}} \right)  \\
  \end{aligned}\]


why should the \(\dot x\) in the bottom be cubed?
\subsection{in class review}
\label{sec:org3114c80}

\[\begin{aligned}
   \frac{d}{dx} \frac{dy}{dx} = \frac{d}{dx} \left( \frac{\frac{dy}{dt}}{\frac{dx}{dt}} \right) = \frac{d}{dx} u = \frac{\frac{du}{dt}}{\frac{dx}{dt}}\\
   = \frac{\frac{d}{dt}u}{\frac{dx}{dt}} = \frac{\frac{d}{dt} \frac{\dot{y}}{\dot{x}} }{\dot{x}}\\
   = \frac{ \frac{\dot{x}\ddot{y}-\dot{y}\ddot{x}}{\dot{x}^2}}{\dot{x}}
   \end{aligned}\]
\end{document}
