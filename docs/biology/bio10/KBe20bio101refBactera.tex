% Created 2021-09-12 Sun 22:49
% Intended LaTeX compiler: xelatex
\documentclass[letterpaper]{article}
\usepackage{graphicx}
\usepackage{grffile}
\usepackage{longtable}
\usepackage{wrapfig}
\usepackage{rotating}
\usepackage[normalem]{ulem}
\usepackage{amsmath}
\usepackage{textcomp}
\usepackage{amssymb}
\usepackage{capt-of}
\usepackage{hyperref}
\usepackage[margin=1in]{geometry}
\usepackage{fontspec}
\usepackage{indentfirst}
\setmainfont[ItalicFont = LiberationSans-Italic, BoldFont = LiberationSans-Bold, BoldItalicFont = LiberationSans-BoldItalic]{LiberationSans}
\newfontfamily\NHLight[ItalicFont = LiberationSansNarrow-Italic, BoldFont       = LiberationSansNarrow-Bold, BoldItalicFont = LiberationSansNarrow-BoldItalic]{LiberationSansNarrow}
\newcommand\textrmlf[1]{{\NHLight#1}}
\newcommand\textitlf[1]{{\NHLight\itshape#1}}
\let\textbflf\textrm
\newcommand\textulf[1]{{\NHLight\bfseries#1}}
\newcommand\textuitlf[1]{{\NHLight\bfseries\itshape#1}}
\usepackage{fancyhdr}
\pagestyle{fancy}
\usepackage{titlesec}
\usepackage{titling}
\makeatletter
\lhead{\textbf{\@title}}
\makeatother
\rhead{\textrmlf{Compiled} \today}
\lfoot{\theauthor\ \textbullet \ \textbf{2021-2022}}
\cfoot{}
\rfoot{\textrmlf{Page} \thepage}
\titleformat{\section} {\Large} {\textrmlf{\thesection} {|}} {0.3em} {\textbf}
\titleformat{\subsection} {\large} {\textrmlf{\thesubsection} {|}} {0.2em} {\textbf}
\titleformat{\subsubsection} {\large} {\textrmlf{\thesubsubsection} {|}} {0.1em} {\textbf}
\setlength{\parskip}{0.45em}
\renewcommand\maketitle{}
\author{Exr0n}
\date{\today}
\title{Bacteria}
\hypersetup{
 pdfauthor={Exr0n},
 pdftitle={Bacteria},
 pdfkeywords={},
 pdfsubject={},
 pdfcreator={Emacs 28.0.50 (Org mode 9.4.4)}, 
 pdflang={English}}
\begin{document}

\maketitle


\section{Symbiosis with plants to make Nitrogen}
\label{sec:org260fc64}
Plants have bacteria that turn \(N_2\) into \(NH_3\). The bacteria has a
enzyme called nitrogenase.

So much fossil fuels are used in the fertilizer industry to do the same
conversion. Why can't we just use the enzyme to make a bioreactor? The
enzymes don't actually work with oxygen around. So how can plants create
a low oxygen environment in their root systems? It turns out that the
root nodules of the plants make leghemoglobin to soak up the oxygen.
Impossible foods found the gene in the plant and put it in yeast, which
is what makes it taste like meat.

\begin{itemize}
\item Oxygen is a non-competitive inhibitor of
nitrogenase\href{20bio201srcOxygenInhibitsNitrogenase.png.org}{20bio201srcOxygenInhibitsNitrogenase.png}
\end{itemize}

\subsection{Also cows, apparently}
\label{sec:orgeb1a053}
\begin{itemize}
\item Also cows, apparently.
\end{itemize}

\section{Cyanobacteria}
\label{sec:org573656a}
\begin{itemize}
\item "fix atmospheric dinitrogen" can make energy from light.
\item They need nitrogen, but they can't do it in the same cells that that
are photosynthesizing because Nitrogenase doesn't work with oxygen

\begin{itemize}
\item So, they somewhat specialized so some cells do the photosynthesis
and some do the nitrogen synthesis
\end{itemize}
\end{itemize}

\noindent\rule{\textwidth}{0.5pt}
\end{document}
