% Created 2021-09-12 Sun 22:49
% Intended LaTeX compiler: xelatex
\documentclass[letterpaper]{article}
\usepackage{graphicx}
\usepackage{grffile}
\usepackage{longtable}
\usepackage{wrapfig}
\usepackage{rotating}
\usepackage[normalem]{ulem}
\usepackage{amsmath}
\usepackage{textcomp}
\usepackage{amssymb}
\usepackage{capt-of}
\usepackage{hyperref}
\usepackage[margin=1in]{geometry}
\usepackage{fontspec}
\usepackage{indentfirst}
\setmainfont[ItalicFont = LiberationSans-Italic, BoldFont = LiberationSans-Bold, BoldItalicFont = LiberationSans-BoldItalic]{LiberationSans}
\newfontfamily\NHLight[ItalicFont = LiberationSansNarrow-Italic, BoldFont       = LiberationSansNarrow-Bold, BoldItalicFont = LiberationSansNarrow-BoldItalic]{LiberationSansNarrow}
\newcommand\textrmlf[1]{{\NHLight#1}}
\newcommand\textitlf[1]{{\NHLight\itshape#1}}
\let\textbflf\textrm
\newcommand\textulf[1]{{\NHLight\bfseries#1}}
\newcommand\textuitlf[1]{{\NHLight\bfseries\itshape#1}}
\usepackage{fancyhdr}
\pagestyle{fancy}
\usepackage{titlesec}
\usepackage{titling}
\makeatletter
\lhead{\textbf{\@title}}
\makeatother
\rhead{\textrmlf{Compiled} \today}
\lfoot{\theauthor\ \textbullet \ \textbf{2021-2022}}
\cfoot{}
\rfoot{\textrmlf{Page} \thepage}
\titleformat{\section} {\Large} {\textrmlf{\thesection} {|}} {0.3em} {\textbf}
\titleformat{\subsection} {\large} {\textrmlf{\thesubsection} {|}} {0.2em} {\textbf}
\titleformat{\subsubsection} {\large} {\textrmlf{\thesubsubsection} {|}} {0.1em} {\textbf}
\setlength{\parskip}{0.45em}
\renewcommand\maketitle{}
\author{Houjun Liu}
\date{\today}
\title{Diseases, an Overview}
\hypersetup{
 pdfauthor={Houjun Liu},
 pdftitle={Diseases, an Overview},
 pdfkeywords={},
 pdfsubject={},
 pdfcreator={Emacs 28.0.50 (Org mode 9.4.4)}, 
 pdflang={English}}
\begin{document}

\maketitle


\section{Diseases, an overview}
\label{sec:orgfa003f7}
\definition{**Disease**}{an abnormal condition that causes impairment in/loss of function of an organism (a.k.a. decreased fitness) that is not due to immediate external injury.}
\subsection{Causes of Disease}
\label{sec:org0a09749}
\begin{itemize}
\item Infectious agents
\item Deficiency disorders
\item Heritable factors
\item Physiological disorders (immunodeficiency, autoimmune disorders,
allergies, etc.)
\end{itemize}

\subsection{Types of Diseases}
\label{sec:org1204b1c}
\subsubsection{Congenital Disease}
\label{sec:orgf790392}
Diseases present at birth due to DNA abnormalities / pregnancy
pathological issues

\subsubsection{Acquired Disease}
\label{sec:org37ad365}
Diseases that begin during the lifetime due to exposure to some
environmental factors\ldots{}

\begin{itemize}
\item \textbf{Microrganism Invasion} => "infectious disease"
\item \textbf{Autoimmune reaction} => your body fighting your body
\item \textbf{Nutrient deficiency} => not eating good
\item \textbf{Mechanical wear} => wear and tear of physical body parts
\item \textbf{Ingestion of noxious chemicals} => eating poison
\end{itemize}

\noindent\rule{\textwidth}{0.5pt}

An aside\ldots{}

\textbf{Infectious diseases actually smaller on the causes of death in the US}

\begin{itemize}
\item Heart disease => wear + deficiency
\item Cancer => heritable + DNA
\item Unintentional injuries => not a disease
\item Chronic respitory disease => wear
\item Stroke => not a disease
\item Alhetimer disease => wear
\item Diabetes => autoimmune, nutrient, wear
\item Influenca <= \textbf{here, finally, an infections disease.} \textbf{*}
\end{itemize}

\subsection{Measuring diseases: pathogenicity + virulence}
\label{sec:orgc3d81e2}
See
\href{KBhBIO101PathogenicityandVirulence.org}{KBhBIO101PathogenicityandVirulence}

\subsection{Disease-causing Agents}
\label{sec:orgf1404d0}
\begin{itemize}
\item \textbf{Protozoan} => single-celled eukaryotes
\item \textbf{Fungal} => single/multi-celled eukarotyes
\item \textbf{Bacteria} => single-celled prokaryotes
\item \textbf{Viral} => acellular parasitic infectious agent
\item \textbf{Helminuthus} => multicellular worms
\item \textbf{Prions} => acellular misfolded proteins
\item \textbf{Viroids} => infections nucleic acids w/o protein coat to make virus
\end{itemize}

\subsubsection{Protozoan}
\label{sec:orga8dfd86}
\begin{itemize}
\item \textbf{Protozoan factors} => direction pathogenisis leading to tissue damage
\item \textbf{Host-mediated factors} => immune evation + escape mechnisms +
immunalsupression
\end{itemize}

Adaptable!!

\subsubsection{Fungal}
\label{sec:org9d0455c}
\begin{itemize}
\item \textbf{Fungal factors} => many shapes and very adaptable, colud produced
specialized enzymes to take root in body
\item \textbf{Host-mediated factors} => cause immunocomprimzation, acquired though
inhalation, etc.
\end{itemize}

\subsubsection{Bacteria}
\label{sec:org051a901}
See
\href{KBhBIO101BacterialInfections.org}{KBhBIO101BacterialInfections}

\subsubsection{Viruses}
\label{sec:orgca98067}
See \href{KBhBIO101Viruses.org}{KBhBIO101Viruses}
\end{document}
