% Created 2021-09-11 Sat 16:41
% Intended LaTeX compiler: xelatex
\documentclass[letterpaper]{article}
\usepackage{graphicx}
\usepackage{grffile}
\usepackage{longtable}
\usepackage{wrapfig}
\usepackage{rotating}
\usepackage[normalem]{ulem}
\usepackage{amsmath}
\usepackage{textcomp}
\usepackage{amssymb}
\usepackage{capt-of}
\usepackage{hyperref}
\usepackage[margin=1in]{geometry}
\usepackage{fontspec}
\usepackage{indentfirst}
\setmainfont[ItalicFont = LiberationSans-Italic, BoldFont = LiberationSans-Bold, BoldItalicFont = LiberationSans-BoldItalic]{LiberationSans}
\newfontfamily\NHLight[ItalicFont = LiberationSansNarrow-Italic, BoldFont       = LiberationSansNarrow-Bold, BoldItalicFont = LiberationSansNarrow-BoldItalic]{LiberationSansNarrow}
\newcommand\textrmlf[1]{{\NHLight#1}}
\newcommand\textitlf[1]{{\NHLight\itshape#1}}
\let\textbflf\textrm
\newcommand\textulf[1]{{\NHLight\bfseries#1}}
\newcommand\textuitlf[1]{{\NHLight\bfseries\itshape#1}}
\usepackage{fancyhdr}
\pagestyle{fancy}
\usepackage{titlesec}
\usepackage{titling}
\makeatletter
\lhead{\textbf{\@title}}
\makeatother
\rhead{\textrmlf{Compiled} \today}
\lfoot{\theauthor\ \textbullet \ \textbf{2021-2022}}
\cfoot{}
\rfoot{\textrmlf{Page} \thepage}
\titleformat{\section} {\Large} {\textrmlf{\thesection} {|}} {0.3em} {\textbf}
\titleformat{\subsection} {\large} {\textrmlf{\thesubsection} {|}} {0.2em} {\textbf}
\titleformat{\subsubsection} {\large} {\textrmlf{\thesubsubsection} {|}} {0.1em} {\textbf}
\setlength{\parskip}{0.45em}
\renewcommand\maketitle{}
\author{Huxley}
\date{\today}
\title{Water Free-write + notes}
\hypersetup{
 pdfauthor={Huxley},
 pdftitle={Water Free-write + notes},
 pdfkeywords={},
 pdfsubject={},
 pdfcreator={Emacs 27.2 (Org mode 9.4.4)}, 
 pdflang={English}}
\begin{document}

\maketitle
\noindent\rule{\textwidth}{0.5pt}

\section{Molecular Dynamics}
\label{sec:org6e5ce23}
H02 to S02 -- water to sulfur-dioxide

Line represent polar attraction? Hydrogen bonds?

\begin{quote}
has a lot to do with electronegativity
\end{quote}

\section{Free-write Start}
\label{sec:org2d90ffa}
\begin{itemize}
\item What do you notice about the difference in movement between H20 and
S02? Why is this?

\begin{itemize}
\item The water molecules are significantly more spastic
\end{itemize}

\item Described what the emergent lines between the atoms represent?

\begin{itemize}
\item I believe the green lines are hydrogen bonds, which form due to the
polar nature of water molecules. The Hydrogens are slightly
positive, and the Oxygens slightly negative. The purple(ish) lines
are dipole bonds. I assume the blue lines are dispersion forces
simply due to their sheer quantity and instability.
\end{itemize}

\item What is there to possibly know about water? How does water support
life?

\begin{itemize}
\item Water in vital to life. It is a polar molecule, meaning it's an
amazing solvent. It also has high surface tension, and has high
adhesion This allows water to flow against gravity.
\end{itemize}

\item What are biological implications of water's properties?

\begin{itemize}
\item Allows water to regulate temperature, transfer nutrients + oxygen,
dissolve said nutrient,
\end{itemize}
\end{itemize}

running out of time here.. can we have more than ten minutes please?
This doesn't even scratch the surface of what I can write about water
from memory.
\end{document}
