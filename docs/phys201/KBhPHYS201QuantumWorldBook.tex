% Created 2021-09-11 Sat 09:36
% Intended LaTeX compiler: xelatex
\documentclass[letterpaper]{article}
\usepackage{graphicx}
\usepackage{grffile}
\usepackage{longtable}
\usepackage{wrapfig}
\usepackage{rotating}
\usepackage[normalem]{ulem}
\usepackage{amsmath}
\usepackage{textcomp}
\usepackage{amssymb}
\usepackage{capt-of}
\usepackage{hyperref}
\usepackage[margin=1in]{geometry}
\usepackage{fontspec}
\usepackage{indentfirst}
\setmainfont[ItalicFont = LiberationSans-Italic, BoldFont = LiberationSans-Bold, BoldItalicFont = LiberationSans-BoldItalic]{LiberationSans}
\newfontfamily\NHLight[ItalicFont = LiberationSansNarrow-Italic, BoldFont       = LiberationSansNarrow-Bold, BoldItalicFont = LiberationSansNarrow-BoldItalic]{LiberationSansNarrow}
\newcommand\textrmlf[1]{{\NHLight#1}}
\newcommand\textitlf[1]{{\NHLight\itshape#1}}
\let\textbflf\textrm
\newcommand\textulf[1]{{\NHLight\bfseries#1}}
\newcommand\textuitlf[1]{{\NHLight\bfseries\itshape#1}}
\usepackage{fancyhdr}
\pagestyle{fancy}
\usepackage{titlesec}
\usepackage{titling}
\makeatletter
\lhead{\textbf{\@title}}
\makeatother
\rhead{\textrmlf{Compiled} \today}
\lfoot{\theauthor\ \textbullet \ \textbf{2021-2022}}
\cfoot{}
\rfoot{\textrmlf{Page} \thepage}
\titleformat{\section} {\Large} {\textrmlf{\thesection} {|}} {0.3em} {\textbf}
\titleformat{\subsection} {\large} {\textrmlf{\thesubsection} {|}} {0.2em} {\textbf}
\titleformat{\subsubsection} {\large} {\textrmlf{\thesubsubsection} {|}} {0.1em} {\textbf}
\setlength{\parskip}{0.45em}
\renewcommand\maketitle{}
\author{Houjun Liu}
\date{\today}
\title{The Quantum World, Notes}
\hypersetup{
 pdfauthor={Houjun Liu},
 pdftitle={The Quantum World, Notes},
 pdfkeywords={},
 pdfsubject={},
 pdfcreator={Emacs 27.2 (Org mode 9.4.4)}, 
 pdflang={English}}
\begin{document}

\maketitle


\section{The Quantum World}
\label{sec:org1b3aa9c}
\begin{itemize}
\item Atoms are small, and the quantum world concern itself with sub-atomic
particles.
\item In the 1920s, Protons and Electrons are know to be the two things that
are subatomic

\begin{itemize}
\item Protons are hitting earth frequently, creating the "primary cosmic
radiation"
\end{itemize}

\item Photons also exists, but it has no mass

\begin{itemize}
\item Protons were not given article status until later, when electrons
are recognized also as being able to be created, anililated
\end{itemize}

\item The 1920s brought a bunch of things

\begin{itemize}
\item Matter, not just light, have wave-like properties
\item Fundimental laws of nature are on a probability curve
\item Electron spin was discovered
\item Antiparticle was discovered
\end{itemize}

\item Importantly, the properties vs the action of the particles often get
mixed up
\item The known particles are mostly built from combination of smaller
fundimental particals
\item Standard model for Subatomic Particles

\begin{itemize}
\item 24 subatomic particles
\end{itemize}
\end{itemize}

Fentometer (10\textsuperscript{-15} m) is the common unit of length. Speed of light,
3x10\textsuperscript{8} = C is the common unit for speed. e- charge as the common united
as charge. eV, voltage of electron, is the common unit for energy.

A Tachyon \emph{may} be able to travel faster than the speed of light. It is
theoretical, may go faster than light, and could break causality in some
reference frames. No one has found it.

\begin{itemize}
\item Gluon => Glue for particles within nucleaus
\item Pion => nuclear collisions driven particle
\end{itemize}

\subsection{Absolutes}
\label{sec:org74dd9cb}
\begin{itemize}
\item Shortest distance: 10 * -18m
\item Shortest time 10 * -26s
\item Longest time: 13.7 billion years
\end{itemize}

Mass: measure of how hard it is to set a stationary object into motion,
deflect, or stop a moving object.

So, to measure a particle's mass, we boink it around in a magnetic field
and measure its path.

\#ask Kinetic energy + mass energy (e=mc\textsuperscript{2}) = energy?

"Mass energy is porportional to mass". Mass represents a highly
concentrated form of energy; a little mass yields lots of energy,
meaning that a lot of energy is needed to make mass.

Humans have done this: if you take two protons and go kaboom by slaming
them together, you put a lot of energy in, you make new mass!

Energy and mass are tipically measured in the same unit => the Electron
Volt. And\ldots{} MeV, is "million electron volts."

Like charges repel; but how does the necleaous stay together? Gluons ---
gluons serve as the glue to glue particles together. But, heavy elements
have very high electrical force from change that cause things to fly
away, so gluons work\ldots{} to a point. This is why uranium+ atoms don't
exist in nature

Charge is measured in Coulomb => charge through a 100 watt light bulb in
a second

Rotation around center: spin

Rotation around object: orbital motion

Both are measured by angular momentum => angular momentum is quantized
by h-bar (Planck's constant/2pi). Meaning particles could only have
angular momentum 0, h-bar, 2h-bar, 3hbar, etc.

In theory, you spinning is also quantized like this, but you don't
notice it because\ldots{} scales.

Natural constant: Planck's Constant (h) \& the speed of light (c).
\end{document}
