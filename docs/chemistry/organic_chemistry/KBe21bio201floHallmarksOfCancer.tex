% Created 2021-09-12 Sun 22:49
% Intended LaTeX compiler: xelatex
\documentclass[letterpaper]{article}
\usepackage{graphicx}
\usepackage{grffile}
\usepackage{longtable}
\usepackage{wrapfig}
\usepackage{rotating}
\usepackage[normalem]{ulem}
\usepackage{amsmath}
\usepackage{textcomp}
\usepackage{amssymb}
\usepackage{capt-of}
\usepackage{hyperref}
\usepackage[margin=1in]{geometry}
\usepackage{fontspec}
\usepackage{indentfirst}
\setmainfont[ItalicFont = LiberationSans-Italic, BoldFont = LiberationSans-Bold, BoldItalicFont = LiberationSans-BoldItalic]{LiberationSans}
\newfontfamily\NHLight[ItalicFont = LiberationSansNarrow-Italic, BoldFont       = LiberationSansNarrow-Bold, BoldItalicFont = LiberationSansNarrow-BoldItalic]{LiberationSansNarrow}
\newcommand\textrmlf[1]{{\NHLight#1}}
\newcommand\textitlf[1]{{\NHLight\itshape#1}}
\let\textbflf\textrm
\newcommand\textulf[1]{{\NHLight\bfseries#1}}
\newcommand\textuitlf[1]{{\NHLight\bfseries\itshape#1}}
\usepackage{fancyhdr}
\pagestyle{fancy}
\usepackage{titlesec}
\usepackage{titling}
\makeatletter
\lhead{\textbf{\@title}}
\makeatother
\rhead{\textrmlf{Compiled} \today}
\lfoot{\theauthor\ \textbullet \ \textbf{2021-2022}}
\cfoot{}
\rfoot{\textrmlf{Page} \thepage}
\titleformat{\section} {\Large} {\textrmlf{\thesection} {|}} {0.3em} {\textbf}
\titleformat{\subsection} {\large} {\textrmlf{\thesubsection} {|}} {0.2em} {\textbf}
\titleformat{\subsubsection} {\large} {\textrmlf{\thesubsubsection} {|}} {0.1em} {\textbf}
\setlength{\parskip}{0.45em}
\renewcommand\maketitle{}
\author{Exr0n}
\date{\today}
\title{flo - hallmarks of cancer (2011) reading}
\hypersetup{
 pdfauthor={Exr0n},
 pdftitle={flo - hallmarks of cancer (2011) reading},
 pdfkeywords={},
 pdfsubject={},
 pdfcreator={Emacs 28.0.50 (Org mode 9.4.4)}, 
 pdflang={English}}
\begin{document}

\maketitle
\section{sources\hfill{}\textsc{source}}
\label{sec:orgc35aad1}
\subsection{assignment: \url{https://nuevaschool.instructure.com/courses/3087/assignments/56036}}
\label{sec:org43e3d7f}
\subsection{reading: \href{KBsrcHallmarksOfCancer2011Reading.pdf}{Hallmarks of Cancer PDF}}
\label{sec:orgb6e1e91}
\section{Flow}
\label{sec:orga90a951}
\subsection{Abstract}
\label{sec:orga67ee5a}
\subsubsection{hallmarks include}
\label{sec:org8955c8a}
\begin{enumerate}
\item sustaining proliferative signaling
\label{sec:org014facb}
\item evading growth suppressors
\label{sec:org2468ef2}
\item resisting cell death
\label{sec:orgdadb2c4}
\item enabling replicative immortality
\label{sec:org28133a4}
\item inducing ingiogenesis
\label{sec:org08717ae}
\item activating invasion and metastasis
\label{sec:org698c22c}
\end{enumerate}
\subsubsection{theese hallmarks are newer}
\label{sec:orgbb7e5f8}
\begin{enumerate}
\item reprogramming of energy metabolism
\label{sec:org1f1173d}
\item evading immune destruction
\label{sec:orgb46a5df}
\end{enumerate}
\subsubsection{underlying}
\label{sec:orgb68213b}
\begin{enumerate}
\item genome instability
\label{sec:orga396ae3}
\begin{enumerate}
\item genetic diversity that expedites acquisition of hallmarks
\label{sec:org1999db7}
\end{enumerate}
\item inflammation
\label{sec:orgb5a0e24}
\begin{enumerate}
\item "fosters multiple hallmark functions"
\label{sec:orgd28fe18}
\end{enumerate}
\end{enumerate}
\subsection{Introduction}
\label{sec:org16bb4f4}
\subsubsection{Cancer cells evolve into cancer cells because they need to be cancer cells??}
\label{sec:org60bd01c}
\begin{enumerate}
\item {\bfseries\sffamily TODO} why do tumors have "the need \ldots{} to acquire the traints that enable them to become tumorigenic and ultimately malignant"?\hfill{}\textsc{question}
\label{sec:org30b06aa}
\end{enumerate}
\subsubsection{tumors are not simple / idle 'insular masses of proliferating cancer cells'}
\label{sec:org85e338d}
\subsubsection{"recruited" normal cells (or 'stromal cells') are active parts of the tumor}
\label{sec:orgdeb22e8}
\subsubsection{'the biology of tumors can no longer be understood simply by enumerating the traits of the cancer cells but instead must encompass the contributions of the "tumor microenvironment" to tumorigenesis.'}
\label{sec:org4ba4db8}
\subsubsection{purpose is to consider new hallmarks that have been found or note that old ones weren't as general as we thought}
\label{sec:orgd6aefb5}
\subsection{section: 'An Emerging Hallmark: Evading Immune Destruction'}
\label{sec:org6211ff5}
\subsubsection{the immune system usually eradicates the 'formation and progression of incipient neoplasias, late-stage tumors, and micrometastases', so why not in these cancers?}
\label{sec:orgf0e76d0}
\subsubsection{'long standing theory of immune surveillance' -> something went interesting}
\label{sec:org1a2fa11}
\begin{enumerate}
\item 'cells and tissues are constantly monitored'
\label{sec:orgbcb670f}
\item surveillance should elim cancer cells before they grow into tumors
\label{sec:org1958244}
\item thus, grown tumors have either hid from surveillance or limited the 'extent of immunological killing'
\label{sec:org08a6e38}
\end{enumerate}
\subsubsection{more cancer in immunocompromised individuals}
\label{sec:orgcc10ea1}
\begin{enumerate}
\item but these are virus-induced cancers
\label{sec:orgb6c096b}
\begin{enumerate}
\item so helping these people = reducing viral infilltration
\label{sec:org906e74f}
\end{enumerate}
\item other cancers still evade the immune system
\label{sec:org00a7b8b}
\end{enumerate}
\subsubsection{'genetically engineered mice and clinical epidemeology suggest that the immune system' actually hurts cancer}
\label{sec:org7b31ace}
\begin{enumerate}
\item mice that are engineered to lack some immune parts got cancer faster/stronger/more
\label{sec:org6189776}
\begin{enumerate}
\item these guys are important in fighting cancer
\label{sec:orgaa37f7b}
\begin{enumerate}
\item CD8\^{}+ cytotoxic T lymphocytes (CTLs)
\label{sec:org8dadf57}
\item CD4\^{}+ T\textsubscript{h1} helper T cells
\label{sec:org366c69a}
\item natural killer (NK) cells
\label{sec:org97d504c}
\end{enumerate}
\item 'demonstrable increase in tumor incidence'
\label{sec:org164403f}
\item lacking multiple -> 'more susceptible to cancer development'
\label{sec:orga122eca}
\item 'both the innate and adaptive cellular arms of the immune system are able to contribute significantly to immune surveillance and thus tumor eradication'\hfill{}\textsc{conclusion}
\label{sec:org02ef71c}
\end{enumerate}
\end{enumerate}
\subsubsection{transplantation experiments}
\label{sec:org24e4dc4}
\begin{enumerate}
\item cancer cells from immunodeficient mice have a bad time in normal mice
\label{sec:org50b74bb}
\item cancer cells from normal mice can initiate tumors in both types of hosts
\label{sec:org36a5a02}
\item maybe some cancer cells are more easily detected and those would normally die in normal hosts but live in comprimised hosts, but when transplanted they meet a competent immune system and die\hfill{}\textsc{conclusion}
\label{sec:orgb48e4a7}
\item Open question: do some carcinogens tend to induce more/less immunogenic cancer cells?\hfill{}\textsc{nextstep}
\label{sec:orgb005a6b}
\end{enumerate}
\subsubsection{the immune system probably includes antitumoral responses}
\label{sec:org285da1c}
\begin{enumerate}
\item patients with colon and ovarian tumors who have lots of CTLs and NK cells have better prognosis
\label{sec:orgc131300}
\begin{enumerate}
\item evidence is not as strong for other cancers\hfill{}\textsc{nextstep}
\label{sec:org0810c9c}
\end{enumerate}
\item immunosupressed organ recievers got cancer from the donor
\label{sec:orga128ffc}
\begin{enumerate}
\item suggests doner had immune system which held cancer down until organ was transplanted
\label{sec:org6de618a}
\end{enumerate}
\end{enumerate}
\subsubsection{{\bfseries\sffamily TODO} 'still, the epidemiology of chronically immunosupressed patients does not indicate significantly increased incidences of the major forms of nonviral human cancer'}
\label{sec:org81e049e}
\subsubsection{{\bfseries\sffamily TODO} something about HIV patients who lack T and B cells and how they should still be able to fight cancer with NK cells and CTLs}
\label{sec:org966608f}
\subsubsection{that was oversimplified as the tumor might also be actively supressing immune responses}
\label{sec:orgd6ae206}
\begin{enumerate}
\item may 'paralyze infiltrating CTLs and NK cells by secreting TGF-\(\beta\) or other immunosuppressive factors'
\label{sec:org2658e12}
\item 'more subtle mechinisms .. recruitment of inflamatory cells that are actively immunosuppressive'
\label{sec:org5d2b491}
\begin{enumerate}
\item 'regulatory T cells (Tregs) and myeloid-derived suppressor cells (MDSCs)'
\label{sec:org7343bd6}
\end{enumerate}
\end{enumerate}
\subsubsection{it is so far unclear whether the immune system plays a large enough role to be considered a hallmark of cancer}
\label{sec:org2c00148}
\section{Vocab}
\label{sec:org93ff29e}
\subsection{neoplastic disease}
\label{sec:org07d98a7}
\subsubsection{anything that causes tumor growth (malignant or benign)}
\label{sec:orgf9dc168}
\subsection{ostensibly}
\label{sec:org0eaa299}
\subsubsection{maybe 'technically'?'}
\label{sec:org04b00d8}
\subsection{tumor microenvironment}
\label{sec:orge8a33bb}
\subsubsection{presumably inflammation, recruited normal cells, and other stuff that helps the tumor grow}
\label{sec:org6dfc83d}
\subsection{pathogenisis}
\label{sec:org912943d}
\subsubsection{evolution of 'pathogen' (cancer)}
\label{sec:org266fb1b}
\subsection{ancillary proposition}
\label{sec:org1103d1c}
\subsubsection{maybe the starting / base proposition}
\label{sec:org98054b1}
\subsection{insular masses}
\label{sec:orga06a7c9}
\subsubsection{stagnant or something, simple}
\label{sec:org8c47369}
\subsection{heterotypic interactions}
\label{sec:org2b194de}
\subsubsection{many types of interactions}
\label{sec:org98bd6e5}
\subsection{tumorigenisis}
\label{sec:orgac4335b}
\subsubsection{the growth / development of a tumor?}
\label{sec:org2c25a16}
\subsection{neoplasias}
\label{sec:org3c845eb}
\subsubsection{new uncontrolled growth of cells}
\label{sec:org6bf3b19}
\subsection{micrometastases}
\label{sec:orgb3ace53}
\subsubsection{clumps of cancer cells that spread around the body}
\label{sec:org5201fa3}
\subsection{etiology}
\label{sec:org1045698}
\subsubsection{study of cancer?}
\label{sec:orgab49acb}
\subsection{immunogenic}
\label{sec:org20681a3}
\subsubsection{easily detected by the immune system}
\label{sec:org8cdac0b}
\subsection{immunoediting}
\label{sec:orgce80c51}
\subsubsection{"natural selection" by the immune system}
\label{sec:orgf0dec10}
\subsection{prognosis}
\label{sec:orgaaa1da2}
\subsubsection{a prediction of the outcome of a cancer (or disease in general, apparently)}
\label{sec:org750c528}
\end{document}
