% Created 2021-09-11 Sat 09:35
% Intended LaTeX compiler: xelatex
\documentclass[letterpaper]{article}
\usepackage{graphicx}
\usepackage{grffile}
\usepackage{longtable}
\usepackage{wrapfig}
\usepackage{rotating}
\usepackage[normalem]{ulem}
\usepackage{amsmath}
\usepackage{textcomp}
\usepackage{amssymb}
\usepackage{capt-of}
\usepackage{hyperref}
\usepackage[margin=1in]{geometry}
\usepackage{fontspec}
\usepackage{indentfirst}
\setmainfont[ItalicFont = LiberationSans-Italic, BoldFont = LiberationSans-Bold, BoldItalicFont = LiberationSans-BoldItalic]{LiberationSans}
\newfontfamily\NHLight[ItalicFont = LiberationSansNarrow-Italic, BoldFont       = LiberationSansNarrow-Bold, BoldItalicFont = LiberationSansNarrow-BoldItalic]{LiberationSansNarrow}
\newcommand\textrmlf[1]{{\NHLight#1}}
\newcommand\textitlf[1]{{\NHLight\itshape#1}}
\let\textbflf\textrm
\newcommand\textulf[1]{{\NHLight\bfseries#1}}
\newcommand\textuitlf[1]{{\NHLight\bfseries\itshape#1}}
\usepackage{fancyhdr}
\pagestyle{fancy}
\usepackage{titlesec}
\usepackage{titling}
\makeatletter
\lhead{\textbf{\@title}}
\makeatother
\rhead{\textrmlf{Compiled} \today}
\lfoot{\theauthor\ \textbullet \ \textbf{2021-2022}}
\cfoot{}
\rfoot{\textrmlf{Page} \thepage}
\titleformat{\section} {\Large} {\textrmlf{\thesection} {|}} {0.3em} {\textbf}
\titleformat{\subsection} {\large} {\textrmlf{\thesubsection} {|}} {0.2em} {\textbf}
\titleformat{\subsubsection} {\large} {\textrmlf{\thesubsubsection} {|}} {0.1em} {\textbf}
\setlength{\parskip}{0.45em}
\renewcommand\maketitle{}
\author{Exr0n}
\date{\today}
\title{Ret Easy 3}
\hypersetup{
 pdfauthor={Exr0n},
 pdftitle={Ret Easy 3},
 pdfkeywords={},
 pdfsubject={},
 pdfcreator={Emacs 27.2 (Org mode 9.4.4)}, 
 pdflang={English}}
\begin{document}

\maketitle


\section{Reading}
\label{sec:orgd7e47a8}
\subsection{Openstax}
\label{sec:org673edc4}
\href{https://openstax.org/books/calculus-volume-1/pages/2-4-continuity}{Link}

\begin{itemize}
\item \#define continuity at a point
\item \[ \lim_{x\to a}f(x) = f(a) \]

\begin{itemize}
\item To ensure that it is defined, connected on both sides, and doesn't
have a random point
\item To check for continuity, just check for \(f(a)\),
\(\lim_{x\to a}f(x)\), and that they are equal
\end{itemize}

\item Rational functions

\begin{itemize}
\item Are continuous on their domains

\begin{itemize}
\item Basically anywhere they are defined
\end{itemize}
\end{itemize}

\item Discontinuity types

\begin{itemize}
\item Removable discontinuities

\begin{itemize}
\item Hole in the graph
\end{itemize}

\item infinite is continuity

\begin{itemize}
\item asymtote
\end{itemize}

\item jump discontinuity
\end{itemize}

\item Continuity from the right and left

\begin{itemize}
\item Same as definition of continuous, but replace the limit with right
and left hand limits respectively
\end{itemize}
\end{itemize}

\subsection{libretexts}
\label{sec:org9ca96e8}
\href{https://math.libretexts.org/Bookshelves/Calculus/Book\%3A\_Calculus\_(Apex)/01\%3A\_Limits/1.05\%3A\_Continuity}{Link} -
Basically the same thing - Properties of continuous functions (group
like bits) - > Let 𝑓 and 𝑔 be continuous functions on an interval 𝐼 ,
let 𝑐 be a real number and let 𝑛 be a positive integer. The following
functions are continuous on 𝐼 . > - Sums/Differences : 𝑓±𝑔 > - Constant
Multiples : 𝑐⋅𝑓 > - Products : 𝑓⋅𝑔 > - Quotients : 𝑓/𝑔 (as long as 𝑔≠0
on 𝐼 ) > - Powers : 𝑓𝑛 > - Roots : \(f(x) = \sqrt[n]{x}\) (if 𝑛 is even
then 𝑓≥0 on 𝐼 ; if 𝑛 is odd, then true for all values of 𝑓 on 𝐼 .) > -
Compositions : Adjust the definitions of 𝑓 and 𝑔 to: Let 𝑓 be continuous
on 𝐼, where the range of 𝑓 on 𝐼 is 𝐽 , and let 𝑔 be continuous on 𝐽.
Then 𝑔∘𝑓, i.e., 𝑔(𝑓(𝑥)), is continuous on 𝐼. -

\noindent\rule{\textwidth}{0.5pt}
\end{document}
