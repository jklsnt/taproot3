% Created 2021-09-11 Sat 09:37
% Intended LaTeX compiler: xelatex
\documentclass[letterpaper]{article}
\usepackage{graphicx}
\usepackage{grffile}
\usepackage{longtable}
\usepackage{wrapfig}
\usepackage{rotating}
\usepackage[normalem]{ulem}
\usepackage{amsmath}
\usepackage{textcomp}
\usepackage{amssymb}
\usepackage{capt-of}
\usepackage{hyperref}
\usepackage[margin=1in]{geometry}
\usepackage{fontspec}
\usepackage{indentfirst}
\setmainfont[ItalicFont = LiberationSans-Italic, BoldFont = LiberationSans-Bold, BoldItalicFont = LiberationSans-BoldItalic]{LiberationSans}
\newfontfamily\NHLight[ItalicFont = LiberationSansNarrow-Italic, BoldFont       = LiberationSansNarrow-Bold, BoldItalicFont = LiberationSansNarrow-BoldItalic]{LiberationSansNarrow}
\newcommand\textrmlf[1]{{\NHLight#1}}
\newcommand\textitlf[1]{{\NHLight\itshape#1}}
\let\textbflf\textrm
\newcommand\textulf[1]{{\NHLight\bfseries#1}}
\newcommand\textuitlf[1]{{\NHLight\bfseries\itshape#1}}
\usepackage{fancyhdr}
\pagestyle{fancy}
\usepackage{titlesec}
\usepackage{titling}
\makeatletter
\lhead{\textbf{\@title}}
\makeatother
\rhead{\textrmlf{Compiled} \today}
\lfoot{\theauthor\ \textbullet \ \textbf{2021-2022}}
\cfoot{}
\rfoot{\textrmlf{Page} \thepage}
\titleformat{\section} {\Large} {\textrmlf{\thesection} {|}} {0.3em} {\textbf}
\titleformat{\subsection} {\large} {\textrmlf{\thesubsection} {|}} {0.2em} {\textbf}
\titleformat{\subsubsection} {\large} {\textrmlf{\thesubsubsection} {|}} {0.1em} {\textbf}
\setlength{\parskip}{0.45em}
\renewcommand\maketitle{}
\author{Exr0n}
\date{\today}
\title{Axler6.B \#9}
\hypersetup{
 pdfauthor={Exr0n},
 pdftitle={Axler6.B \#9},
 pdfkeywords={},
 pdfsubject={},
 pdfcreator={Emacs 27.2 (Org mode 9.4.4)}, 
 pdflang={English}}
\begin{document}

\maketitle
\section{Problem}
\label{sec:orge16035f}
\begin{quote}
What happens if the Gram–Schmidt Procedure is applied to a list of vectors that is not linearly independent?
\end{quote}
\section{Answer}
\label{sec:orgc05875c}
Suppose the list \(v_1, \ldots, v_n\) is linearly dependent. Then, there exists some \(v_j\) s.t. \(v_1, \ldots, v_{j-1}\) is linearly independent while \(v_1, \ldots, v_j\) is not. Then, \(v_j \in \text{span}(v_1, \ldots, v_{j-1})\)

Because the Gram-Schmidt procedure preserves prefix spans,
\[ v_j \in \text{span}(e_1, \ldots, e_{j-1}) \]

Because of how a vector is written as a linear combination of an orthonormal basis, the denominator in the \$j\$-th step of the procedure is equivalent to
\[
  \lVert v - v \rVert = \lVert 0 \rVert = 0
  \]
and a division by zero occurs. Thus, the Gram-Schmidt procedure cannot be used on a linearly dependent list.
\end{document}
