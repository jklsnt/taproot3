% Created 2021-09-11 Sat 08:18
% Intended LaTeX compiler: xelatex
\documentclass[letterpaper]{article}
\usepackage{graphicx}
\usepackage{grffile}
\usepackage{longtable}
\usepackage{wrapfig}
\usepackage{rotating}
\usepackage[normalem]{ulem}
\usepackage{amsmath}
\usepackage{textcomp}
\usepackage{amssymb}
\usepackage{capt-of}
\usepackage{hyperref}
\usepackage[margin=1in]{geometry}
\usepackage{fontspec}
\usepackage{indentfirst}
\setmainfont[ItalicFont = LiberationSans-Italic, BoldFont = LiberationSans-Bold, BoldItalicFont = LiberationSans-BoldItalic]{LiberationSans}
\newfontfamily\NHLight[ItalicFont = LiberationSansNarrow-Italic, BoldFont       = LiberationSansNarrow-Bold, BoldItalicFont = LiberationSansNarrow-BoldItalic]{LiberationSansNarrow}
\newcommand\textrmlf[1]{{\NHLight#1}}
\newcommand\textitlf[1]{{\NHLight\itshape#1}}
\let\textbflf\textrm
\newcommand\textulf[1]{{\NHLight\bfseries#1}}
\newcommand\textuitlf[1]{{\NHLight\bfseries\itshape#1}}
\usepackage{fancyhdr}
\pagestyle{fancy}
\usepackage{titlesec}
\usepackage{titling}
\makeatletter
\lhead{\textbf{\@title}}
\makeatother
\rhead{\textrmlf{Compiled} \today}
\lfoot{\theauthor\ \textbullet \ \textbf{2021-2022}}
\cfoot{}
\rfoot{\textrmlf{Page} \thepage}
\titleformat{\section} {\Large} {\textrmlf{\thesection} {|}} {0.3em} {\textbf}
\titleformat{\subsection} {\large} {\textrmlf{\thesubsection} {|}} {0.2em} {\textbf}
\titleformat{\subsubsection} {\large} {\textrmlf{\thesubsubsection} {|}} {0.1em} {\textbf}
\setlength{\parskip}{0.45em}
\renewcommand\maketitle{}
\author{Exr0n}
\date{\today}
\title{Linalg Flow 13}
\hypersetup{
 pdfauthor={Exr0n},
 pdftitle={Linalg Flow 13},
 pdfkeywords={},
 pdfsubject={},
 pdfcreator={Emacs 27.2 (Org mode 9.4.4)}, 
 pdflang={English}}
\begin{document}

\maketitle
\#flo \#disorganized \#incomplete

\section{Bases}
\label{sec:org1c48ee8}
\begin{itemize}
\item \#icr \href{KBeRefLinAlgBases.org}{KBeRefLinAlgBases}
\item (plural of 'basis')
\item the \emph{standard basis} of
\(\mathbb{F}^n = (1, 0, 0, \ldots 0), (0, 1, 0, \ldots, 0), \ldots, (0, 0, 0, \ldots, 1)\)
\end{itemize}

\subsection{Axler2.31}
\label{sec:org60073a1}
\begin{itemize}
\item The first step is a shortcut: the only way the first vector could be
dependent is if it's zero

\begin{itemize}
\item but as you go on, you are more and more "likely" to have redundancy
\end{itemize}

\item How to check the third vector?

\begin{itemize}
\item Calculate the formula for the plane of the first two (cross product
to get something orthogonal)
\item Then, take the dot product to see if the third vector is
perpendicular to the orthogonal and therefore in the plane.

\begin{itemize}
\item (the third one is in the plane if the dot product with the
orthogonal is zero)
\end{itemize}
\end{itemize}

\item This is a more "concrete" ? version of the linear dependence lemma
(see it in \href{KBe20math530flo11.org}{KBe20math530flo11})

\begin{itemize}
\item The linear dependence lemma starts from the end, where as this
algorithm starts from the beginning
\item it basically says "there will be one you can get rid of" while this
one tells you which:

\begin{itemize}
\item for the sake of the algorithm, throw away the one that comes
later.
\end{itemize}
\end{itemize}
\end{itemize}

\subsection{Axler2.34}
\label{sec:orge8dd6cf}
\begin{itemize}
\item You can start with a whole space and knock out of the basis, or a
subspace and extend the basis
\item to show that it's a direct sum

\begin{itemize}
\item show that the sum works

\begin{itemize}
\item because we expanded to make the list a basis of \(V\), it must
span \(V\) so all vectors are contained within.
\end{itemize}

\item show that the intersection works

\begin{itemize}
\item we know that \(v\) is in both \(U\) and \(W\), so we can subtract
the equations and get that the sum is zero
\item and because the list is a basis, it is linearly independent, so
all the coefficients are zero, so \(v\) must be \(0\)
\end{itemize}
\end{itemize}
\end{itemize}

\section{Connections of Linear Independence}
\label{sec:orgaf342e2}
\#toexpand \#incomplete \#icr of the recent test \#\# Linear Independence and
Systems of Equations - A system of equations's coefficients written as
vectors has one solution if and only if the vectors are linearly
independent - A worked example: $\backslash$[
\begin{aligned}
3x-y+z = 5\\
x + 2y + z = 0\\
4x+y+2z = 2
\end{aligned}
$\backslash$] becomes $\backslash$[
\begin{aligned}
\begin{bmatrix}
3&-1&1\\
1&2&1\\
4&1&1\\
\end{bmatrix}
\begin{bmatrix}
x\\
y\\
z\\
\end{bmatrix}
= 
\begin{bmatrix}
5\\0\\2
\end{bmatrix}
\end{aligned}
$\backslash$]

When the system above is graphed, the third plane is parallel to the
line of intersection between the first two. This means that the vectors
(the rows of the components of the vectors in \(\mathbb{R}^3\))are
linearly dependent?

\begin{itemize}
\item \textbf{linear combinations of the column vectors}
\item \textbf{The fact that this system is not uniquely determined is a fact of the
left side.} It doesn't matter what the numbers on the right of the
system are. There will be \(\infty\) solutions if the solution column
is in plane of the column vectors, or in the span of the list.
\item Column vectors are coplanar if the row vectors are linearly dependent.
\end{itemize}

\section{A homogeneous system}
\label{sec:org441e787}
\begin{itemize}
\item When everything on the right of the systems are zero
\end{itemize}

\#todo-exr0n move to linalg questions - questions - standard basis of
polynomials - \(z^0, z^1, z^2, \ldots, z^n\) - system of equations of
polynomials - less obvious - turn your polynomial space into a more
standard vector space and then use that?

\subsection{Linear Independence and Direct Sums}
\label{sec:orgaa0169b}
\begin{itemize}
\item In a direct sum, every vector in that sum can only be represented in
one way
\item for linear independence the span of the list is only represented as
one linear combination of the list (esp. 0, which is the definition of
linear independence)
\end{itemize}

\section{Referencing}
\label{sec:orgd0180dd}
\begin{itemize}
\item Say either the number or the title, or both.

\begin{itemize}
\item title is useful, number is specific
\end{itemize}
\end{itemize}

\noindent\rule{\textwidth}{0.5pt}
\end{document}
