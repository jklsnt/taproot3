% Created 2021-09-11 Sat 08:18
% Intended LaTeX compiler: xelatex
\documentclass[letterpaper]{article}
\usepackage{graphicx}
\usepackage{grffile}
\usepackage{longtable}
\usepackage{wrapfig}
\usepackage{rotating}
\usepackage[normalem]{ulem}
\usepackage{amsmath}
\usepackage{textcomp}
\usepackage{amssymb}
\usepackage{capt-of}
\usepackage{hyperref}
\usepackage[margin=1in]{geometry}
\usepackage{fontspec}
\usepackage{indentfirst}
\setmainfont[ItalicFont = LiberationSans-Italic, BoldFont = LiberationSans-Bold, BoldItalicFont = LiberationSans-BoldItalic]{LiberationSans}
\newfontfamily\NHLight[ItalicFont = LiberationSansNarrow-Italic, BoldFont       = LiberationSansNarrow-Bold, BoldItalicFont = LiberationSansNarrow-BoldItalic]{LiberationSansNarrow}
\newcommand\textrmlf[1]{{\NHLight#1}}
\newcommand\textitlf[1]{{\NHLight\itshape#1}}
\let\textbflf\textrm
\newcommand\textulf[1]{{\NHLight\bfseries#1}}
\newcommand\textuitlf[1]{{\NHLight\bfseries\itshape#1}}
\usepackage{fancyhdr}
\pagestyle{fancy}
\usepackage{titlesec}
\usepackage{titling}
\makeatletter
\lhead{\textbf{\@title}}
\makeatother
\rhead{\textrmlf{Compiled} \today}
\lfoot{\theauthor\ \textbullet \ \textbf{2021-2022}}
\cfoot{}
\rfoot{\textrmlf{Page} \thepage}
\titleformat{\section} {\Large} {\textrmlf{\thesection} {|}} {0.3em} {\textbf}
\titleformat{\subsection} {\large} {\textrmlf{\thesubsection} {|}} {0.2em} {\textbf}
\titleformat{\subsubsection} {\large} {\textrmlf{\thesubsubsection} {|}} {0.1em} {\textbf}
\setlength{\parskip}{0.45em}
\renewcommand\maketitle{}
\author{Taproot}
\date{\today}
\title{Orthonormal Basis}
\hypersetup{
 pdfauthor={Taproot},
 pdftitle={Orthonormal Basis},
 pdfkeywords={},
 pdfsubject={},
 pdfcreator={Emacs 27.2 (Org mode 9.4.4)}, 
 pdflang={English}}
\begin{document}

\maketitle
\section{Axler6.27 orthonormal basis\hfill{}\textsc{def}}
\label{sec:org5ac9708}
\begin{quote}
An \emph{orthonormal basis} of \(V\) is an orthonormal list of vectors in \(V\) that is also a basis of \(V\).
\end{quote}

Pretty self explanatory.

How do we find an orthonormal basis? see the
\section{results}
\label{sec:orgf4c6266}
\subsection{Axler6.28 orthonormal list of the right length is a basis}
\label{sec:org6258e75}
Because it's linearly independent, and linearly independent lists of the right length are bases (Axler2.39).
\subsection{Axler6.30 vector as a linear combo of orthonormal basis}
\label{sec:orgbfdf127}
Suppose \(e_1, \ldots, e_m\) is an orthonormal basis of \(V\) and \(v \in  V\). Then,
\[\begin{aligned}
   v = \langle  v, e_1 \rangle e_1 + \cdots \langle  v, e_n \rangle e_n
   \end{aligned}\]
and

\[\begin{aligned}
   \lVert v \rVert ^2 = |\langle v, e_1 \rangle|^2 + \cdots + \lvert \langle  v, e_n \rangle \rvert ^2
   \end{aligned}\]

By taking the inner product of both sides of the equation
\[\begin{aligned}
   v = a_1e_1 + \cdots + a_n e_n
   \end{aligned}\]
with \(e_j\) for each \(e_j\). And also the Pythagorean theorem

\section{see also}
\label{sec:orge821a5d}
\subsection{\href{KBrefOrthonormal.org}{orthonormal}}
\label{sec:orgf42a324}
\end{document}
