% Created 2021-09-11 Sat 09:36
% Intended LaTeX compiler: xelatex
\documentclass[letterpaper]{article}
\usepackage{graphicx}
\usepackage{grffile}
\usepackage{longtable}
\usepackage{wrapfig}
\usepackage{rotating}
\usepackage[normalem]{ulem}
\usepackage{amsmath}
\usepackage{textcomp}
\usepackage{amssymb}
\usepackage{capt-of}
\usepackage{hyperref}
\usepackage[margin=1in]{geometry}
\usepackage{fontspec}
\usepackage{indentfirst}
\setmainfont[ItalicFont = LiberationSans-Italic, BoldFont = LiberationSans-Bold, BoldItalicFont = LiberationSans-BoldItalic]{LiberationSans}
\newfontfamily\NHLight[ItalicFont = LiberationSansNarrow-Italic, BoldFont       = LiberationSansNarrow-Bold, BoldItalicFont = LiberationSansNarrow-BoldItalic]{LiberationSansNarrow}
\newcommand\textrmlf[1]{{\NHLight#1}}
\newcommand\textitlf[1]{{\NHLight\itshape#1}}
\let\textbflf\textrm
\newcommand\textulf[1]{{\NHLight\bfseries#1}}
\newcommand\textuitlf[1]{{\NHLight\bfseries\itshape#1}}
\usepackage{fancyhdr}
\pagestyle{fancy}
\usepackage{titlesec}
\usepackage{titling}
\makeatletter
\lhead{\textbf{\@title}}
\makeatother
\rhead{\textrmlf{Compiled} \today}
\lfoot{\theauthor\ \textbullet \ \textbf{2021-2022}}
\cfoot{}
\rfoot{\textrmlf{Page} \thepage}
\titleformat{\section} {\Large} {\textrmlf{\thesection} {|}} {0.3em} {\textbf}
\titleformat{\subsection} {\large} {\textrmlf{\thesubsection} {|}} {0.2em} {\textbf}
\titleformat{\subsubsection} {\large} {\textrmlf{\thesubsubsection} {|}} {0.1em} {\textbf}
\setlength{\parskip}{0.45em}
\renewcommand\maketitle{}
\author{Exr0n}
\date{\today}
\title{Eigenstuff Followup}
\hypersetup{
 pdfauthor={Exr0n},
 pdftitle={Eigenstuff Followup},
 pdfkeywords={},
 pdfsubject={},
 pdfcreator={Emacs 27.2 (Org mode 9.4.4)}, 
 pdflang={English}}
\begin{document}

\maketitle
\section{Algebreic and Geometric Multiplicities}
\label{sec:orgf6418d7}
I missed the last ten minutes of class and had to look up what the algebreic and geometric multiplicities are. I used this \href{https://people.math.carleton.ca/\~kcheung/math/notes/MATH1107/wk10/10\_algebraic\_and\_geometric\_multiplicities.html}{source}.

Also it says something about
\begin{quote}
It is a fact that summing up the algebraic multiplicities of all the eigenvalues of an \(n\times n\) matrix \(A\) gives exactly \(n\).
\end{quote}
Which reminds me of the fundamental theorem of algebra\ldots{}

\subsection{\(\begin{pmatrix}4&-12\\2&0\end{pmatrix}\)}
\label{sec:org784a9e2}

\subsubsection{Geometric multiplicity}
\label{sec:org0040023}
The null space is \(\text{span } \begin{pmatrix}1\\1\end{pmatrix}\) which is dimension \(\boxed{1}\).

\subsubsection{Algebraic multiplicity}
\label{sec:orge95a298}
The determinant of \(\begin{pmatrix}2&-2\\2&-2\end{pmatrix}\) is
\[ -\lambda(4-\lambda) - (-4) = \lambda ^2 -4\lambda + 4 = (\lambda -2)^2 \]
So the algebraic multiplicity is \(\boxed{2}\)

\subsection{\(\begin{pmatrix}1&1&2\\0&1&-1\\0&0&3\end{pmatrix}\)}
\label{sec:org0b5ca4b}

\subsubsection{Geometric}
\label{sec:orgde5a514}
Null space of 1 (\((x, 0, 0)\)) has dim 1. Null space of 3 (\(\left(x, \frac{-2x}{3}, \frac{4x}{3}\right)\)) has dim 1 as well.

\subsubsection{Algebraic}
\label{sec:org5961f0f}
The determinant simplifies to one factored term:
\[ (1-\lambda)^2(3-\lambda) \]
So 1 has a multiplicity 2 and 3 has multiplicity 1?


\subsection{\(\begin{pmatrix}1&0&2\\0&1&-1\\0&0&3\end{pmatrix}\)}
\label{sec:orgc08d72b}

\subsubsection{Geometric}
\label{sec:orge31d24a}
For \(\lambda = 1\), null space is \((x, y, 0)\) so dim 2. For \(\lambda = 3\), null space is \((x, \frac{-x}{2}, x)\) so dim 1.

\subsubsection{Algebraic}
\label{sec:org44e3111}
The determinant is the same as the previous matrix, so once again, 1 has multiplicity 2 and 3 has multiplicity 1.
\end{document}
