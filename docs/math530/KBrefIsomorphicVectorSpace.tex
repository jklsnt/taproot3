% Created 2021-09-11 Sat 08:18
% Intended LaTeX compiler: xelatex
\documentclass[letterpaper]{article}
\usepackage{graphicx}
\usepackage{grffile}
\usepackage{longtable}
\usepackage{wrapfig}
\usepackage{rotating}
\usepackage[normalem]{ulem}
\usepackage{amsmath}
\usepackage{textcomp}
\usepackage{amssymb}
\usepackage{capt-of}
\usepackage{hyperref}
\usepackage[margin=1in]{geometry}
\usepackage{fontspec}
\usepackage{indentfirst}
\setmainfont[ItalicFont = LiberationSans-Italic, BoldFont = LiberationSans-Bold, BoldItalicFont = LiberationSans-BoldItalic]{LiberationSans}
\newfontfamily\NHLight[ItalicFont = LiberationSansNarrow-Italic, BoldFont       = LiberationSansNarrow-Bold, BoldItalicFont = LiberationSansNarrow-BoldItalic]{LiberationSansNarrow}
\newcommand\textrmlf[1]{{\NHLight#1}}
\newcommand\textitlf[1]{{\NHLight\itshape#1}}
\let\textbflf\textrm
\newcommand\textulf[1]{{\NHLight\bfseries#1}}
\newcommand\textuitlf[1]{{\NHLight\bfseries\itshape#1}}
\usepackage{fancyhdr}
\pagestyle{fancy}
\usepackage{titlesec}
\usepackage{titling}
\makeatletter
\lhead{\textbf{\@title}}
\makeatother
\rhead{\textrmlf{Compiled} \today}
\lfoot{\theauthor\ \textbullet \ \textbf{2021-2022}}
\cfoot{}
\rfoot{\textrmlf{Page} \thepage}
\titleformat{\section} {\Large} {\textrmlf{\thesection} {|}} {0.3em} {\textbf}
\titleformat{\subsection} {\large} {\textrmlf{\thesubsection} {|}} {0.2em} {\textbf}
\titleformat{\subsubsection} {\large} {\textrmlf{\thesubsubsection} {|}} {0.1em} {\textbf}
\setlength{\parskip}{0.45em}
\renewcommand\maketitle{}
\author{Exr0n}
\date{\today}
\title{Isomorphic vector spaces and isomorphisms}
\hypersetup{
 pdfauthor={Exr0n},
 pdftitle={Isomorphic vector spaces and isomorphisms},
 pdfkeywords={},
 pdfsubject={},
 pdfcreator={Emacs 27.2 (Org mode 9.4.4)}, 
 pdflang={English}}
\begin{document}

\maketitle
\section{Isomorphism\hfill{}\textsc{def}}
\label{sec:orgd7fbf00}
\begin{quote}
An \emph{isomorphism} is an invertible linear map
\end{quote}
\section{Isomorphic\hfill{}\textsc{def}}
\label{sec:orgcfb9d24}
\begin{quote}
Two vector spaces are called \emph{isomorphic} if there is an isomorphism from one vector space into the other
\end{quote}
\subsection{intuition}
\label{sec:org766e68f}
Can be thought of as relabeling each element \(v\) from one space into an element \(Tv\) in the other.
\subsection{results}
\label{sec:orgf3644bb}
\subsubsection{equal dimension iff isomorphic Axler3.59}
\label{sec:org3b6a4b6}
Two vector spaces over some field \(\mathbb F\) are isomorphic iff they have the same dimension.
\subsubsection{\(\mathcal L(V, W)\) and \(\mathbb F^{m, n}\) are isomorphic}
\label{sec:org457cc24}
\begin{quote}
Given two bases of \(V\) and \(W\), \(\mathcal M\) is an isomorphism between \(\mathcal L(V, W)\) and \(\mathbb F^{m, n}\)
\end{quote}
\subsubsection{Axler3.61 \(\text{dim } \mathcal L(V, W) = \left(\text{dim } V\right) \left(\text{dim } V\right)\)}
\label{sec:org65daa15}
\subsection{intuition}
\label{sec:org0649be4}
Not only do two isomorphic spaces have a one to one correspondence between them, that coresspondence is linear which means that they way the elements interact on one side is the same on the other.
\end{document}
