% Created 2021-09-27 Mon 12:02
% Intended LaTeX compiler: xelatex
\documentclass[letterpaper]{article}
\usepackage{graphicx}
\usepackage{grffile}
\usepackage{longtable}
\usepackage{wrapfig}
\usepackage{rotating}
\usepackage[normalem]{ulem}
\usepackage{amsmath}
\usepackage{textcomp}
\usepackage{amssymb}
\usepackage{capt-of}
\usepackage{hyperref}
\setlength{\parindent}{0pt}
\usepackage[margin=1in]{geometry}
\usepackage{fontspec}
\usepackage{svg}
\usepackage{cancel}
\usepackage{indentfirst}
\setmainfont[ItalicFont = LiberationSans-Italic, BoldFont = LiberationSans-Bold, BoldItalicFont = LiberationSans-BoldItalic]{LiberationSans}
\newfontfamily\NHLight[ItalicFont = LiberationSansNarrow-Italic, BoldFont       = LiberationSansNarrow-Bold, BoldItalicFont = LiberationSansNarrow-BoldItalic]{LiberationSansNarrow}
\newcommand\textrmlf[1]{{\NHLight#1}}
\newcommand\textitlf[1]{{\NHLight\itshape#1}}
\let\textbflf\textrm
\newcommand\textulf[1]{{\NHLight\bfseries#1}}
\newcommand\textuitlf[1]{{\NHLight\bfseries\itshape#1}}
\usepackage{fancyhdr}
\pagestyle{fancy}
\usepackage{titlesec}
\usepackage{titling}
\makeatletter
\lhead{\textbf{\@title}}
\makeatother
\rhead{\textrmlf{Compiled} \today}
\lfoot{\theauthor\ \textbullet \ \textbf{2021-2022}}
\cfoot{}
\rfoot{\textrmlf{Page} \thepage}
\renewcommand{\tableofcontents}{}
\titleformat{\section} {\Large} {\textrmlf{\thesection} {|}} {0.3em} {\textbf}
\titleformat{\subsection} {\large} {\textrmlf{\thesubsection} {|}} {0.2em} {\textbf}
\titleformat{\subsubsection} {\large} {\textrmlf{\thesubsubsection} {|}} {0.1em} {\textbf}
\setlength{\parskip}{0.45em}
\renewcommand\maketitle{}
\author{Houjun Liu}
\date{\today}
\title{Escape Velocity}
\hypersetup{
 pdfauthor={Houjun Liu},
 pdftitle={Escape Velocity},
 pdfkeywords={},
 pdfsubject={},
 pdfcreator={Emacs 28.0.50 (Org mode 9.4.4)}, 
 pdflang={English}}
\begin{document}

\tableofcontents



\section{Earth's Gravitatonal Field}
\label{sec:orgdcf5001}
Extends infinitely large, but there is a point whereby it is
imperceivable.

First, \emph{Newton's Law of Universal Gravitation}.

\(\vec{F_g} = -\frac{M_2 M_1 G}{R^2} \cdot \hat{r}\), where \(G\) is the
Universal Gravity Constant and \(r\) is a direction ("unit vector") tha
points from \(M_1\) to \(M_2\) (so \(\hat{r} = \frac{\vec{r}}{r}\)).

Often, this is written as\ldots{}

\(\vec{F_g} = \frac{-G M_1 M_2}{r^3} \vec{r}\). It is \(r^2\), but doing
this and multiplying it to \(\vec{r}\) is to resolve the need to rid if
the direction.

\section{Potential Energy}
\label{sec:orgcb2f604}
\textbf{Whenever we talk about "energy", we are really talking about the CHANGE
in energy.}

Potential Energy Location:

\begin{verbatim}
* ... ... ... \infty
\end{verbatim}
\end{document}
