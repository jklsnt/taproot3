% Created 2021-09-11 Sat 16:41
% Intended LaTeX compiler: xelatex
\documentclass[letterpaper]{article}
\usepackage{graphicx}
\usepackage{grffile}
\usepackage{longtable}
\usepackage{wrapfig}
\usepackage{rotating}
\usepackage[normalem]{ulem}
\usepackage{amsmath}
\usepackage{textcomp}
\usepackage{amssymb}
\usepackage{capt-of}
\usepackage{hyperref}
\usepackage[margin=1in]{geometry}
\usepackage{fontspec}
\usepackage{indentfirst}
\setmainfont[ItalicFont = LiberationSans-Italic, BoldFont = LiberationSans-Bold, BoldItalicFont = LiberationSans-BoldItalic]{LiberationSans}
\newfontfamily\NHLight[ItalicFont = LiberationSansNarrow-Italic, BoldFont       = LiberationSansNarrow-Bold, BoldItalicFont = LiberationSansNarrow-BoldItalic]{LiberationSansNarrow}
\newcommand\textrmlf[1]{{\NHLight#1}}
\newcommand\textitlf[1]{{\NHLight\itshape#1}}
\let\textbflf\textrm
\newcommand\textulf[1]{{\NHLight\bfseries#1}}
\newcommand\textuitlf[1]{{\NHLight\bfseries\itshape#1}}
\usepackage{fancyhdr}
\pagestyle{fancy}
\usepackage{titlesec}
\usepackage{titling}
\makeatletter
\lhead{\textbf{\@title}}
\makeatother
\rhead{\textrmlf{Compiled} \today}
\lfoot{\theauthor\ \textbullet \ \textbf{2021-2022}}
\cfoot{}
\rfoot{\textrmlf{Page} \thepage}
\titleformat{\section} {\Large} {\textrmlf{\thesection} {|}} {0.3em} {\textbf}
\titleformat{\subsection} {\large} {\textrmlf{\thesubsection} {|}} {0.2em} {\textbf}
\titleformat{\subsubsection} {\large} {\textrmlf{\thesubsubsection} {|}} {0.1em} {\textbf}
\setlength{\parskip}{0.45em}
\renewcommand\maketitle{}
\author{Houjun Liu}
\date{\today}
\title{Energy, a review!}
\hypersetup{
 pdfauthor={Houjun Liu},
 pdftitle={Energy, a review!},
 pdfkeywords={},
 pdfsubject={},
 pdfcreator={Emacs 27.2 (Org mode 9.4.4)}, 
 pdflang={English}}
\begin{document}

\maketitle
So let's talk about energy!

\section{Types of Energy}
\label{sec:orgd5c945e}
\begin{itemize}
\item Potential Energy \(PE_{grav}=mgh\) (which is work (force times
distance) for moving stuff up \(\vec{F} \cdot \vec{h}\))
\item Kinetic Energy \(KE_{translational} = \frac{1}{2}mv^2\) +
\(KE_{rotational} = \frac{1}{2}I \omega^2\)
\end{itemize}

Where\ldots{}

\begin{itemize}
\item \(I\): moment of inertia
\item \(\omega\): rotational velocity
\end{itemize}

\section{Work}
\label{sec:org7408b3a}
\(W = \vec{F} \cdot \vec{d}\), where \(\vec{F}\) force and \(\vec{d}\)
change of distance that the force manifest.

=> \(W = |\vec{F}|\cos{\theta} \times |\vec{d}|\)

which, => \(W = |\vec{d}|\cos{\theta} \times |\vec{F}|\)

so, essentially, work is either displacement times parallel as part of
force, or visa versa.

Why?

\subsection{The Dot Product, a review}
\label{sec:orgf045b45}
\subsubsection{What is it}
\label{sec:org766c764}
The Dot product is a measure of the "pararllelity" of \(\vec{F}\) with
\(\vec{D}\).

=> Dot product: the component of the first vector parallel to the second
vector multiplied to the magnitude of d.

\(\vec{A} \cdot \vec{B} = |\vec{A}||\vec{B}|\cos{\theta}\)

\subsubsection{Calculating it}
\label{sec:orgb19cc92}
Given two vectors

\begin{itemize}
\item \(\vec{V_1} = <a_x, a_y, a_z>\)
\item \(\vec{V_2} = <b_x, b_y, b_z>\)
\end{itemize}

The dot product is\ldots{}

\(\vec{V1} \times \vec{V2} = a_x b_x + a_y b_y + a_z b_z\)

\section{Potential Energy}
\label{sec:org885ea16}
Potential energy exists because of a force field. There is an object
"propping" it up pending release of energy.

\subsection{Where did \(\Delta PE = W = mg \Delta h\) come from?}
\label{sec:org8346638}
So, define \(PE = -W_{AB}\). Which is "potential energy of A to B."
Gravity will do a certain amount of work from one point to anther, it
will do the opposite the other way.

\(\Delta PE_g = -W_{AB} = -\vec{F} \cdot \vec{d}\)

\(\Delta PE_g = -((-mg) \cdot \Delta h)\) The negative again! \$ is
\end{document}
