% Created 2021-10-14 Thu 19:31
% Intended LaTeX compiler: xelatex
\documentclass[letterpaper]{article}
\usepackage{graphicx}
\usepackage{grffile}
\usepackage{longtable}
\usepackage{wrapfig}
\usepackage{rotating}
\usepackage[normalem]{ulem}
\usepackage{amsmath}
\usepackage{textcomp}
\usepackage{amssymb}
\usepackage{capt-of}
\usepackage{hyperref}
\setlength{\parindent}{0pt}
\usepackage[margin=1in]{geometry}
\usepackage{fontspec}
\usepackage{svg}
\usepackage{tikz}
\usepackage{cancel}
\usepackage{pgfplots}
\usepackage{indentfirst}
\setmainfont[ItalicFont = LiberationSans-Italic, BoldFont = LiberationSans-Bold, BoldItalicFont = LiberationSans-BoldItalic]{LiberationSans}
\newfontfamily\NHLight[ItalicFont = LiberationSansNarrow-Italic, BoldFont       = LiberationSansNarrow-Bold, BoldItalicFont = LiberationSansNarrow-BoldItalic]{LiberationSansNarrow}
\newcommand\textrmlf[1]{{\NHLight#1}}
\newcommand\textitlf[1]{{\NHLight\itshape#1}}
\let\textbflf\textrm
\newcommand\textulf[1]{{\NHLight\bfseries#1}}
\newcommand\textuitlf[1]{{\NHLight\bfseries\itshape#1}}
\usepackage{fancyhdr}
\usepackage{csquotes}
\pagestyle{fancy}
\usepackage{titlesec}
\usepackage{titling}
\makeatletter
\lhead{\textbf{\@title}}
\makeatother
\rhead{\textrmlf{Compiled} \today}
\lfoot{\theauthor\ \textbullet \ \textbf{2021-2022}}
\cfoot{}
\rfoot{\textrmlf{Page} \thepage}
\renewcommand{\tableofcontents}{}
\titleformat{\section} {\Large} {\textrmlf{\thesection} {|}} {0.3em} {\textbf}
\titleformat{\subsection} {\large} {\textrmlf{\thesubsection} {|}} {0.2em} {\textbf}
\titleformat{\subsubsection} {\large} {\textrmlf{\thesubsubsection} {|}} {0.1em} {\textbf}
\setlength{\parskip}{0.45em}
\renewcommand\maketitle{}
\author{Huxley Marvit}
\date{\today}
\title{Cryptography}
\hypersetup{
 pdfauthor={Huxley Marvit},
 pdftitle={Cryptography},
 pdfkeywords={},
 pdfsubject={},
 pdfcreator={Emacs 28.0.50 (Org mode 9.4.4)}, 
 pdflang={English}}
\begin{document}

\tableofcontents

\#ret \#incomplete \#hw \textbf{*}

\section{Cryptography!}
\label{sec:org7332979}
\subsection{Hashes}
\label{sec:org543acc4}
\subsubsection{Requirements for a hash}
\label{sec:org2d81fa0}
First, how do we know if it works? A hash needs to be: - One way? -
Deterministic - Unique

How we we prove it is one way? uh, we can't. unless we prove P!=NP. hash
function zoo! \url{https://ehash.iaik.tugraz.at/wiki/The\_Hash\_Function\_Zoo}
\subsubsection{-}
\label{sec:orge98b353}
\href{https://stackoverflow.com/questions/2889473/when-is-it-safe-to-use-a-broken-hash-function}{source} -
No preimage: given \emph{y}, it should not be feasible to find \emph{x} such that
\emph{h(x) = y}. - No second preimage: given \emph{x1}, it should not be feasible
to find \emph{x2} (distinct from \emph{x1}) such that \emph{h(x1) = h(x2)}. - No
collision: it should not be feasible to find any \emph{x1} and \emph{x2} (distinct
from each other) such that \emph{h(x1) = h(x2)}.

\begin{itemize}
\item what this means

\begin{itemize}
\item not feasible to get the original from the function output
\item not feasible to find a colliding hash?
\item not feasible to find collisions
\end{itemize}

\item breaking a hash function, from
\href{https://stackoverflow.com/questions/2889473/when-is-it-safe-to-use-a-broken-hash-function}{here}
\end{itemize}

\begin{verbatim}
title: what does it mean for a hash function to be broken?

"For a hash function with a _n_-bit output, there are generic attacks (which work regardless of the details of the hash function) in _2n_ operations for the two first properties, and _2n/2_ operations for the third. If, for a given hash function, an attack is found, which, by exploiting special details of how the hash function operates, finds a preimage, a second preimage or a collision faster than the corresponding generic attack, then the hash function is said to be 'broken.'"
\end{verbatim}

\subsection{Custom hashing function}
\label{sec:org5e88917}
what if\ldots{} we just use a neural network?

create a giant, randomly initialized neural network. then, have
permuting layers in the middle which make the output space
non-continuous

\href{KBxCryptographyRet.org}{KBxCryptographyRet}
\end{document}
