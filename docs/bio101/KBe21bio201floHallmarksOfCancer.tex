% Created 2021-09-11 Sat 09:35
% Intended LaTeX compiler: xelatex
\documentclass[letterpaper]{article}
\usepackage{graphicx}
\usepackage{grffile}
\usepackage{longtable}
\usepackage{wrapfig}
\usepackage{rotating}
\usepackage[normalem]{ulem}
\usepackage{amsmath}
\usepackage{textcomp}
\usepackage{amssymb}
\usepackage{capt-of}
\usepackage{hyperref}
\usepackage[margin=1in]{geometry}
\usepackage{fontspec}
\usepackage{indentfirst}
\setmainfont[ItalicFont = LiberationSans-Italic, BoldFont = LiberationSans-Bold, BoldItalicFont = LiberationSans-BoldItalic]{LiberationSans}
\newfontfamily\NHLight[ItalicFont = LiberationSansNarrow-Italic, BoldFont       = LiberationSansNarrow-Bold, BoldItalicFont = LiberationSansNarrow-BoldItalic]{LiberationSansNarrow}
\newcommand\textrmlf[1]{{\NHLight#1}}
\newcommand\textitlf[1]{{\NHLight\itshape#1}}
\let\textbflf\textrm
\newcommand\textulf[1]{{\NHLight\bfseries#1}}
\newcommand\textuitlf[1]{{\NHLight\bfseries\itshape#1}}
\usepackage{fancyhdr}
\pagestyle{fancy}
\usepackage{titlesec}
\usepackage{titling}
\makeatletter
\lhead{\textbf{\@title}}
\makeatother
\rhead{\textrmlf{Compiled} \today}
\lfoot{\theauthor\ \textbullet \ \textbf{2021-2022}}
\cfoot{}
\rfoot{\textrmlf{Page} \thepage}
\titleformat{\section} {\Large} {\textrmlf{\thesection} {|}} {0.3em} {\textbf}
\titleformat{\subsection} {\large} {\textrmlf{\thesubsection} {|}} {0.2em} {\textbf}
\titleformat{\subsubsection} {\large} {\textrmlf{\thesubsubsection} {|}} {0.1em} {\textbf}
\setlength{\parskip}{0.45em}
\renewcommand\maketitle{}
\author{Exr0n}
\date{\today}
\title{flo - hallmarks of cancer (2011) reading}
\hypersetup{
 pdfauthor={Exr0n},
 pdftitle={flo - hallmarks of cancer (2011) reading},
 pdfkeywords={},
 pdfsubject={},
 pdfcreator={Emacs 27.2 (Org mode 9.4.4)}, 
 pdflang={English}}
\begin{document}

\maketitle
\section{sources\hfill{}\textsc{source}}
\label{sec:org302612a}
\subsection{assignment: \url{https://nuevaschool.instructure.com/courses/3087/assignments/56036}}
\label{sec:org3cb163e}
\subsection{reading: \href{KBsrcHallmarksOfCancer2011Reading.pdf}{Hallmarks of Cancer PDF}}
\label{sec:org655c636}
\section{Flow}
\label{sec:org9dce9c1}
\subsection{Abstract}
\label{sec:org61c8a0f}
\subsubsection{hallmarks include}
\label{sec:orgaf7eee0}
\begin{enumerate}
\item sustaining proliferative signaling
\label{sec:orgd0d75e0}
\item evading growth suppressors
\label{sec:org2e3bfe9}
\item resisting cell death
\label{sec:org090d14b}
\item enabling replicative immortality
\label{sec:orgd4bb396}
\item inducing ingiogenesis
\label{sec:org7ce7f00}
\item activating invasion and metastasis
\label{sec:org358d3e2}
\end{enumerate}
\subsubsection{theese hallmarks are newer}
\label{sec:org6ba5d13}
\begin{enumerate}
\item reprogramming of energy metabolism
\label{sec:org1225a71}
\item evading immune destruction
\label{sec:org3329a0e}
\end{enumerate}
\subsubsection{underlying}
\label{sec:org4ba0847}
\begin{enumerate}
\item genome instability
\label{sec:orgc53bb6a}
\begin{enumerate}
\item genetic diversity that expedites acquisition of hallmarks
\label{sec:org764a783}
\end{enumerate}
\item inflammation
\label{sec:orgc752d6b}
\begin{enumerate}
\item "fosters multiple hallmark functions"
\label{sec:org8dac500}
\end{enumerate}
\end{enumerate}
\subsection{Introduction}
\label{sec:org2a7c46f}
\subsubsection{Cancer cells evolve into cancer cells because they need to be cancer cells??}
\label{sec:org3df21fc}
\begin{enumerate}
\item {\bfseries\sffamily TODO} why do tumors have "the need \ldots{} to acquire the traints that enable them to become tumorigenic and ultimately malignant"?\hfill{}\textsc{question}
\label{sec:org4fe9baa}
\end{enumerate}
\subsubsection{tumors are not simple / idle 'insular masses of proliferating cancer cells'}
\label{sec:org358234e}
\subsubsection{"recruited" normal cells (or 'stromal cells') are active parts of the tumor}
\label{sec:org8f70059}
\subsubsection{'the biology of tumors can no longer be understood simply by enumerating the traits of the cancer cells but instead must encompass the contributions of the "tumor microenvironment" to tumorigenesis.'}
\label{sec:orgc689950}
\subsubsection{purpose is to consider new hallmarks that have been found or note that old ones weren't as general as we thought}
\label{sec:org4c8d423}
\subsection{section: 'An Emerging Hallmark: Evading Immune Destruction'}
\label{sec:org93d0f0e}
\subsubsection{the immune system usually eradicates the 'formation and progression of incipient neoplasias, late-stage tumors, and micrometastases', so why not in these cancers?}
\label{sec:org72a65fe}
\subsubsection{'long standing theory of immune surveillance' -> something went interesting}
\label{sec:org89bee31}
\begin{enumerate}
\item 'cells and tissues are constantly monitored'
\label{sec:orgc9d86ad}
\item surveillance should elim cancer cells before they grow into tumors
\label{sec:org3636dfb}
\item thus, grown tumors have either hid from surveillance or limited the 'extent of immunological killing'
\label{sec:org811d465}
\end{enumerate}
\subsubsection{more cancer in immunocompromised individuals}
\label{sec:orga4ddb37}
\begin{enumerate}
\item but these are virus-induced cancers
\label{sec:org6707493}
\begin{enumerate}
\item so helping these people = reducing viral infilltration
\label{sec:orgf878cd8}
\end{enumerate}
\item other cancers still evade the immune system
\label{sec:org54e081f}
\end{enumerate}
\subsubsection{'genetically engineered mice and clinical epidemeology suggest that the immune system' actually hurts cancer}
\label{sec:orgd775415}
\begin{enumerate}
\item mice that are engineered to lack some immune parts got cancer faster/stronger/more
\label{sec:org0a2f75f}
\begin{enumerate}
\item these guys are important in fighting cancer
\label{sec:org4a4660c}
\begin{enumerate}
\item CD8\^{}+ cytotoxic T lymphocytes (CTLs)
\label{sec:org85a8df0}
\item CD4\^{}+ T\textsubscript{h1} helper T cells
\label{sec:org5299607}
\item natural killer (NK) cells
\label{sec:org159cb20}
\end{enumerate}
\item 'demonstrable increase in tumor incidence'
\label{sec:org15f458e}
\item lacking multiple -> 'more susceptible to cancer development'
\label{sec:org76be7a1}
\item 'both the innate and adaptive cellular arms of the immune system are able to contribute significantly to immune surveillance and thus tumor eradication'\hfill{}\textsc{conclusion}
\label{sec:orgfb17c22}
\end{enumerate}
\end{enumerate}
\subsubsection{transplantation experiments}
\label{sec:org35cdd72}
\begin{enumerate}
\item cancer cells from immunodeficient mice have a bad time in normal mice
\label{sec:org941bd92}
\item cancer cells from normal mice can initiate tumors in both types of hosts
\label{sec:org297ace8}
\item maybe some cancer cells are more easily detected and those would normally die in normal hosts but live in comprimised hosts, but when transplanted they meet a competent immune system and die\hfill{}\textsc{conclusion}
\label{sec:orgf8730b3}
\item Open question: do some carcinogens tend to induce more/less immunogenic cancer cells?\hfill{}\textsc{nextstep}
\label{sec:org40377c4}
\end{enumerate}
\subsubsection{the immune system probably includes antitumoral responses}
\label{sec:orga826cb3}
\begin{enumerate}
\item patients with colon and ovarian tumors who have lots of CTLs and NK cells have better prognosis
\label{sec:org96190e4}
\begin{enumerate}
\item evidence is not as strong for other cancers\hfill{}\textsc{nextstep}
\label{sec:orga207e98}
\end{enumerate}
\item immunosupressed organ recievers got cancer from the donor
\label{sec:orgccdbcaa}
\begin{enumerate}
\item suggests doner had immune system which held cancer down until organ was transplanted
\label{sec:org11d32e3}
\end{enumerate}
\end{enumerate}
\subsubsection{{\bfseries\sffamily TODO} 'still, the epidemiology of chronically immunosupressed patients does not indicate significantly increased incidences of the major forms of nonviral human cancer'}
\label{sec:org2756975}
\subsubsection{{\bfseries\sffamily TODO} something about HIV patients who lack T and B cells and how they should still be able to fight cancer with NK cells and CTLs}
\label{sec:orgb4e94a0}
\subsubsection{that was oversimplified as the tumor might also be actively supressing immune responses}
\label{sec:orge592528}
\begin{enumerate}
\item may 'paralyze infiltrating CTLs and NK cells by secreting TGF-\(\beta\) or other immunosuppressive factors'
\label{sec:orgab39a6d}
\item 'more subtle mechinisms .. recruitment of inflamatory cells that are actively immunosuppressive'
\label{sec:org612da48}
\begin{enumerate}
\item 'regulatory T cells (Tregs) and myeloid-derived suppressor cells (MDSCs)'
\label{sec:org54d939f}
\end{enumerate}
\end{enumerate}
\subsubsection{it is so far unclear whether the immune system plays a large enough role to be considered a hallmark of cancer}
\label{sec:orgd724256}
\section{Vocab}
\label{sec:org933ff09}
\subsection{neoplastic disease}
\label{sec:orgc613282}
\subsubsection{anything that causes tumor growth (malignant or benign)}
\label{sec:org4f48e9f}
\subsection{ostensibly}
\label{sec:org70eba55}
\subsubsection{maybe 'technically'?'}
\label{sec:org362fef7}
\subsection{tumor microenvironment}
\label{sec:orgcc18b0e}
\subsubsection{presumably inflammation, recruited normal cells, and other stuff that helps the tumor grow}
\label{sec:org14d3cca}
\subsection{pathogenisis}
\label{sec:org282f03c}
\subsubsection{evolution of 'pathogen' (cancer)}
\label{sec:orgf304f0d}
\subsection{ancillary proposition}
\label{sec:org41a03df}
\subsubsection{maybe the starting / base proposition}
\label{sec:org1151e4e}
\subsection{insular masses}
\label{sec:orgcedcfc6}
\subsubsection{stagnant or something, simple}
\label{sec:org4c2bf77}
\subsection{heterotypic interactions}
\label{sec:org12c31b5}
\subsubsection{many types of interactions}
\label{sec:org12964db}
\subsection{tumorigenisis}
\label{sec:orgd501212}
\subsubsection{the growth / development of a tumor?}
\label{sec:org2755001}
\subsection{neoplasias}
\label{sec:org356f3df}
\subsubsection{new uncontrolled growth of cells}
\label{sec:orge433805}
\subsection{micrometastases}
\label{sec:orgd9312c3}
\subsubsection{clumps of cancer cells that spread around the body}
\label{sec:org4f6e292}
\subsection{etiology}
\label{sec:orgcc9c1bd}
\subsubsection{study of cancer?}
\label{sec:org511bf9f}
\subsection{immunogenic}
\label{sec:orge6ccd13}
\subsubsection{easily detected by the immune system}
\label{sec:orgcc13340}
\subsection{immunoediting}
\label{sec:orga1a8841}
\subsubsection{"natural selection" by the immune system}
\label{sec:org9548ea9}
\subsection{prognosis}
\label{sec:orgb97933e}
\subsubsection{a prediction of the outcome of a cancer (or disease in general, apparently)}
\label{sec:org5ca0e00}
\end{document}
