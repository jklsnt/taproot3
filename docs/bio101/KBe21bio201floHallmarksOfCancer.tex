% Created 2021-09-11 Sat 08:17
% Intended LaTeX compiler: xelatex
\documentclass[letterpaper]{article}
\usepackage{graphicx}
\usepackage{grffile}
\usepackage{longtable}
\usepackage{wrapfig}
\usepackage{rotating}
\usepackage[normalem]{ulem}
\usepackage{amsmath}
\usepackage{textcomp}
\usepackage{amssymb}
\usepackage{capt-of}
\usepackage{hyperref}
\usepackage[margin=1in]{geometry}
\usepackage{fontspec}
\usepackage{indentfirst}
\setmainfont[ItalicFont = LiberationSans-Italic, BoldFont = LiberationSans-Bold, BoldItalicFont = LiberationSans-BoldItalic]{LiberationSans}
\newfontfamily\NHLight[ItalicFont = LiberationSansNarrow-Italic, BoldFont       = LiberationSansNarrow-Bold, BoldItalicFont = LiberationSansNarrow-BoldItalic]{LiberationSansNarrow}
\newcommand\textrmlf[1]{{\NHLight#1}}
\newcommand\textitlf[1]{{\NHLight\itshape#1}}
\let\textbflf\textrm
\newcommand\textulf[1]{{\NHLight\bfseries#1}}
\newcommand\textuitlf[1]{{\NHLight\bfseries\itshape#1}}
\usepackage{fancyhdr}
\pagestyle{fancy}
\usepackage{titlesec}
\usepackage{titling}
\makeatletter
\lhead{\textbf{\@title}}
\makeatother
\rhead{\textrmlf{Compiled} \today}
\lfoot{\theauthor\ \textbullet \ \textbf{2021-2022}}
\cfoot{}
\rfoot{\textrmlf{Page} \thepage}
\titleformat{\section} {\Large} {\textrmlf{\thesection} {|}} {0.3em} {\textbf}
\titleformat{\subsection} {\large} {\textrmlf{\thesubsection} {|}} {0.2em} {\textbf}
\titleformat{\subsubsection} {\large} {\textrmlf{\thesubsubsection} {|}} {0.1em} {\textbf}
\setlength{\parskip}{0.45em}
\renewcommand\maketitle{}
\author{Exr0n}
\date{\today}
\title{flo - hallmarks of cancer (2011) reading}
\hypersetup{
 pdfauthor={Exr0n},
 pdftitle={flo - hallmarks of cancer (2011) reading},
 pdfkeywords={},
 pdfsubject={},
 pdfcreator={Emacs 27.2 (Org mode 9.4.4)}, 
 pdflang={English}}
\begin{document}

\maketitle
\section{sources\hfill{}\textsc{source}}
\label{sec:org16146f3}
\subsection{assignment: \url{https://nuevaschool.instructure.com/courses/3087/assignments/56036}}
\label{sec:orga7efac2}
\subsection{reading: \href{KBsrcHallmarksOfCancer2011Reading.pdf}{Hallmarks of Cancer PDF}}
\label{sec:org3383bfa}
\section{Flow}
\label{sec:orge6d6eb8}
\subsection{Abstract}
\label{sec:orgb692ea5}
\subsubsection{hallmarks include}
\label{sec:orgd424c80}
\begin{enumerate}
\item sustaining proliferative signaling
\label{sec:org9c59455}
\item evading growth suppressors
\label{sec:org94e7a4c}
\item resisting cell death
\label{sec:orgd48cd7f}
\item enabling replicative immortality
\label{sec:orgaeb64c7}
\item inducing ingiogenesis
\label{sec:org108467e}
\item activating invasion and metastasis
\label{sec:org4ab433b}
\end{enumerate}
\subsubsection{theese hallmarks are newer}
\label{sec:orgd801484}
\begin{enumerate}
\item reprogramming of energy metabolism
\label{sec:org1465bac}
\item evading immune destruction
\label{sec:org290181a}
\end{enumerate}
\subsubsection{underlying}
\label{sec:org94ab572}
\begin{enumerate}
\item genome instability
\label{sec:orgfa11034}
\begin{enumerate}
\item genetic diversity that expedites acquisition of hallmarks
\label{sec:org012e19f}
\end{enumerate}
\item inflammation
\label{sec:org57424f8}
\begin{enumerate}
\item "fosters multiple hallmark functions"
\label{sec:orgbb25953}
\end{enumerate}
\end{enumerate}
\subsection{Introduction}
\label{sec:orgab5a4fd}
\subsubsection{Cancer cells evolve into cancer cells because they need to be cancer cells??}
\label{sec:org536703d}
\begin{enumerate}
\item {\bfseries\sffamily TODO} why do tumors have "the need \ldots{} to acquire the traints that enable them to become tumorigenic and ultimately malignant"?\hfill{}\textsc{question}
\label{sec:orga5ad68b}
\end{enumerate}
\subsubsection{tumors are not simple / idle 'insular masses of proliferating cancer cells'}
\label{sec:org5a542c5}
\subsubsection{"recruited" normal cells (or 'stromal cells') are active parts of the tumor}
\label{sec:orgbc4ba06}
\subsubsection{'the biology of tumors can no longer be understood simply by enumerating the traits of the cancer cells but instead must encompass the contributions of the "tumor microenvironment" to tumorigenesis.'}
\label{sec:org919fed2}
\subsubsection{purpose is to consider new hallmarks that have been found or note that old ones weren't as general as we thought}
\label{sec:org93c8d7c}
\subsection{section: 'An Emerging Hallmark: Evading Immune Destruction'}
\label{sec:org0a8364e}
\subsubsection{the immune system usually eradicates the 'formation and progression of incipient neoplasias, late-stage tumors, and micrometastases', so why not in these cancers?}
\label{sec:orgb129a50}
\subsubsection{'long standing theory of immune surveillance' -> something went interesting}
\label{sec:orga748d9b}
\begin{enumerate}
\item 'cells and tissues are constantly monitored'
\label{sec:org99f65d0}
\item surveillance should elim cancer cells before they grow into tumors
\label{sec:org9516466}
\item thus, grown tumors have either hid from surveillance or limited the 'extent of immunological killing'
\label{sec:org1246982}
\end{enumerate}
\subsubsection{more cancer in immunocompromised individuals}
\label{sec:org70d4067}
\begin{enumerate}
\item but these are virus-induced cancers
\label{sec:org563f48e}
\begin{enumerate}
\item so helping these people = reducing viral infilltration
\label{sec:orgbddbed5}
\end{enumerate}
\item other cancers still evade the immune system
\label{sec:org50c9c6f}
\end{enumerate}
\subsubsection{'genetically engineered mice and clinical epidemeology suggest that the immune system' actually hurts cancer}
\label{sec:org1420c47}
\begin{enumerate}
\item mice that are engineered to lack some immune parts got cancer faster/stronger/more
\label{sec:org5694b24}
\begin{enumerate}
\item these guys are important in fighting cancer
\label{sec:orgfb643fb}
\begin{enumerate}
\item CD8\^{}+ cytotoxic T lymphocytes (CTLs)
\label{sec:org3458e07}
\item CD4\^{}+ T\textsubscript{h1} helper T cells
\label{sec:org22ba2f7}
\item natural killer (NK) cells
\label{sec:orgc71c403}
\end{enumerate}
\item 'demonstrable increase in tumor incidence'
\label{sec:org2461ac4}
\item lacking multiple -> 'more susceptible to cancer development'
\label{sec:orgf274c27}
\item 'both the innate and adaptive cellular arms of the immune system are able to contribute significantly to immune surveillance and thus tumor eradication'\hfill{}\textsc{conclusion}
\label{sec:org8893aad}
\end{enumerate}
\end{enumerate}
\subsubsection{transplantation experiments}
\label{sec:orge4c54a2}
\begin{enumerate}
\item cancer cells from immunodeficient mice have a bad time in normal mice
\label{sec:orge93b391}
\item cancer cells from normal mice can initiate tumors in both types of hosts
\label{sec:org2eb677b}
\item maybe some cancer cells are more easily detected and those would normally die in normal hosts but live in comprimised hosts, but when transplanted they meet a competent immune system and die\hfill{}\textsc{conclusion}
\label{sec:orgf1d1596}
\item Open question: do some carcinogens tend to induce more/less immunogenic cancer cells?\hfill{}\textsc{nextstep}
\label{sec:orgd9c2783}
\end{enumerate}
\subsubsection{the immune system probably includes antitumoral responses}
\label{sec:orgd775c38}
\begin{enumerate}
\item patients with colon and ovarian tumors who have lots of CTLs and NK cells have better prognosis
\label{sec:orgda4aabf}
\begin{enumerate}
\item evidence is not as strong for other cancers\hfill{}\textsc{nextstep}
\label{sec:orge138516}
\end{enumerate}
\item immunosupressed organ recievers got cancer from the donor
\label{sec:org50b41c9}
\begin{enumerate}
\item suggests doner had immune system which held cancer down until organ was transplanted
\label{sec:org5680dc9}
\end{enumerate}
\end{enumerate}
\subsubsection{{\bfseries\sffamily TODO} 'still, the epidemiology of chronically immunosupressed patients does not indicate significantly increased incidences of the major forms of nonviral human cancer'}
\label{sec:orgc326607}
\subsubsection{{\bfseries\sffamily TODO} something about HIV patients who lack T and B cells and how they should still be able to fight cancer with NK cells and CTLs}
\label{sec:org91e4e57}
\subsubsection{that was oversimplified as the tumor might also be actively supressing immune responses}
\label{sec:org064b8c1}
\begin{enumerate}
\item may 'paralyze infiltrating CTLs and NK cells by secreting TGF-\(\beta\) or other immunosuppressive factors'
\label{sec:org1ce3ebc}
\item 'more subtle mechinisms .. recruitment of inflamatory cells that are actively immunosuppressive'
\label{sec:org18d5c28}
\begin{enumerate}
\item 'regulatory T cells (Tregs) and myeloid-derived suppressor cells (MDSCs)'
\label{sec:org0873418}
\end{enumerate}
\end{enumerate}
\subsubsection{it is so far unclear whether the immune system plays a large enough role to be considered a hallmark of cancer}
\label{sec:org2e94f18}
\section{Vocab}
\label{sec:orgf46e5b8}
\subsection{neoplastic disease}
\label{sec:orge84b526}
\subsubsection{anything that causes tumor growth (malignant or benign)}
\label{sec:org1060697}
\subsection{ostensibly}
\label{sec:orgc23898f}
\subsubsection{maybe 'technically'?'}
\label{sec:orga5e0767}
\subsection{tumor microenvironment}
\label{sec:org9914fd7}
\subsubsection{presumably inflammation, recruited normal cells, and other stuff that helps the tumor grow}
\label{sec:org55ae7af}
\subsection{pathogenisis}
\label{sec:org3418617}
\subsubsection{evolution of 'pathogen' (cancer)}
\label{sec:org1fa5a8c}
\subsection{ancillary proposition}
\label{sec:orge7c748a}
\subsubsection{maybe the starting / base proposition}
\label{sec:orga1e329f}
\subsection{insular masses}
\label{sec:orgaba8b4c}
\subsubsection{stagnant or something, simple}
\label{sec:orgb67bc94}
\subsection{heterotypic interactions}
\label{sec:org640bdb8}
\subsubsection{many types of interactions}
\label{sec:org9437e58}
\subsection{tumorigenisis}
\label{sec:org5f1d4d4}
\subsubsection{the growth / development of a tumor?}
\label{sec:orgea5074e}
\subsection{neoplasias}
\label{sec:orgfd08f55}
\subsubsection{new uncontrolled growth of cells}
\label{sec:org4deac06}
\subsection{micrometastases}
\label{sec:orgc6fee97}
\subsubsection{clumps of cancer cells that spread around the body}
\label{sec:org4730c56}
\subsection{etiology}
\label{sec:org1349989}
\subsubsection{study of cancer?}
\label{sec:orgcb16fb3}
\subsection{immunogenic}
\label{sec:org244d4fe}
\subsubsection{easily detected by the immune system}
\label{sec:org3df181c}
\subsection{immunoediting}
\label{sec:org434b33a}
\subsubsection{"natural selection" by the immune system}
\label{sec:orgdb3d07b}
\subsection{prognosis}
\label{sec:org5b26333}
\subsubsection{a prediction of the outcome of a cancer (or disease in general, apparently)}
\label{sec:org58a5267}
\end{document}
