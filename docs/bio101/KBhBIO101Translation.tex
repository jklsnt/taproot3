% Created 2021-09-11 Sat 09:35
% Intended LaTeX compiler: xelatex
\documentclass[letterpaper]{article}
\usepackage{graphicx}
\usepackage{grffile}
\usepackage{longtable}
\usepackage{wrapfig}
\usepackage{rotating}
\usepackage[normalem]{ulem}
\usepackage{amsmath}
\usepackage{textcomp}
\usepackage{amssymb}
\usepackage{capt-of}
\usepackage{hyperref}
\usepackage[margin=1in]{geometry}
\usepackage{fontspec}
\usepackage{indentfirst}
\setmainfont[ItalicFont = LiberationSans-Italic, BoldFont = LiberationSans-Bold, BoldItalicFont = LiberationSans-BoldItalic]{LiberationSans}
\newfontfamily\NHLight[ItalicFont = LiberationSansNarrow-Italic, BoldFont       = LiberationSansNarrow-Bold, BoldItalicFont = LiberationSansNarrow-BoldItalic]{LiberationSansNarrow}
\newcommand\textrmlf[1]{{\NHLight#1}}
\newcommand\textitlf[1]{{\NHLight\itshape#1}}
\let\textbflf\textrm
\newcommand\textulf[1]{{\NHLight\bfseries#1}}
\newcommand\textuitlf[1]{{\NHLight\bfseries\itshape#1}}
\usepackage{fancyhdr}
\pagestyle{fancy}
\usepackage{titlesec}
\usepackage{titling}
\makeatletter
\lhead{\textbf{\@title}}
\makeatother
\rhead{\textrmlf{Compiled} \today}
\lfoot{\theauthor\ \textbullet \ \textbf{2021-2022}}
\cfoot{}
\rfoot{\textrmlf{Page} \thepage}
\titleformat{\section} {\Large} {\textrmlf{\thesection} {|}} {0.3em} {\textbf}
\titleformat{\subsection} {\large} {\textrmlf{\thesubsection} {|}} {0.2em} {\textbf}
\titleformat{\subsubsection} {\large} {\textrmlf{\thesubsubsection} {|}} {0.1em} {\textbf}
\setlength{\parskip}{0.45em}
\renewcommand\maketitle{}
\author{Houjun Liu}
\date{\today}
\title{Protein Translation}
\hypersetup{
 pdfauthor={Houjun Liu},
 pdftitle={Protein Translation},
 pdfkeywords={},
 pdfsubject={},
 pdfcreator={Emacs 27.2 (Org mode 9.4.4)}, 
 pdflang={English}}
\begin{document}

\maketitle


\section{Protein Translation}
\label{sec:org3de419e}
After \href{KBhBIO101DNATranscription.org}{KBhBIO101DNATranscription}
and
\href{KBhBIO101mRNAPreprocessing.pdf.org}{KBhBIO101mRNAPreprocessing.pdf},
the mature mRNA was sent to ribosome. mRNA must travel to the cytoplasm
in the Eukarotes to catch the RNA, whereas in prokarotes they don't have
to go anywhere.

\subsection{Ribosomes}
\label{sec:orgb36cc46}
Ribosomes are the protein devices that takes mRNA and create the actual
sequence of amino acids that are folded together to create a protein.

Ribosomes has two units: 50S unit + 30S unit => they come together
whenever a mRNA needs it. Each contained specialized rRNA + tTRNA to
catalyze attachment of and carry amino acids + adapt the incoming mRNA
respectively.

\subsection{The Actual Process of Translation}
\label{sec:orgfc6d572}
Firstly, a \textbf{Note! The beginning of mRNA is not translated.} There a
portion on the 5' end of the mRNA (starts with AGGAGG) --- about 170
nuclotides in humans, and shorter in bacteria --- that's called UTR
(untranslated region.) This region helps ribosomes bind to it + stablize
the binds.

Basically: smaller ribosome unit grabs the incoming mRNA, larger
facilate the attraction of amino-acid carrying tRNA to the mRNA and
pluck the resulting amino acids on the tRNA to form an amino acid.

\begin{enumerate}
\item 3 protein factors IF1, IF2, IF3 forms a complex for transcription by
binding to a subunit on the ribosome
\item Methionine-carrying tRNA binds to the start of the mRNA, which forms
the initiation complex. This is typically removed after translation
if not coded for (f M-A amino acid pair coded for, methonine removed;
but if M-L pairs coded for, methonine not removed.)
\item A-site: translates mRNA to tRNA --- anti-codon pairs
\item P-site: amino acid dumped from tRNA to the actual chain being built
\item Spent tRNA ejected to the E-site, which is then recycled
\item Catalyst tRNA combines with rRNA to catalyze amino acid peptide bond
\item Each codon (group of 3 units in tRNA), matches a specific
\href{KBhBIO101AminoAcids.org}{KBhBIO101AminoAcids}
\end{enumerate}

After the amino acids are assembled, it's time for
\href{KBe2020bio101refProteinFolding.org}{KBe2020bio101refProteinFolding}.
See also \href{KBhBIO101Proteins.org}{KBhBIO101Proteins} => After the
amino acid sequence is done, shaperones fold proteins, and if its finds
proteins impossible to fold, it flags it using ubiquitin to send to the
garbage
\end{document}
