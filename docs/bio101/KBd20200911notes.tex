% Created 2021-09-11 Sat 09:35
% Intended LaTeX compiler: xelatex
\documentclass[letterpaper]{article}
\usepackage{graphicx}
\usepackage{grffile}
\usepackage{longtable}
\usepackage{wrapfig}
\usepackage{rotating}
\usepackage[normalem]{ulem}
\usepackage{amsmath}
\usepackage{textcomp}
\usepackage{amssymb}
\usepackage{capt-of}
\usepackage{hyperref}
\usepackage[margin=1in]{geometry}
\usepackage{fontspec}
\usepackage{indentfirst}
\setmainfont[ItalicFont = LiberationSans-Italic, BoldFont = LiberationSans-Bold, BoldItalicFont = LiberationSans-BoldItalic]{LiberationSans}
\newfontfamily\NHLight[ItalicFont = LiberationSansNarrow-Italic, BoldFont       = LiberationSansNarrow-Bold, BoldItalicFont = LiberationSansNarrow-BoldItalic]{LiberationSansNarrow}
\newcommand\textrmlf[1]{{\NHLight#1}}
\newcommand\textitlf[1]{{\NHLight\itshape#1}}
\let\textbflf\textrm
\newcommand\textulf[1]{{\NHLight\bfseries#1}}
\newcommand\textuitlf[1]{{\NHLight\bfseries\itshape#1}}
\usepackage{fancyhdr}
\pagestyle{fancy}
\usepackage{titlesec}
\usepackage{titling}
\makeatletter
\lhead{\textbf{\@title}}
\makeatother
\rhead{\textrmlf{Compiled} \today}
\lfoot{\theauthor\ \textbullet \ \textbf{2021-2022}}
\cfoot{}
\rfoot{\textrmlf{Page} \thepage}
\titleformat{\section} {\Large} {\textrmlf{\thesection} {|}} {0.3em} {\textbf}
\titleformat{\subsection} {\large} {\textrmlf{\thesubsection} {|}} {0.2em} {\textbf}
\titleformat{\subsubsection} {\large} {\textrmlf{\thesubsubsection} {|}} {0.1em} {\textbf}
\setlength{\parskip}{0.45em}
\renewcommand\maketitle{}
\date{\today}
\title{}
\hypersetup{
 pdfauthor={},
 pdftitle={},
 pdfkeywords={},
 pdfsubject={},
 pdfcreator={Emacs 27.2 (Org mode 9.4.4)}, 
 pdflang={English}}
\begin{document}

\begin{center}
\begin{tabular}{l}
title: 2020-09-11 Notes\\
course: BIO101\\
author: @dwlg00\\
source: \#index\\
\end{tabular}
\end{center}

\noindent\rule{\textwidth}{0.5pt}

\section{Vocab}
\label{sec:orgda74a1a}
\begin{itemize}
\item "Mer"

\begin{itemize}
\item Monomer

\begin{itemize}
\item "One-Body"
\item e.g. Glucose
\end{itemize}

\item Polymer

\begin{itemize}
\item "Many-Body"
\end{itemize}

\item Hexamer

\begin{itemize}
\item 6 bodies
\end{itemize}
\end{itemize}
\end{itemize}

\section{Cellulose}
\label{sec:orga5808d8}
\begin{itemize}
\item Made up of chains

\begin{itemize}
\item Each chain has identical sections that attach to each other to make
a chain
\end{itemize}

\item Act as a fiber
\end{itemize}

\section{Polymers}
\label{sec:org74b4790}
\begin{itemize}
\item Each unit is a sugar
\item A polymer with n units has more energy than n monomers
\item On average: \textbf{The more units there are, the more energy it contains}

\begin{itemize}
\item Bonds between units store energy

\begin{itemize}
\item Right when the bonds are broken, energy is released

\begin{itemize}
\item Requires energy to break bonds, though
\end{itemize}
\end{itemize}

\item Creation of bonds stores energy

\begin{itemize}
\item How lipids work???
\href{polymer\_energy\_diagram.png.org}{polymer\textsubscript{energy}\textsubscript{diagram.png}}
\end{itemize}
\end{itemize}
\end{itemize}
\end{document}
