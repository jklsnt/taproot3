% Created 2021-09-11 Sat 09:36
% Intended LaTeX compiler: xelatex
\documentclass[letterpaper]{article}
\usepackage{graphicx}
\usepackage{grffile}
\usepackage{longtable}
\usepackage{wrapfig}
\usepackage{rotating}
\usepackage[normalem]{ulem}
\usepackage{amsmath}
\usepackage{textcomp}
\usepackage{amssymb}
\usepackage{capt-of}
\usepackage{hyperref}
\usepackage[margin=1in]{geometry}
\usepackage{fontspec}
\usepackage{indentfirst}
\setmainfont[ItalicFont = LiberationSans-Italic, BoldFont = LiberationSans-Bold, BoldItalicFont = LiberationSans-BoldItalic]{LiberationSans}
\newfontfamily\NHLight[ItalicFont = LiberationSansNarrow-Italic, BoldFont       = LiberationSansNarrow-Bold, BoldItalicFont = LiberationSansNarrow-BoldItalic]{LiberationSansNarrow}
\newcommand\textrmlf[1]{{\NHLight#1}}
\newcommand\textitlf[1]{{\NHLight\itshape#1}}
\let\textbflf\textrm
\newcommand\textulf[1]{{\NHLight\bfseries#1}}
\newcommand\textuitlf[1]{{\NHLight\bfseries\itshape#1}}
\usepackage{fancyhdr}
\pagestyle{fancy}
\usepackage{titlesec}
\usepackage{titling}
\makeatletter
\lhead{\textbf{\@title}}
\makeatother
\rhead{\textrmlf{Compiled} \today}
\lfoot{\theauthor\ \textbullet \ \textbf{2021-2022}}
\cfoot{}
\rfoot{\textrmlf{Page} \thepage}
\titleformat{\section} {\Large} {\textrmlf{\thesection} {|}} {0.3em} {\textbf}
\titleformat{\subsection} {\large} {\textrmlf{\thesubsection} {|}} {0.2em} {\textbf}
\titleformat{\subsubsection} {\large} {\textrmlf{\thesubsubsection} {|}} {0.1em} {\textbf}
\setlength{\parskip}{0.45em}
\renewcommand\maketitle{}
\author{Huxley}
\date{\today}
\title{Mason Chapter Eight}
\hypersetup{
 pdfauthor={Huxley},
 pdftitle={Mason Chapter Eight},
 pdfkeywords={},
 pdfsubject={},
 pdfcreator={Emacs 27.2 (Org mode 9.4.4)}, 
 pdflang={English}}
\begin{document}

\maketitle
\#flo \#ref \#disorganized \#incomplete \#todo: re-read this one.

\noindent\rule{\textwidth}{0.5pt}

\section{Anoda wahn}
\label{sec:org86b099c}
\begin{itemize}
\item ending of 19th century, European powers

\begin{itemize}
\item \begin{quote}
engaged in a competitive struggle to extend their influence around
the world.
\end{quote}
\end{itemize}

\item Large inequality in land ownership
\item scramble for africa occured
\end{itemize}

\subsection{EUROPEAN EXPANSIONISM BEFORE THE NINETEENTH CENTURY}
\label{sec:orgfa5020b}
\begin{itemize}
\item def:

\begin{itemize}
\item \begin{quote}
the process by which one state, with supe- rior military strength
and more advanced technology, imposes its control over the land,
resources, and population of a less developed region.”
\end{quote}
\end{itemize}

\item other stuff
\end{itemize}

\subsection{THE MOTIVATIONS FOR IMPERIALISM}
\label{sec:orgbf299cc}
\begin{itemize}
\item primary motive for imperialist expansion was economic
\item Industrial Rev stimulated the market and such
\item economic depression in Europe from 1873 to mid 1890s
\item tried to do protectionism and put taxes on foreign trade, sending them
deeper into a depression
\item European states formed "sheltered markets" free from the restrictions
\item Imperialism was very expensive, large investments, led to formation of
armys and soldiers
\item imperialism was the final stage of capitalism, according to lenin, and
thus it would lead to the downfall of capitalism
\item and some other stuff?
\end{itemize}

\subsection{THE SCRAMBLE FOR AFRICA}
\label{sec:orgd111c84}
\begin{itemize}
\item roughly 15-20 years
\item bunch of dates and numbers!
\item defeated the boers, led to the creation of the british union
\item british mke the apartheid, a policy for racial seperation which
exploited the black population
\end{itemize}

\subsection{THE COLONIZATION OF ASIA}
\label{sec:org7dac27a}
\begin{itemize}
\item Europe left only Japan alone
\item india had alot of resources

\begin{itemize}
\item cotton, tea, opium
\end{itemize}

\item took control of india
\item china got beat up by japan, russia, and the british
\item (and the US)
\item led to the collapse of the Qing
\end{itemize}

\subsection{PATTERNS OF COLONIAL RULE}
\label{sec:org8440d09}
\begin{itemize}
\item uh
\end{itemize}

\subsection{THE LEGACY AND CONSEQUENCES OF EUROPEAN IMPERIALISM}
\label{sec:orgf9e2244}
\begin{itemize}
\item European contries controled 7\% of the planet by land surface in 1500
\item in 1800, they had 35\%
\item 1914, they got 84\%
\item stuff is now based around europe
\item whenever someone achieved independance, they would normally fail to
establish a working project
\item people hated the europeans
\item this hurt trade and such
\item europe brought ideas of democracy and such to the colonies, which then
strted to want freedom (more?)
\item lost a whole bunch of colonies
\item 
\end{itemize}
\end{document}
