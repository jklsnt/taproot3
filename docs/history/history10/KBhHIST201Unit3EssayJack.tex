% Created 2021-09-12 Sun 22:48
% Intended LaTeX compiler: xelatex
\documentclass[letterpaper]{article}
\usepackage{graphicx}
\usepackage{grffile}
\usepackage{longtable}
\usepackage{wrapfig}
\usepackage{rotating}
\usepackage[normalem]{ulem}
\usepackage{amsmath}
\usepackage{textcomp}
\usepackage{amssymb}
\usepackage{capt-of}
\usepackage{hyperref}
\usepackage[margin=1in]{geometry}
\usepackage{fontspec}
\usepackage{indentfirst}
\setmainfont[ItalicFont = LiberationSans-Italic, BoldFont = LiberationSans-Bold, BoldItalicFont = LiberationSans-BoldItalic]{LiberationSans}
\newfontfamily\NHLight[ItalicFont = LiberationSansNarrow-Italic, BoldFont       = LiberationSansNarrow-Bold, BoldItalicFont = LiberationSansNarrow-BoldItalic]{LiberationSansNarrow}
\newcommand\textrmlf[1]{{\NHLight#1}}
\newcommand\textitlf[1]{{\NHLight\itshape#1}}
\let\textbflf\textrm
\newcommand\textulf[1]{{\NHLight\bfseries#1}}
\newcommand\textuitlf[1]{{\NHLight\bfseries\itshape#1}}
\usepackage{fancyhdr}
\pagestyle{fancy}
\usepackage{titlesec}
\usepackage{titling}
\makeatletter
\lhead{\textbf{\@title}}
\makeatother
\rhead{\textrmlf{Compiled} \today}
\lfoot{\theauthor\ \textbullet \ \textbf{2021-2022}}
\cfoot{}
\rfoot{\textrmlf{Page} \thepage}
\titleformat{\section} {\Large} {\textrmlf{\thesection} {|}} {0.3em} {\textbf}
\titleformat{\subsection} {\large} {\textrmlf{\thesubsection} {|}} {0.2em} {\textbf}
\titleformat{\subsubsection} {\large} {\textrmlf{\thesubsubsection} {|}} {0.1em} {\textbf}
\setlength{\parskip}{0.45em}
\renewcommand\maketitle{}
\author{Houjun Liu}
\date{\today}
\title{Industrial Revolution Essay Planning, Jack}
\hypersetup{
 pdfauthor={Houjun Liu},
 pdftitle={Industrial Revolution Essay Planning, Jack},
 pdfkeywords={},
 pdfsubject={},
 pdfcreator={Emacs 28.0.50 (Org mode 9.4.4)}, 
 pdflang={English}}
\begin{document}

\maketitle


\section{Industrial Revolution Essay}
\label{sec:org5ee42c5}
For Sushu
\href{KBhHIST201Unit3EssayOutlineForSushuJack.org}{KBhHIST201Unit3EssayOutlineForSushuJack}

\subsection{General Information}
\label{sec:org02e7184}
\begin{center}
\begin{tabular}{lll}
Due Date & Topic & Important Documents\\
\hline
Friday, December 18th & 19th Century Transformations & Mason, Bulliet, Ropp, and Gelvin\\
\end{tabular}
\end{center}

\subsection{Prompt}
\label{sec:org211d7ac}
*What factor or factors most strongly determined which states would
become "winners" and "losers" in this time period (1815 to roughly
1900)?*

\begin{itemize}
\item Address several examples
\item Show wins (GB, Japan, Germany, Italy) and show losses (Egypt, India,
China, SEA, Africa)
\end{itemize}

\subsection{Preparation}
\label{sec:org60c0a1e}
\begin{itemize}
\item \textbf{Erosion of tariff barrieriers + widespread free trade ideals} => new
international order

\begin{itemize}
\item vs. Ottomans' promotion of control
\end{itemize}

\item Reliance on artisan and manual manufacturing instead of high-tech IR
stuff => decline! b/c higher tech means more ability to leverage
\href{KBhHIST201IRonWarfare.org}{KBhHIST201IRonWarfare}

\begin{itemize}
\item Germany and Italy used high-tech warfare to get their dominance
(\href{KBhHIST201MasonAndKennedy.org}{KBhHIST201MasonAndKennedy})
\end{itemize}

\item IR created a discrepancy of technology, which is difficult to fight
back from using old-style technology.

\item Winning strategy: nationalism. Loosing strategy: soverignty control
(the Ottomans again) + seperatism
\end{itemize}

\begin{center}
\begin{tabular}{lll}
Concept & Winning Solution & Loosing Solution\\
\hline
Modern warfare & Approaching industrialization and all the new-fangled \href{KBhHIST201IRonWarfare.org}{KBhHIST201IRonWarfare} that it brings, like what happened to the British Royal Navy & how much, or centrol control\\
Trading & Erosion of tariff barriers + widespread free trade ideals (a la European city-states) & Controlling trade (a la the Ottomans' defense developmentalism)\\
State organization & Germany's \href{KBhHIST201GermanicNationalism.org}{KBhHIST201GermanicNationalism} & Dynastarial control a la China; weak emperors => downfall\\
\end{tabular}
\end{center}

\subsubsection{Quotes bin, too!}
\label{sec:orgd818676}
\begin{itemize}
\item "Prussian success was due in large measure to the application of new
technologies to logistics and warfare"
@\href{KBhHIST201MasonCh7.org}{KBhHIST201MasonCh7} AA
\item "Despite a steady reduction in its own numbers after 1815, the Royal
Navy was at some times probably as powerful as the next three or four
navies in actual fighting power."
@\href{KBhHIST201KennedyCh4.org}{KBhHIST201KennedyCh4} BA
\item "The firepower gap, like that which had opened up in industrial
productivity, meant that the leading nations possessed resources fifty
or a hundred times greater than those at the bottom."
@\href{KBhHIST201KennedyCh4.org}{KBhHIST201KennedyCh4} BB
\item "The advanced technology of steam engines and machine-made tools gave
Europe decisive economic and military advantages."
@\href{KBhHIST201KennedyCh4.org}{KBhHIST201KennedyCh4} BC
\item “the steam-driven gunboat meant that European sea power, already
supreme in open waters, could be extended inland. \ldots{} The ironclad
\_Nemesis \ldots{} was a disaster for the defending Chinese forces, which
were easily brushed aside.”
@\href{KBhHIST201KennedyCh4.org}{KBhHIST201KennedyCh4} BD
\item "conditions naturally encouraged long-term commercial and industrial
investment, thereby stimulating the growth of a global economy."
@\href{KBhHIST201KennedyCh4.org}{KBhHIST201KennedyCh4} BE
\item "The erosion of tariff barriers and othermercantilist devices,
together with the widespread propagation ofideas about free trade and
international harmony, suggested that a newinternational order had
arisen," @\href{KBhHIST201KennedyCh4.org}{KBhHIST201KennedyCh4} BF
\item "Several distinct changes facilitated the expansion and transformation
of Britian's overseas empire \ldots{} new policies favored free trade over
mercantilism" @\href{KBhHIST201BritishRaj.org}{KBhHIST201BritishRaj}
(Bulliet) CA
\item "To get around this, European states sought"sheltered markets” free
from such restrictions to trade and found them in the colonies they
estab- lished in Africa and Asia.”
@\href{KBhHIST201MasonCh8.org}{KBhHIST201MasonCh8} DA
\item "Over the course of the nineteenth century, the Ottoman Empire backed
away from \ldots{} free trade that had done so much to integrate it into
the world economy."
@\href{KBhHIST201GelvinChapter5.org}{KBhHIST201GelvinChapter5} EA
\item "Attempts to establish state-run factories floundered \ldots{} because \ldots{}
[of] shortages of skilled labors \ldots{} [and] a lack of investment
capital As a result, the state turned instead to programs that were
intended to foster private industrial production."
@\href{KBhHIST201GelvinChapter5.org}{KBhHIST201GelvinChapter5} EB
\item "Unification projects had the support of powerful states and leaders
in Piedmont and Prussia and, in the Italian case at least, outside
support (from France) as well."
@\href{KBhHIST201MasonCh7.org}{KBhHIST201MasonCh7} FA
\item "Not by speeches and majority votes are the great questions of the day
decided---that was the great error of 1848 and 1849---but by iron and
blood" @\href{KBhHIST201MasonCh7.org}{KBhHIST201MasonCh7}, FB, Otto
Von Bismark
\item "His forceful actions to achieve German unification"
@\href{KBhHIST201MasonCh7.org}{KBhHIST201MasonCh7} FC
\item "statesmen in Europe were using warfare and civic nationalism to forge
\textbf{powerful new nation-states}"
@\href{KBhHIST201MasonCh7.org}{KBhHIST201MasonCh7} FD
\item "When nationalism arises in multinational states or empires \ldots{}
national groups typically want to break away from the larger empire,
which is dominated by other nationalities\ldots{} [this] lead to the
breakup of the Ottoman Empire."
@\href{KBhHIST201MasonCh7.org}{KBhHIST201MasonCh7} FE
\item "But the Qing Empire was a vast, poor, mostly agricultural and
overpopulated territory with a small, weak government, and the
modernization efforts were confined to tiny coastal areas that had
little impact inland."
@\href{KBhHIST101LateQingChina.org}{KBhHIST101LateQingChina} Ropp GA
\item "The court was torn between conserva- tive and reformist officials,
and she maintained her power by alternating appeals to each group,
allowing neither to dominate for long."
@\href{KBhHIST101LateQingChina.org}{KBhHIST101LateQingChina} Ropp GB
\end{itemize}

\subsection{Claim Synthesis}
\label{sec:org5b7517e}
Three aspects: embracing Industrialization-tech based military warfare
that originated in Europe; promoting free trade and market economy;
organizing states under civic top-down nationalism.

\subsubsection{Modern Warfare}
\label{sec:org19729b1}
England, through utilizing
\href{KBhHIST201IRonWarfare.org}{KBhHIST201IRonWarfare}, was able to
gather a huge competitive advantage over other nations.

BC --- new tech made military much better. They did two things: they
made the navy much more economically and physically efficient but also
just made them good ol' fashioned \emph{better}.

Per the BA, the British royal navy was actually able to have a huge
boost in fighting power even while reducing its count during the 1810s.

Such efficiencies is unrivaled even compared to the traditionally large
nations. By embracing the
\href{KBhHIST201IndustrialRevolution.org}{KBhHIST201IndustrialRevolution},
new conquirings are possible that would not have been possible before.
BD against traditionally large nations that simply did not had the speed
to embrace and master
\href{KBhHIST201IRonWarfare.org}{KBhHIST201IRonWarfare}.

\subsubsection{Free Trade and Private Production}
\label{sec:org9e73175}
After the BF Great War caused a disruptions of trade, European nations
started to use and embrace free trade. BE.

In addition to its seapower, British domination could also be attributed
to that of free trade. CA. European nations actively urged the process
of free trade, even setting up colonies (DA) for the basic avoidance of
interruption to free trade.

Sidenote, they also actively weaponized free trade a la opium war. =>
they were to willing back up and support free trade with their stronger
army.

(Colonization is more mercantilist; so instead, talk about trade that is
aligned with industrial needs. "what aspects of European trade was more
successful?")

If a country does not follow the operations of free trade, they are very
sad. See: the Ottomans + Defensive Developmentalism.

The principles of DD encouraged the creation of more centralized control
of both military power and economy based on an anti-european sentiment.
So, the Ottoms tried to EA --- back away from trade. That was a bad
idea: EB --- their centralized control failed and they evenually turned
back to relying on the free market + private production.

\subsubsection{\href{KBhHIST201Nationalism.org}{KBhHIST201Nationalism}, namely}
\label{sec:orgf842de3}
Civic Nationalism
:CUSTOM\textsubscript{ID}: filekbhhist201nationalism.orgkbhhist201nationalism-namely-civic-nationalism
Powerful Nation-States were created by nationalism. FD

\href{KBhHIST201GermanicNationalism.org}{KBhHIST201GermanicNationalism}
worked! FA --- countries like Italy and Germany supported the efforts of
Nationalism.

Although not strictly a nationalist speech, FB --- Iron and Blood ---
showed clear nationalist sentiments. he executed FC --- forceful actions
--- that ensured the unification of Germany by consolidating Prussian
power with those nearby: rallying under the image of Germanic civic
nationalism.

When a country does not do nationalism, it does not go well.

Either\ldots{}

\begin{enumerate}
\item A country tries to enforce dynastarial control over disperate regions
instead of an identity centered around nationalism, things go wrong.
FE --- \emph{seperatism} is engendered, which does not promote the
national identity definition to national ideals w/ common culture.
This is party how the Ottomans went.

\item A country tries to give up control and have weaker, local governance.
This means that important national efforts like in China's case for
westernization GA could have much lesser effect than if there were to
be a stronger central government with shared \textbf{ideals}, which would
safeguard against GB Cixi-style trading favourites and politiking
\end{enumerate}

**For a nation-state to achieve global success and dominance in the 19th
century, they must leverage the forces of industrial modernization ---
bringing increased fighting power of an industrialized modern military,
the economic benefits of transitioning from mercantilism to free trade
and market economy, and the centralized political control offered by
top-down civic nationalism.”

\noindent\rule{\textwidth}{0.5pt}

There is always
\href{https://wp.ucla.edu/wp-content/uploads/2016/01/UWC\_handouts\_What-How-So-What-Thesis-revised-5-4-15-RZ.pdf}{UCLA
Writing Lab}
\end{document}
