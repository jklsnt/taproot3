% Created 2021-10-14 Thu 19:58
% Intended LaTeX compiler: xelatex
\documentclass[letterpaper]{article}
\usepackage{graphicx}
\usepackage{grffile}
\usepackage{longtable}
\usepackage{wrapfig}
\usepackage{rotating}
\usepackage[normalem]{ulem}
\usepackage{amsmath}
\usepackage{textcomp}
\usepackage{amssymb}
\usepackage{capt-of}
\usepackage{hyperref}
\setlength{\parindent}{0pt}
\usepackage[margin=1in]{geometry}
\usepackage{fontspec}
\usepackage{svg}
\usepackage{tikz}
\usepackage{cancel}
\usepackage{pgfplots}
\usepackage{indentfirst}
\setmainfont[ItalicFont = LiberationSans-Italic, BoldFont = LiberationSans-Bold, BoldItalicFont = LiberationSans-BoldItalic]{LiberationSans}
\newfontfamily\NHLight[ItalicFont = LiberationSansNarrow-Italic, BoldFont       = LiberationSansNarrow-Bold, BoldItalicFont = LiberationSansNarrow-BoldItalic]{LiberationSansNarrow}
\newcommand\textrmlf[1]{{\NHLight#1}}
\newcommand\textitlf[1]{{\NHLight\itshape#1}}
\let\textbflf\textrm
\newcommand\textulf[1]{{\NHLight\bfseries#1}}
\newcommand\textuitlf[1]{{\NHLight\bfseries\itshape#1}}
\usepackage{fancyhdr}
\usepackage{csquotes}
\pagestyle{fancy}
\usepackage{titlesec}
\usepackage{titling}
\makeatletter
\lhead{\textbf{\@title}}
\makeatother
\rhead{\textrmlf{Compiled} \today}
\lfoot{\theauthor\ \textbullet \ \textbf{2021-2022}}
\cfoot{}
\rfoot{\textrmlf{Page} \thepage}
\renewcommand{\tableofcontents}{}
\titleformat{\section} {\Large} {\textrmlf{\thesection} {|}} {0.3em} {\textbf}
\titleformat{\subsection} {\large} {\textrmlf{\thesubsection} {|}} {0.2em} {\textbf}
\titleformat{\subsubsection} {\large} {\textrmlf{\thesubsubsection} {|}} {0.1em} {\textbf}
\setlength{\parskip}{0.45em}
\renewcommand\maketitle{}
\author{Houjun Liu}
\date{\today}
\title{MVC Vectors Add}
\hypersetup{
 pdfauthor={Houjun Liu},
 pdftitle={MVC Vectors Add},
 pdfkeywords={},
 pdfsubject={},
 pdfcreator={Emacs 28.0.50 (Org mode 9.4.4)}, 
 pdflang={English}}
\begin{document}

\tableofcontents

Take, for instance, problem \texttt{e}. From taking two partial derivatives in the \(x\) and \(y\) dimensions, we deduce that the partial derivative values are\ldots{}

\begin{equation}
    \frac{\partial f}{\partial x} = 10 
\end{equation}

\begin{equation}
    \frac{\partial f}{\partial y} = 10 \sqrt{3}
\end{equation}

We could, therefore, treat these terms as two separate vectors lying at the \(x\) and \(y\) directions. That is, we know that the multidimentional "slope" of the function could be represented by a combination of vectors\ldots{}

\begin{equation}
\left\{\begin{pmatrix}10 \\ 0 \end{pmatrix},\begin{pmatrix}0 \\ 10\sqrt{3} \end{pmatrix} \right\}
\end{equation}

The "slope" created by the two slope values at a \(60^{\circ}\) angle is essential the sums of the two partial derivative vectors projected at \(60^{\circ}\). Hence, we have to project the two vectors' magnitudes to a shared \(60^{\circ}\) angle, and sum it up.

We first note that, to project the two \emph{orthogonal} vectors to the same, shared "60-degrees" direction, we must project one vector to \(60^{\circ}\) and the other to \((90-60)^{\circ}\) to actually result in the projections' alignment.

Conventionally, we will project the x-direction vector to \(60^{\circ}\). and the y-direction vector to \((90-60)^{\circ}\), but the 60 degree direction that we aim to share is actual arbitrary.

To perform the actual magnitude projection, we perform the follows.

\begin{align}
    let\ \vec{X} &= \begin{pmatrix}10\\0\end{pmatrix} \\
    \vec{Y} &= \begin{pmatrix}0\\10\sqrt{3}\end{pmatrix} \\
 \vec{X_p} &= \|X\|cos(60^{\circ}) \\ 
 &= 10 \times \frac{1}{2} \\
 &= 5 \\
 \vec{Y_p} &= \|Y\|cos((90-60)^{\circ}) \\
 &= \|Y\|sin(60^{\circ}) \\
 &= 10\sqrt{3}\times\frac{\sqrt{3}}{2} \\
 &= \frac{30}{2} = 15
\end{align}

Finally, the sum of slopes in that shared direction would therefore be \(20\).
\end{document}
