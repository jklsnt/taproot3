% Created 2021-09-11 Sat 16:43
% Intended LaTeX compiler: xelatex
\documentclass[letterpaper]{article}
\usepackage{graphicx}
\usepackage{grffile}
\usepackage{longtable}
\usepackage{wrapfig}
\usepackage{rotating}
\usepackage[normalem]{ulem}
\usepackage{amsmath}
\usepackage{textcomp}
\usepackage{amssymb}
\usepackage{capt-of}
\usepackage{hyperref}
\usepackage[margin=1in]{geometry}
\usepackage{fontspec}
\usepackage{indentfirst}
\setmainfont[ItalicFont = LiberationSans-Italic, BoldFont = LiberationSans-Bold, BoldItalicFont = LiberationSans-BoldItalic]{LiberationSans}
\newfontfamily\NHLight[ItalicFont = LiberationSansNarrow-Italic, BoldFont       = LiberationSansNarrow-Bold, BoldItalicFont = LiberationSansNarrow-BoldItalic]{LiberationSansNarrow}
\newcommand\textrmlf[1]{{\NHLight#1}}
\newcommand\textitlf[1]{{\NHLight\itshape#1}}
\let\textbflf\textrm
\newcommand\textulf[1]{{\NHLight\bfseries#1}}
\newcommand\textuitlf[1]{{\NHLight\bfseries\itshape#1}}
\usepackage{fancyhdr}
\pagestyle{fancy}
\usepackage{titlesec}
\usepackage{titling}
\makeatletter
\lhead{\textbf{\@title}}
\makeatother
\rhead{\textrmlf{Compiled} \today}
\lfoot{\theauthor\ \textbullet \ \textbf{2021-2022}}
\cfoot{}
\rfoot{\textrmlf{Page} \thepage}
\titleformat{\section} {\Large} {\textrmlf{\thesection} {|}} {0.3em} {\textbf}
\titleformat{\subsection} {\large} {\textrmlf{\thesubsection} {|}} {0.2em} {\textbf}
\titleformat{\subsubsection} {\large} {\textrmlf{\thesubsubsection} {|}} {0.1em} {\textbf}
\setlength{\parskip}{0.45em}
\renewcommand\maketitle{}
\author{Houjun Liu}
\date{\today}
\title{The Limit Notation, Single and Double Sided Limits}
\hypersetup{
 pdfauthor={Houjun Liu},
 pdftitle={The Limit Notation, Single and Double Sided Limits},
 pdfkeywords={},
 pdfsubject={},
 pdfcreator={Emacs 27.2 (Org mode 9.4.4)}, 
 pdflang={English}}
\begin{document}

\maketitle


\section{The Limit Notation}
\label{sec:org43812d3}
\subsection{Single-Sided Limits}
\label{sec:org0959e92}
\definition["What is $y$ approaching when $x$ approaches $a$ from the right ($+$)?"]{Right Single-Sided Limit}\{\(\lim_{x\to a^+} f(x)\)\}
\definition["What is $y$ approaching when $x$ approaches $a$ from the left ($-$)?"]{Left Single-Sided Limit}\{\(\lim_{x\to a^-} f(x)\)\}
\textbf{Watch!} If both the left and right single-sided limit exists and is the
same, the Double-Sided Limit exists.

\subsection{Double-sided Limits}
\label{sec:orgefb6068}
\definition["What is \(y\) approaching when \(x\) approaches \(a\)?" This exists only if \(\lim_{x\to a^-} f(x)=\lim_{x\to a^+} f(x)\) ]\{Left Single-Sided Limit\}\{\(\lim_{x\to a} f(x)\)\}
\textbf{Vocab!} When the Double-Sided Limit does not exist, it is called \emph{DOES
NOT EXIST!}. It is not! \(\cancel{undefined}\)
\end{document}
