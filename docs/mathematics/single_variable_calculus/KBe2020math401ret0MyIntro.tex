% Created 2021-09-12 Sun 22:50
% Intended LaTeX compiler: xelatex
\documentclass[letterpaper]{article}
\usepackage{graphicx}
\usepackage{grffile}
\usepackage{longtable}
\usepackage{wrapfig}
\usepackage{rotating}
\usepackage[normalem]{ulem}
\usepackage{amsmath}
\usepackage{textcomp}
\usepackage{amssymb}
\usepackage{capt-of}
\usepackage{hyperref}
\usepackage[margin=1in]{geometry}
\usepackage{fontspec}
\usepackage{indentfirst}
\setmainfont[ItalicFont = LiberationSans-Italic, BoldFont = LiberationSans-Bold, BoldItalicFont = LiberationSans-BoldItalic]{LiberationSans}
\newfontfamily\NHLight[ItalicFont = LiberationSansNarrow-Italic, BoldFont       = LiberationSansNarrow-Bold, BoldItalicFont = LiberationSansNarrow-BoldItalic]{LiberationSansNarrow}
\newcommand\textrmlf[1]{{\NHLight#1}}
\newcommand\textitlf[1]{{\NHLight\itshape#1}}
\let\textbflf\textrm
\newcommand\textulf[1]{{\NHLight\bfseries#1}}
\newcommand\textuitlf[1]{{\NHLight\bfseries\itshape#1}}
\usepackage{fancyhdr}
\pagestyle{fancy}
\usepackage{titlesec}
\usepackage{titling}
\makeatletter
\lhead{\textbf{\@title}}
\makeatother
\rhead{\textrmlf{Compiled} \today}
\lfoot{\theauthor\ \textbullet \ \textbf{2021-2022}}
\cfoot{}
\rfoot{\textrmlf{Page} \thepage}
\titleformat{\section} {\Large} {\textrmlf{\thesection} {|}} {0.3em} {\textbf}
\titleformat{\subsection} {\large} {\textrmlf{\thesubsection} {|}} {0.2em} {\textbf}
\titleformat{\subsubsection} {\large} {\textrmlf{\thesubsubsection} {|}} {0.1em} {\textbf}
\setlength{\parskip}{0.45em}
\renewcommand\maketitle{}
\author{Exr0n}
\date{\today}
\title{My Intro}
\hypersetup{
 pdfauthor={Exr0n},
 pdftitle={My Intro},
 pdfkeywords={},
 pdfsubject={},
 pdfcreator={Emacs 28.0.50 (Org mode 9.4.4)}, 
 pdflang={English}}
\begin{document}

\maketitle
I think a good math problem is one that takes you somewhere and reveals
something useful and elegant. Jana talks about math as a game, where we
define the set of rules based on what makes the game fun or useful. I
love this analogy because it reminds me of a particular set of rules
that are dear to me, the rules surrounding computational complexity
analysis.

In that universe, "constant factors" aka coefficients don't matter, and
we simply drop them when presenting an answer. Although this may feel
wrong at first glance, it makes lots of sense in the one place where
it's used--because we needed a way to evaluate algorithms based on not
exact speed but how complex they were, since the exact amount of time or
number of steps required might change as hardware improves or becomes
abstract. My favorite example of the beauty of this has to do with a
data structure called disjoint set union. It is a lazy propagation
structure that stores temporary data to remember what sets have been
merged. The proof for the complexity of this algorithm takes advantage
of figuring out how much we expect the structure to grow, and end up
with it being roughly constant time. I would also make a note about how
Djikstra's time complexity somewhat hijacks the system by taking
advantage of \(log\)'s power to coefficient manipulations, but this is
getting kind of long. My point is, a good math problem/project is a set
of rules that do something interesting, elegant, and useful, especially
if you might not have expected it.

As you may have gathered from the previous paragraph, I am excited about
computers and algorithms. I also enjoy mountain biking and fiddling with
my workflow.

\noindent\rule{\textwidth}{0.5pt}
\end{document}
