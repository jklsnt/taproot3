% Created 2021-09-12 Sun 22:50
% Intended LaTeX compiler: xelatex
\documentclass[letterpaper]{article}
\usepackage{graphicx}
\usepackage{grffile}
\usepackage{longtable}
\usepackage{wrapfig}
\usepackage{rotating}
\usepackage[normalem]{ulem}
\usepackage{amsmath}
\usepackage{textcomp}
\usepackage{amssymb}
\usepackage{capt-of}
\usepackage{hyperref}
\usepackage[margin=1in]{geometry}
\usepackage{fontspec}
\usepackage{indentfirst}
\setmainfont[ItalicFont = LiberationSans-Italic, BoldFont = LiberationSans-Bold, BoldItalicFont = LiberationSans-BoldItalic]{LiberationSans}
\newfontfamily\NHLight[ItalicFont = LiberationSansNarrow-Italic, BoldFont       = LiberationSansNarrow-Bold, BoldItalicFont = LiberationSansNarrow-BoldItalic]{LiberationSansNarrow}
\newcommand\textrmlf[1]{{\NHLight#1}}
\newcommand\textitlf[1]{{\NHLight\itshape#1}}
\let\textbflf\textrm
\newcommand\textulf[1]{{\NHLight\bfseries#1}}
\newcommand\textuitlf[1]{{\NHLight\bfseries\itshape#1}}
\usepackage{fancyhdr}
\pagestyle{fancy}
\usepackage{titlesec}
\usepackage{titling}
\makeatletter
\lhead{\textbf{\@title}}
\makeatother
\rhead{\textrmlf{Compiled} \today}
\lfoot{\theauthor\ \textbullet \ \textbf{2021-2022}}
\cfoot{}
\rfoot{\textrmlf{Page} \thepage}
\titleformat{\section} {\Large} {\textrmlf{\thesection} {|}} {0.3em} {\textbf}
\titleformat{\subsection} {\large} {\textrmlf{\thesubsection} {|}} {0.2em} {\textbf}
\titleformat{\subsubsection} {\large} {\textrmlf{\thesubsubsection} {|}} {0.1em} {\textbf}
\setlength{\parskip}{0.45em}
\renewcommand\maketitle{}
\author{Exr0n}
\date{\today}
\title{Groups}
\hypersetup{
 pdfauthor={Exr0n},
 pdftitle={Groups},
 pdfkeywords={},
 pdfsubject={},
 pdfcreator={Emacs 28.0.50 (Org mode 9.4.4)}, 
 pdflang={English}}
\begin{document}

\maketitle


\section{Groups}
\label{sec:org2ca14a5}
\begin{itemize}
\item definition

\begin{itemize}
\item closed

\begin{itemize}
\item if \(a, b \in S\) then \(a + b \in S\)
\end{itemize}

\item has an identity \(e\)

\begin{itemize}
\item \(e + a = a + e = a\)
\end{itemize}

\item each element has an inverse

\begin{itemize}
\item \(-a + a = a + -a = e\)
\end{itemize}

\item needs to be associative

\begin{itemize}
\item \((a + b) + c\) = \(a + (b + c)\)
\end{itemize}
\end{itemize}

\item communitivity is nice but not required

\begin{itemize}
\item \(a + b\) = \(b + a\)
\end{itemize}

\item Which number systems are groups under addition and multiplication?
\end{itemize}

\begin{center}
\begin{tabular}{lll}
Number System & Multiplication & Addition\\
\hline
Natural Numbers & No inverse & No identity\\
Whole Numbers & No inverse & No inverse\\
Integers & No inverse & Yes\\
Rationals & Yes* & Yes\\
Reals & Yes* & Yes\\
Complex Numbers & Yes* & Yes\\
\end{tabular}
\end{center}

\textbf{Zero doesn't have an inverse, so it usually gets dropped. For example,
Q} is Q w/o zero \#todo-exr0n: rewrite in latex say \$\$

\begin{itemize}
\item \href{SRC20200825135700.png.org}{SRC20200825135700.png} $\backslash$[
\begin{bmatrix}
8 &2 \\
-2 &0
\end{bmatrix}
$\backslash$]
\end{itemize}

\noindent\rule{\textwidth}{0.5pt}
\end{document}
