% Created 2021-09-11 Sat 16:42
% Intended LaTeX compiler: xelatex
\documentclass[letterpaper]{article}
\usepackage{graphicx}
\usepackage{grffile}
\usepackage{longtable}
\usepackage{wrapfig}
\usepackage{rotating}
\usepackage[normalem]{ulem}
\usepackage{amsmath}
\usepackage{textcomp}
\usepackage{amssymb}
\usepackage{capt-of}
\usepackage{hyperref}
\usepackage[margin=1in]{geometry}
\usepackage{fontspec}
\usepackage{indentfirst}
\setmainfont[ItalicFont = LiberationSans-Italic, BoldFont = LiberationSans-Bold, BoldItalicFont = LiberationSans-BoldItalic]{LiberationSans}
\newfontfamily\NHLight[ItalicFont = LiberationSansNarrow-Italic, BoldFont       = LiberationSansNarrow-Bold, BoldItalicFont = LiberationSansNarrow-BoldItalic]{LiberationSansNarrow}
\newcommand\textrmlf[1]{{\NHLight#1}}
\newcommand\textitlf[1]{{\NHLight\itshape#1}}
\let\textbflf\textrm
\newcommand\textulf[1]{{\NHLight\bfseries#1}}
\newcommand\textuitlf[1]{{\NHLight\bfseries\itshape#1}}
\usepackage{fancyhdr}
\pagestyle{fancy}
\usepackage{titlesec}
\usepackage{titling}
\makeatletter
\lhead{\textbf{\@title}}
\makeatother
\rhead{\textrmlf{Compiled} \today}
\lfoot{\theauthor\ \textbullet \ \textbf{2021-2022}}
\cfoot{}
\rfoot{\textrmlf{Page} \thepage}
\titleformat{\section} {\Large} {\textrmlf{\thesection} {|}} {0.3em} {\textbf}
\titleformat{\subsection} {\large} {\textrmlf{\thesubsection} {|}} {0.2em} {\textbf}
\titleformat{\subsubsection} {\large} {\textrmlf{\thesubsubsection} {|}} {0.1em} {\textbf}
\setlength{\parskip}{0.45em}
\renewcommand\maketitle{}
\author{Exr0n}
\date{\today}
\title{Invertibility of Linear Maps}
\hypersetup{
 pdfauthor={Exr0n},
 pdftitle={Invertibility of Linear Maps},
 pdfkeywords={},
 pdfsubject={},
 pdfcreator={Emacs 27.2 (Org mode 9.4.4)}, 
 pdflang={English}}
\begin{document}

\maketitle
\section{invertible, inverse\hfill{}\textsc{def}}
\label{sec:orgc729992}
\begin{quote}
\begin{itemize}
\item A linear map \(T \in \mathcal L(V, W)\) is \emph{invertible} if there exists a linear map \(S\in \mathcal(W, V)\) such that \(ST\) equals the identity map on \(V\) and \(TS\) equals the identity map on \(W\).
\item A linear map \(S \in \mathcal(W, V)\) satisfying \(ST = I\) and \(TS = I\) is called an \emph{inverse} of \(T\)
\item If \(T\) is invertable, \(T^{-1}\) denotes the inverse of \(T\)
\end{itemize}
\end{quote}
\subsection{careful}
\label{sec:org5a980cd}
\subsubsection{the inverse of a map has to be commutative (\(TS = I\) and \(ST = I\))}
\label{sec:orgda2f20f}
\subsubsection{the target identity is in one space on one side and in the other space on the other side}
\label{sec:orgb78a840}
\subsection{results}
\label{sec:org113d95c}
\subsubsection{unique}
\label{sec:orgf4ad899}
any invertible map has exactly one inverse
\subsubsection{equivalant to injectivity and surjectivity (bijectivity)}
\label{sec:org1043db8}
See \href{KBrefBijectiveFunction.org}{bijectivity}. Iff a map is bijective, then it is invertable.
\subsubsection{Equivalent Condition with eigenvalues}
\label{sec:orgf2fa67a}
if a map has zero as an eigenvalue, then it is singular (5.A exercise 21)
\subsubsection{non-singular matrices are invertible}
\label{sec:orga8940ff}
\subsubsection{operators that are injective or surjective are bijective}
\label{sec:org8ca87ce}
\subsubsection{matrices with linearly independent columns and rows are bijective}
\label{sec:org07438b6}
\end{document}
