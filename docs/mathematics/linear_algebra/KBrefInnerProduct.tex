% Created 2021-09-27 Mon 11:53
% Intended LaTeX compiler: xelatex
\documentclass[letterpaper]{article}
\usepackage{graphicx}
\usepackage{grffile}
\usepackage{longtable}
\usepackage{wrapfig}
\usepackage{rotating}
\usepackage[normalem]{ulem}
\usepackage{amsmath}
\usepackage{textcomp}
\usepackage{amssymb}
\usepackage{capt-of}
\usepackage{hyperref}
\setlength{\parindent}{0pt}
\usepackage[margin=1in]{geometry}
\usepackage{fontspec}
\usepackage{svg}
\usepackage{cancel}
\usepackage{indentfirst}
\setmainfont[ItalicFont = LiberationSans-Italic, BoldFont = LiberationSans-Bold, BoldItalicFont = LiberationSans-BoldItalic]{LiberationSans}
\newfontfamily\NHLight[ItalicFont = LiberationSansNarrow-Italic, BoldFont       = LiberationSansNarrow-Bold, BoldItalicFont = LiberationSansNarrow-BoldItalic]{LiberationSansNarrow}
\newcommand\textrmlf[1]{{\NHLight#1}}
\newcommand\textitlf[1]{{\NHLight\itshape#1}}
\let\textbflf\textrm
\newcommand\textulf[1]{{\NHLight\bfseries#1}}
\newcommand\textuitlf[1]{{\NHLight\bfseries\itshape#1}}
\usepackage{fancyhdr}
\pagestyle{fancy}
\usepackage{titlesec}
\usepackage{titling}
\makeatletter
\lhead{\textbf{\@title}}
\makeatother
\rhead{\textrmlf{Compiled} \today}
\lfoot{\theauthor\ \textbullet \ \textbf{2021-2022}}
\cfoot{}
\rfoot{\textrmlf{Page} \thepage}
\renewcommand{\tableofcontents}{}
\titleformat{\section} {\Large} {\textrmlf{\thesection} {|}} {0.3em} {\textbf}
\titleformat{\subsection} {\large} {\textrmlf{\thesubsection} {|}} {0.2em} {\textbf}
\titleformat{\subsubsection} {\large} {\textrmlf{\thesubsubsection} {|}} {0.1em} {\textbf}
\setlength{\parskip}{0.45em}
\renewcommand\maketitle{}
\author{Taproot}
\date{\today}
\title{Inner Product}
\hypersetup{
 pdfauthor={Taproot},
 pdftitle={Inner Product},
 pdfkeywords={},
 pdfsubject={},
 pdfcreator={Emacs 28.0.50 (Org mode 9.4.4)}, 
 pdflang={English}}
\begin{document}

\tableofcontents

\section{inner product\hfill{}\textsc{def}}
\label{sec:org14b6a6d}
\begin{quote}
An \emph{inner product} on \(V\) is a function that takes each ordered pair \((u, v)\) of elements of \(V\) to a number \(\langle u, v \rangle \in \FF\)  and has the following properties
\begin{itemize}
\item \textbf{positivity} \(\langle v, v \rangle \geq  0 \forall v\in V\)
\item \textbf{definiteness} \(\langle v, v \rangle = 0 \iff v = 0\)
\item \textbf{additivity in first slot} \(\langle u+v, w \rangle = \langle u, w \rangle + \langle v, w \rangle \forall u, v, w, \in V\)
\item \textbf{homogeneity in first slot} \(\langle \lambda u, v \rangle = \lambda \langle u,v \rangle \forall \lambda \in \FF, u,v \in  V\)
\item \textbf{conjugate symmetry} \(\langle u,v \rangle = \overline{\langle v,u \rangle} \forall u,v \in V\)
\begin{itemize}
\item Over the reals, this is equal to \(\langle u,v \rangle = \langle v, u \rangle\)
\end{itemize}
\end{itemize}
\end{quote}
\section{motivation}
\label{sec:org3781096}
\subsection{The norm of a complex number \(\lVert z \rVert\) should be non-negative, so we can define it as}
\label{sec:org904a56d}
\[\begin{aligned}
   \lVert z \rVert =\sqrt{|z_1|^2 + \cdots + |z_n| ^2}
   \end{aligned}\]
Since the square of the absolute value is just a complex number times a conjugate, and because the norm squared should be the inner product of \(z\) with itself, maybe the inner product of \(w, z \in  \CC ^n\) should equal
\[\begin{aligned}
   w_1 \overline{z_1} + \cdots + w_n \overline{z_n}
   \end{aligned}\]
\subsection{positivity: we want inner product to be the size of the vector, so it should be a positive and real number}
\label{sec:org19ff17d}
\subsection{notation}
\label{sec:org6771b6d}
For a complex scalar \(\lambda \in \CC\), \(\lambda \geq 0\) means \(\lambda\) is real and non-negative

\(\langle u, v \rangle\) denotes an inner product.
\section{intuition}
\label{sec:orga47fec1}
\subsection{additivity/homogeneity in the first slot also implies additivity in the second slot, and 'conjugate homogeneity in the second slot'}
\label{sec:orgebb4af2}
\subsection{we want the norm to be a real scalar, so we need to take the complex conjugate of the second one}
\label{sec:orga8c0487}
\subsubsection{so, the Euclidean inner product is conjugate the second, then dot product}
\label{sec:orgd9610de}
\[\begin{aligned}
    \langle u, v \rangle = u \overline{z}
	\end{aligned}\]
\end{document}
