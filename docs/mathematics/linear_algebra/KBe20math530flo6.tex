% Created 2021-09-12 Sun 22:50
% Intended LaTeX compiler: xelatex
\documentclass[letterpaper]{article}
\usepackage{graphicx}
\usepackage{grffile}
\usepackage{longtable}
\usepackage{wrapfig}
\usepackage{rotating}
\usepackage[normalem]{ulem}
\usepackage{amsmath}
\usepackage{textcomp}
\usepackage{amssymb}
\usepackage{capt-of}
\usepackage{hyperref}
\usepackage[margin=1in]{geometry}
\usepackage{fontspec}
\usepackage{indentfirst}
\setmainfont[ItalicFont = LiberationSans-Italic, BoldFont = LiberationSans-Bold, BoldItalicFont = LiberationSans-BoldItalic]{LiberationSans}
\newfontfamily\NHLight[ItalicFont = LiberationSansNarrow-Italic, BoldFont       = LiberationSansNarrow-Bold, BoldItalicFont = LiberationSansNarrow-BoldItalic]{LiberationSansNarrow}
\newcommand\textrmlf[1]{{\NHLight#1}}
\newcommand\textitlf[1]{{\NHLight\itshape#1}}
\let\textbflf\textrm
\newcommand\textulf[1]{{\NHLight\bfseries#1}}
\newcommand\textuitlf[1]{{\NHLight\bfseries\itshape#1}}
\usepackage{fancyhdr}
\pagestyle{fancy}
\usepackage{titlesec}
\usepackage{titling}
\makeatletter
\lhead{\textbf{\@title}}
\makeatother
\rhead{\textrmlf{Compiled} \today}
\lfoot{\theauthor\ \textbullet \ \textbf{2021-2022}}
\cfoot{}
\rfoot{\textrmlf{Page} \thepage}
\titleformat{\section} {\Large} {\textrmlf{\thesection} {|}} {0.3em} {\textbf}
\titleformat{\subsection} {\large} {\textrmlf{\thesubsection} {|}} {0.2em} {\textbf}
\titleformat{\subsubsection} {\large} {\textrmlf{\thesubsubsection} {|}} {0.1em} {\textbf}
\setlength{\parskip}{0.45em}
\renewcommand\maketitle{}
\author{Exr0n}
\date{\today}
\title{Linalg Flo 6}
\hypersetup{
 pdfauthor={Exr0n},
 pdftitle={Linalg Flo 6},
 pdfkeywords={},
 pdfsubject={},
 pdfcreator={Emacs 28.0.50 (Org mode 9.4.4)}, 
 pdflang={English}}
\begin{document}

\maketitle
\#flo

\section{Talking about the reading (vector spaces)}
\label{sec:org92e526a}
\subsection{Vector space}
\label{sec:orgcf448ea}
\subsubsection{Identity}
\label{sec:orgd208116}
\begin{itemize}
\item It would be the additive identity, because the multiplicitive one
doesn't count because multiply doesn't take two elements from the same
field \#\#\#\# Operations
\item Scalar multiplication

\begin{itemize}
\item Not a multiplication on \(V\)
\item We need another field of scalars
\item Fundamental difference: \textbf{operates on different objects} (only
happens on scalar multiplications)
\end{itemize}

\item addition \#\#\#\# Linearity
\item Something that's linear means "things work for addition and scalar
multiplication"
\item Take \(-2x+1y=3\)

\begin{itemize}
\item Multiplying by scalars
\item adding them
\item similar to a line in standard form--slope stays constant
\end{itemize}

\item Take \(2x-3y+1z=2\)

\begin{itemize}
\item a plane in 3d
\item if you pick a direction, the slope stays the same
\item thus, a plane is linear \#\#\#\# Vector
\end{itemize}

\item Something in a vector space
\item inifinite lists

\begin{itemize}
\item It's like decimals, except you can chose any number instead of just
[0-9]
\item base infinity basically
\end{itemize}

\item Most common vector space

\begin{itemize}
\item \(\mathbb{F}^n\), like \(\mathbb{R}^3\) (might also be
\(\mathbb{C}^2\) or something, although that's hard to visualize)
\item \#definition canonical

\begin{itemize}
\item something "standard", basically everyone should know what you are
talking about
\item canonical vector space is \(\mathbb{R}^2\) \#\#\#\# Distributive
property
\end{itemize}
\end{itemize}

\item Important to tie operations together
\end{itemize}

\subsubsection{Vector Space as a Set of Functions}
\label{sec:org52eb50f}
\begin{itemize}
\item like \(\mathbb{R}^{[0, 1]}\): the functions from \([0, 1]\) that end
up as real numbers

\begin{itemize}
\item Identity = \(f(x) = 0\) \#\#\#\# Subspaces
\end{itemize}

\item A subspace of this has to be a group on it's own
\item Conditions for a subspace

\begin{itemize}
\item See 1.34
\item Just check

\begin{itemize}
\item additive identity
\item closed under addition
\item closed under scalar multiplication
\end{itemize}
\end{itemize}

\item What other subspaces of this vector space are there that also have a
domain from \([0, 1]\)?

\begin{itemize}
\item Like continuous functions from zero to one
\item functions who's derivatives are continuous or constant or zero
\item even functions are also a subspace
\href{KBe20math530srcEvenFunctionsAreSubspacesOfFtotheS.png.org}{KBe20math530srcEvenFunctionsAreSubspacesOfFtotheS.png}
\end{itemize}

\item Subspaces of \(\mathbb{F}^3\)

\begin{itemize}
\item Most contain infinite vectors (except \(\{ 0 \}\))
\item \(\begin{bmatrix}x\\y\\0\end{bmatrix}\) is a subspace with infinite
vectors \#\#\#\# Notation
\end{itemize}

\item \#note \(\mathbb{F}^2\) is almost always either \(\mathbb{R}^2\) or
\(\mathbb{C}^2\), mostly \(\mathbb{R}^2\)
\end{itemize}

\subsection{Direct sums}
\label{sec:org7e24e8a}
\begin{itemize}
\item Something that isn't a direct sum

\begin{itemize}
\item in \(\mathbb{R}^3\), \(\begin{bmatrix}x\\y\\0\end{bmatrix}\) and
\(\begin{bmatrix}x\\x\\0\end{bmatrix}\)

\begin{itemize}
\item Two ways to write \(0\):

\begin{itemize}
\item \(\begin{bmatrix}0\\0\\0\end{bmatrix} + \begin{bmatrix}0\\0\\0\\\end{bmatrix} = \begin{bmatrix}0\\0\\0\end{bmatrix} = \begin{bmatrix}1\\1\\0\end{bmatrix} + \begin{bmatrix}-1\\-1\\0\end{bmatrix}\)
\#\#\# \(\mathbb{F}^\infty\)
\end{itemize}
\end{itemize}
\end{itemize}

\item Functions from naturals to your field, (assign an element to each
natural)

\begin{itemize}
\item that would be the same as ordering the elements in your field?
\item Tons of functions, any one is an infinite vector??
\end{itemize}
\end{itemize}

\section{If and Only If proofs (iff)}
\label{sec:orgd5b673d}
\begin{itemize}
\item You have to take the proof in both directions
\item \textbf{Assumption}: "now suppose the only way to write 0 as a sum of u\textsubscript{1} +
\ldots{} | u\textsubscript{m}, where each u\textsubscript{j} is in U\textsubscript{j}, is by taking each u\textsubscript{j} equal to 0"

\begin{itemize}
\item Assume the red part, then show the green part. Then, assume the
green and show it gets the red.
\item \href{KBe20math530srcIfOnlyIfProofs.png.org}{KBe20math530srcIfOnlyIfProofs.png}
\end{itemize}

\item \#future geometrical interpretation of determinants
\end{itemize}

\noindent\rule{\textwidth}{0.5pt}
\end{document}
