% Created 2021-09-27 Mon 11:53
% Intended LaTeX compiler: xelatex
\documentclass[letterpaper]{article}
\usepackage{graphicx}
\usepackage{grffile}
\usepackage{longtable}
\usepackage{wrapfig}
\usepackage{rotating}
\usepackage[normalem]{ulem}
\usepackage{amsmath}
\usepackage{textcomp}
\usepackage{amssymb}
\usepackage{capt-of}
\usepackage{hyperref}
\setlength{\parindent}{0pt}
\usepackage[margin=1in]{geometry}
\usepackage{fontspec}
\usepackage{svg}
\usepackage{cancel}
\usepackage{indentfirst}
\setmainfont[ItalicFont = LiberationSans-Italic, BoldFont = LiberationSans-Bold, BoldItalicFont = LiberationSans-BoldItalic]{LiberationSans}
\newfontfamily\NHLight[ItalicFont = LiberationSansNarrow-Italic, BoldFont       = LiberationSansNarrow-Bold, BoldItalicFont = LiberationSansNarrow-BoldItalic]{LiberationSansNarrow}
\newcommand\textrmlf[1]{{\NHLight#1}}
\newcommand\textitlf[1]{{\NHLight\itshape#1}}
\let\textbflf\textrm
\newcommand\textulf[1]{{\NHLight\bfseries#1}}
\newcommand\textuitlf[1]{{\NHLight\bfseries\itshape#1}}
\usepackage{fancyhdr}
\pagestyle{fancy}
\usepackage{titlesec}
\usepackage{titling}
\makeatletter
\lhead{\textbf{\@title}}
\makeatother
\rhead{\textrmlf{Compiled} \today}
\lfoot{\theauthor\ \textbullet \ \textbf{2021-2022}}
\cfoot{}
\rfoot{\textrmlf{Page} \thepage}
\renewcommand{\tableofcontents}{}
\titleformat{\section} {\Large} {\textrmlf{\thesection} {|}} {0.3em} {\textbf}
\titleformat{\subsection} {\large} {\textrmlf{\thesubsection} {|}} {0.2em} {\textbf}
\titleformat{\subsubsection} {\large} {\textrmlf{\thesubsubsection} {|}} {0.1em} {\textbf}
\setlength{\parskip}{0.45em}
\renewcommand\maketitle{}
\author{Taproot}
\date{\today}
\title{Norm}
\hypersetup{
 pdfauthor={Taproot},
 pdftitle={Norm},
 pdfkeywords={},
 pdfsubject={},
 pdfcreator={Emacs 28.0.50 (Org mode 9.4.4)}, 
 pdflang={English}}
\begin{document}

\tableofcontents

\section{norm, \(\lVert x \rVert\)\hfill{}\textsc{def}}
\label{sec:orga81081a}


\begin{quote}
For some \(x \in \RR ^n\),
\[\begin{aligned}
  \lVert x \rVert =\sqrt{x_1^2 + \cdots + x_n^2}
  \end{aligned}\]
\end{quote}
Using the definition of an \href{KBrefInnerProduct.org}{inner product}, norms can be defined for complex vectors in \href{KBrefInnerProductSpaces.org}{inner product spaces}

\begin{quote}
For \(v \in  V\), the \emph{norm} of \(v\), denoted \(\lVert v \rVert\), is defined by
\[\begin{aligned}
  \lVert v \rVert =\sqrt{\langle v, v \rangle}
  \end{aligned}\]
\end{quote}
\section{properties}
\label{sec:org29caba3}
\subsection{\(\lVert v \rVert = 0 \iff  v=0\)}
\label{sec:org68ea444}
\subsection{\(\lVert \lambda v \rVert = |\lambda | \lVert v \rVert\) for all \(\lambda \in  \FF\)}
\label{sec:org2fff2c7}
\section{aka euclidean distance}
\label{sec:org6d29e7b}
\section{not linear, so we use the \href{KBrefDotProduct.org}{dot product} to 'inject linearity'}
\label{sec:org199dec4}
\end{document}
