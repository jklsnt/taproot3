% Created 2021-09-12 Sun 22:49
% Intended LaTeX compiler: xelatex
\documentclass[letterpaper]{article}
\usepackage{graphicx}
\usepackage{grffile}
\usepackage{longtable}
\usepackage{wrapfig}
\usepackage{rotating}
\usepackage[normalem]{ulem}
\usepackage{amsmath}
\usepackage{textcomp}
\usepackage{amssymb}
\usepackage{capt-of}
\usepackage{hyperref}
\usepackage[margin=1in]{geometry}
\usepackage{fontspec}
\usepackage{indentfirst}
\setmainfont[ItalicFont = LiberationSans-Italic, BoldFont = LiberationSans-Bold, BoldItalicFont = LiberationSans-BoldItalic]{LiberationSans}
\newfontfamily\NHLight[ItalicFont = LiberationSansNarrow-Italic, BoldFont       = LiberationSansNarrow-Bold, BoldItalicFont = LiberationSansNarrow-BoldItalic]{LiberationSansNarrow}
\newcommand\textrmlf[1]{{\NHLight#1}}
\newcommand\textitlf[1]{{\NHLight\itshape#1}}
\let\textbflf\textrm
\newcommand\textulf[1]{{\NHLight\bfseries#1}}
\newcommand\textuitlf[1]{{\NHLight\bfseries\itshape#1}}
\usepackage{fancyhdr}
\pagestyle{fancy}
\usepackage{titlesec}
\usepackage{titling}
\makeatletter
\lhead{\textbf{\@title}}
\makeatother
\rhead{\textrmlf{Compiled} \today}
\lfoot{\theauthor\ \textbullet \ \textbf{2021-2022}}
\cfoot{}
\rfoot{\textrmlf{Page} \thepage}
\titleformat{\section} {\Large} {\textrmlf{\thesection} {|}} {0.3em} {\textbf}
\titleformat{\subsection} {\large} {\textrmlf{\thesubsection} {|}} {0.2em} {\textbf}
\titleformat{\subsubsection} {\large} {\textrmlf{\thesubsubsection} {|}} {0.1em} {\textbf}
\setlength{\parskip}{0.45em}
\renewcommand\maketitle{}
\date{\today}
\title{}
\hypersetup{
 pdfauthor={},
 pdftitle={},
 pdfkeywords={},
 pdfsubject={},
 pdfcreator={Emacs 28.0.50 (Org mode 9.4.4)}, 
 pdflang={English}}
\begin{document}

\textbf{**}--- title: How do we know that we arn't wrong? author: Houjun Liu
source: KBISOSMasterIndex course: ISOS101 ---

\section{HDWDNWANW}
\label{sec:org6f075a9}
\#flo \#disorganized

\begin{itemize}
\item Prominent scientists could be wrong!

\begin{itemize}
\item Consensus formed during the 20th century about a long of scientific
discovery, for most thought that the important questions have been
answered
\item So, consensus does not mean correctness
\end{itemize}

\item Climate science unusual because of political motivations
\item One way to test hypothesis is to do a review of the state of that
field

\begin{itemize}
\item This was originally trivial, but gets much harder nowadays
\item Too many papers published for one to read efficiently
\end{itemize}

\item Now, Knowledge = Scientific Consensus => only over the simple
\emph{realities} of a hypothesis

\begin{itemize}
\item Claims with scientific consensus are rounded on verified new
realities
\item Claims of current causes is not prediction of the future
\end{itemize}

\item So, why do people think that people disagree on scientifically confirm
consensus?

\begin{itemize}
\item People are conflating scientific evidence with political decisions
\item Climate science heavily predicated upon future effects, which is not
always easy and effective
\item Scientists have sometimes failed to explain themselves beyond their
communities

\begin{itemize}
\item Actually, scientists sometimes thought that the mere worry about
dissemination is wasting time
\item "Popularizers" are dismissed
\end{itemize}

\item Scientists commenting on contested issue often called "politicizing"
\item Organization sometimes propergating alternative views
\end{itemize}

\item How do we know that we arn't wrong?

\begin{itemize}
\item There is actually no singular scientific method!
\item No one answer and standard method of science
\item Scientists use a variety of methods \& philosophers proposed various
helpful criteria:

\begin{itemize}
\item Inductive and deductive reasoning => generalizing from examples
"100 white swans means that all swans are white. 10000 white
swans? I am more sure now"
\item Hypo-deductive model => proving hypothesis "if I wash my hands
after doing an autopsy, I won't hurt babies"

\begin{itemize}
\item Easy to get trapped in "affirming the consequent" => autopsies
don't cause died babies, dead people germs do. However, people
for a while seriously thought that cadaveric matter in and of
itself causes dead babies/dead people.
\end{itemize}

\item Falsificationism => you could never proof something true; you
could only prove it false
\item Consillience of Evidence => look for multiple independent lines of
evidence that allow a fact to be shown
\item Inference to the best explanation => getting a lot of evidence and
choosing from it simply the "best available."
\end{itemize}
\end{itemize}
\end{itemize}
\end{document}
